\documentclass{article}
\usepackage{fullpage}
\usepackage{graphicx}
\usepackage{amsmath}
\usepackage{mathtools}
\usepackage{xcolor}
\usepackage{bm}
\usepackage{natbib}
\usepackage[hidelinks]{hyperref}
\usepackage{doi}
\usepackage{charter}
\usepackage[bitstream-charter]{mathdesign}
\usepackage[final,babel]{microtype}
\usepackage[utf8]{inputenc}
\usepackage[british]{babel}
\usepackage{csquotes}
\usepackage[T1]{fontenc}
\usepackage{siunitx}
\usepackage[font={small}]{caption}


\title{Multidimensional advection over steep terrain on a new type of Cartesian mesh \\ \TODO{(working title)}}
\author{James Shaw}

\newcommand{\iu}{{i\mkern1mu}}
\newcommand{\iunit}{\boldsymbol{\hat \imath}}
\newcommand{\junit}{\boldsymbol{\hat \jmath}}
\newcommand{\kunit}{\boldsymbol{\hat k}}
\newcommand{\TODO}[1]{\textcolor{purple}{TODO: \emph{#1}}}

\begin{document}
\maketitle

\section{Introduction}

First, we present a multidimensional advection scheme that is computationally cheap and suitable for complex flows on arbitrary meshes.  Second, we present a new type of Cartesian mesh, the slanted cell mesh, that avoids the small cell problem associated with cut cell meshes.   We apply the advection scheme to tests over steep orography and show that accurate results are obtained on slanted cell meshes.  Finally, we challenge the multidimensional advection scheme using a test of deformational flow on a geodesic mesh.


\section{Multidimensional advection scheme}

The advection of a dependent variable $\phi$ is given by the conservation equation
\begin{align}
	\frac{\partial \phi}{\partial t} + \nabla \cdot \left( \mathbf{u} \phi \right) = 0 \label{eqn:advection}
\end{align}
where $\mathbf{u}$ is a prescribed wind field.  The time derivative is discretised using a three-stage, second-order Runge-Kutta scheme:
\begin{subequations}
\begin{align}
	\phi^\star &= \phi^{(n)} + \Delta t f(\phi^{(n)}) \\
	\phi^{\star\star} &= \phi^{(n)} + \frac{\Delta t}{2} \left( f(\phi^{(n)}) + f(\phi^\star) \right) \\
	\phi^{(n+1)} &= \phi^{(n)} + \frac{\Delta t}{2} \left( f(\phi^{(n)}) + f(\phi^{\star\star}) \right)
\end{align}
\end{subequations}
where \(f(\phi^{(n)}) = - \nabla \cdot (\mathbf{u} \phi^{(n)})\) at time level \(n\).

Using the finite volume method, the wind field is prescribed at face centroids and the dependent variable is stored at cell centroids.  The divergence term in equation~\eqref{eqn:advection} is discretised using Gauss's theorem:
\begin{align}
	\nabla \cdot \left( \mathbf{u} \phi \right) \approx \frac{1}{\mathcal{V}_c} \sum_{f \in c} \mathbf{u}_f \cdot \mathbf{S}_f \phi_F
\end{align}
where $\mathcal{V}_c$ is the cell volume, $\mathbf{u}_f$ is a wind vector prescribed at a face, ${\mathbf{S}_f}$ is the surface area vector with a direction outward normal to the face and a magnitude equal to the face area, and $\sum_{f \in c}$ denotes a summation over all faces $f$ belonging to cell $c$.  The value of the dependent variable at the face, $\phi_F$, is approximated by a least squares fit over a stencil of surrounding cell centre values.

\begin{figure}
	\centering
	\includegraphics{../fig-upwind-stencil/fig-upwind-stencil.pdf}
	\caption{An upwind-biased stencil on a two-dimensional rectangular grid.  The stencil is used to fit a multidimensional polynomial to twelve cell centred values, $\phi_c$, marked by grey circles, in order to approximate the value $\phi_F$ at the face centroid marked by an open circle.  $\phi_u$ and $\phi_d$ are the values at the centroids of the upwind and downwind cells neighbouring the target face, drawn with a heavy line.  The wind vector $\mathbf{u}_f$ is prescribed at face $f$ and determines the choice of stencil at each timestep.}
	\label{fig:interiorQuadStencil}
\end{figure}

To introduce the approximation method, we will consider how an approximate value is calculated for a face in the interior of a two-dimensional uniform rectangular mesh.  For any mesh, every interior face connects two adjacent cells.  The wind direction at the face determines which of the two adjacent cells is the upwind cell.  Since the stencil is upwind-biased, two stencils must be constructed for every interior face, and the appropriate stencil is chosen for each face depending on the wind direction at that face for every timestep.

The upwind-biased stencil for a face $f$ is shown in figure~\ref{fig:interiorQuadStencil}.  The wind at the face, $\mathbf{u}_f$, is blowing from the upwind cell $c_u$ to the downwind cell $c_d$.
To obtain an approximate value at $f$, a polynomial least squares fit is calculated using the stencil values.
The stencil has \num{4} points in $x$ and \num{3} points in $y$, leading to a natural choice of polynomial that is cubic in $x$ and quadratic in $y$:
\begin{align}
	\phi = a_1 + a_2 x + a_3 y + a_4 x^2 + a_5 xy + a_6 y^2 + a_7 x^3 + a_8 x^2 y + a_9 x y^2
\end{align}
A least squares approach is needed because the system of equations is overdetermined, with \num{12} stencil values but only \num{9} polynomial terms.  If the stencil geometry is expressed in a local coordinate system with the face centroid as the origin, then the approximated value $\phi_F$ is equal to the constant term $a_1$.

The remainder of this section generalises the the approximation technique for arbitrary meshes, explaining the methods for constructing stencils, choosing the terms of the polynomial, and ensuring numerical stability of the advection scheme.

\subsection{Stencil construction}
For every interior face, two stencils are constructed, one for each of the possible upwind cells.  For a given face $f$ and upwind cell $c_u$, we find those faces that are connected to $c_u$ and `oppose' face $f$.  These are called the \textit{opposing faces}.
The opposing faces for face $f$ and upwind cell $c_u$ are determined as follows.
Defining $G$ to be the set of other faces bounding cell $c_u$, we calculate the `opposedness', $\mathrm{Opp}$, between faces $f$ and $g \in G$, defined as
\begin{align}
	\mathrm{Opp}(f, g) \equiv - \frac{\mathbf{S}_f \cdot \mathbf{S}_g}{|\mathbf{S}_f|^2} \label{eqn:opp}
\end{align}
where $\mathbf{S}_f$ and $\mathbf{S}_g$ are the surface area vectors pointing outward from cell $c_u$ for faces $f$ and $g$ respectively.
Using the fact that $\mathbf{a} \cdot \mathbf{b} = |\mathbf{a}|\:|\mathbf{b}| \cos(\theta)$ we can rewrite equation~\eqref{eqn:opp} as
\begin{align}
	\mathrm{Opp}(f, g) = - \frac{|\mathbf{S}_g|}{|\mathbf{S}_f|} \cos(\theta)
\end{align}
where $\theta$ is the angle between faces $f$ and $g$.  In this form, it can be seen that $\mathrm{Opp}$ is a measure of the area of $g$ and how closely it parallels face $f$.

The set of opposing faces, $\mathrm{OF}$, is a subset of $G$, comprising those faces with $\mathrm{Opp} \geq 0.5$, and the face with the maximum opposedness.  Expressed in set notation, this is
\begin{align}
	\mathrm{OF}(f,c_u) \equiv \{ g : \mathrm{Opp}(f, g) \geq 0.5 \} \cup \{ g : \max(\mathrm{Opp}(f, g)) \} 
\end{align}
On a two-dimensional rectangular mesh, there is always one opposing face that is exactly parallel to the face $f$.

Once the opposing faces have been determined, the set of internal and external cells must be found.  The \textit{internal cells} are those cells that are connected to the opposing faces.  Note that $c_u$ is always an internal cell.  The \textit{external cells} are those cells that share vertices with the internal cells.  Note that $c_d$ is always an external cell.  Having found these two sets of cells, the stencil is constructed to comprise all internal and external cells.

\begin{figure}
	\centering
	\includegraphics{../fig-double-upwind-stencil/fig-double-upwind-stencil.pdf}
	%
	\caption{A thirteen-cell, upwind-biased stencil for face $f$ connecting the pentagonal upwind cell, $c_u$, and the downwind cell $c_d$.  The dashed lines denote the two faces of cell $c_u$ that oppose $f$, and black circles mark the centroids of the internal cells that are connected to these two opposing faces.  The stencil is extended outwards by including cells that share vertices with the three internal cells, where black squares mark these vertices.  The local coordinate system $(x, y)$ has its origin at the centroid of face $f$, marked by an open circle, with $x$ normal to $f$ and $y$ perpendicular.}
	\label{fig:double-upwind-stencil}
\end{figure}

Figure~\ref{fig:double-upwind-stencil} illustrates a stencil construction for face $f$ connecting upwind cell $c_u$ and downwind cell $c_d$.  The two opposing faces are denoted by thick dashed lines and the centres of the three adjoining internal cells are marked by black circles.  The stencil is extended outwards by including the external cells that share vertices with the internal cells, marked by black squares.  The resultant stencil contains 13 cells.


\subsection{Polynomial generation}
% generating candidates
% full rank check

\subsection{Stabilisation procedure}
% stability constraints
% reweighting

Stability constriants:
\begin{align}
	0.5 \leq u \leq 1 \\
	0 \leq d \leq 0.5 \\
	u - d \geq \max(|p|)
\end{align}

\begin{figure}
	\includegraphics[width=\textwidth]{stencilConstruction.png}
	\caption{\TODO{example stencils in interior of quad and hex meshes, and example stencil near boundary of a slanted cell mesh (taken from one of the test cases)}}
\end{figure}
\end{document}



\section{Results}

\subsection{Deformational flow}
Following \citep{lauritzen2012}.  Tests on cubed sphere, hex and triangular meshes, again comparing cubicUpwindCPCFit against linearUpwind.  \citet{lauritzen2012} had six classes of test:
\begin{enumerate}
	\item \textbf{convergence tests with Gaussian hills}
	\item \textbf{minimal resolution test with cosine bell}
	\item filament preservation
	\item "rough" distribution with a slotted cylinder
	\item correlation preservation
	\item \textbf{cosine bell in divergent flow}
\end{enumerate}
We will reproduce 1, 2 and 6, highlighted in bold.

\begin{figure}
	\includegraphics[width=\textwidth]{hexCubUpEvolution.png}
	\caption{\TODO{evolution of deformational flow test case with plots at $t=0$, $t=T/2$ and $t=T$.  The analytic solution at $t=T$ is identical to the initial condition.  This figure is supposed to give a sense of what `should' happen, so plot at a high resolution using whichever mesh gives better results.}}
\end{figure}

\begin{figure}
	\includegraphics{../fig-deformationSphere-convergence/fig-deformationSphere-convergence.pdf}
	\caption{\TODO{deformational flow l2 and linf convergence plots comparing cubed sphere, hex and tri, cubicUpwindCPCFit and linearUpwind}}
\end{figure}

% what do other people plot/analyse when they do this test?
% lauritzen2014 plot T/2 for the slotted cylinder at 1.5 and 0.75 degrees

\begin{figure}
	\includegraphics{../fig-solidBodySphere-convergence/fig-solidBodySphere-convergence.pdf}
	\caption{\TODO{solid body l2 and linf convergence plots comparing cubed sphere, hex and tri, cubicUpwindCPCFit and linearUpwind}}
\end{figure}

\subsection{Mountain advection}


\begin{figure}
	\includegraphics{../fig-mountainAdvection-convergence/fig-mountainAdvection-convergence.pdf}
	\caption{\TODO{mountain advection l2 and linf convergence plots comparing btf, slanted cells and cut cells, cubicUpwindCPCFit and linearUpwind}}
\end{figure}

\section{Conclusions}

The advection scheme is
\begin{itemize}
	\item suitable for complex flows on arbitrary meshes
	\item computationally cheap at runtime, with more expensive computations depending only on the mesh geometry
	\item fourth-order convergent at best, first-order convergent at worst
	\item stable for Courant numbers up to 1
\end{itemize}

\section{Acknowledgements}
\TODO{Supervisors, funding bodies.  ASAM group for the mesh generator---I should ask permission to use cut cell meshes in this paper.  Dr Tristan Pryer.  Dr Shing Hing Man.}

\section*{Appendix: One-dimensional von Neumann stability analysis}
Two analyses are performed in order to find stability constraints on the weights $\mathbf{w} = \mathbf{\tilde{b}_1^+} \cdot \mathbf{m}$ as appear in equation~\eqref{eqn:weightedPinv}.  The first analysis uses two points to derived separate constraints on the upwind weight $w_u$ and downwind weight $w_d$.  The second analysis uses three points to derive a constraint that considers all weights in a stencil.

\subsection*{Two-point analysis}
We start with the conservation equation for a dependent variable $\phi$ that is discrete-in-space and continuous-in-time
\begin{align}
\frac{\partial \phi_j}{\partial t} &= - u \frac{\phi_R - \phi_L}{\Delta x} \label{eqn:advectionLR} \\
%
\shortintertext{where the left and right fluxes, $\phi_L$ and $\phi_R$ are weighted averages of the neighbouring points.  Assuming that $u$ is positive}
%
\phi_L &= \alpha_u \phi_{j-1} + \alpha_d \phi_j \\
\phi_R &= \beta_u \phi_j + \beta_d \phi_{j+1}
\end{align}
where $\alpha_u$ and $\beta_u$ are the upwind weights and $\alpha_d$ and $\beta_d$ are the downwind weights for the left and right fluxes respectively.  A subscript $j$ denotes the value at a given point $x = j \Delta x$ where $\Delta x$ is a uniform mesh spacing.

At a given time $t = n \Delta t$ at time-level $n$ and with a time-step $\Delta t$, we assume a wave-like solution with an amplification factor $A$, such that
\begin{align}
	\phi_j^{(n)} &= A^n e^{\iu j k \Delta x}
\end{align}
where $\phi_j^{(n)}$ denotes a value of $\phi$ at position $j$ and time-level $n$.  Using this to rewrite the left-hand side of equation~\eqref{eqn:advectionLR}
\begin{align}
\frac{\partial \phi_j}{\partial t} &= \frac{\partial}{\partial t} \left( A^{t / \Delta t} \right) e^{ijk\Delta x} = \frac{\ln A}{\Delta t} A^n e^{ikj\Delta x} \\
%
\shortintertext{hence equation~\eqref{eqn:advectionLR} becomes}
%
\frac{\ln A}{\Delta t} &= - \frac{u}{\Delta x} \left( \beta_u + \beta_d e^{ik\Delta x} - \alpha_u e^{-ik\Delta x} - \alpha_d \right) \\
\ln A &= -c \left( \beta_u - \alpha_d + \beta_d \cos k\Delta x + \iu \beta_d \sin k \Delta x - \alpha_u \cos k\Delta x + \iu \alpha_u \sin k\Delta x \right)
%
\intertext{where the Courant number $c = u \Delta t / \Delta x$.
Let $\Re = \beta_u - \alpha_d + \beta_d \cos k\Delta x - \alpha_u \cos k\Delta x$ and
$\Im = \beta_d \sin k \Delta x + \alpha_u \sin k\Delta x$, then}
%
\ln A &= -c \left( \Re + \iu \Im \right) \\
A &= e^{-c \Re} e^{-\iu c \Im} \\
%
\shortintertext{and the complex modulus and complex argument of $A$ are found to be}
%
|A| &= e^{-c \Re} = \exp \left( -c \left( \beta_u - \alpha_d + \left(\beta_d - \alpha_u \right) \cos k\Delta x \right) \right) \quad \text{and} \\
\arg(A) &= -c \Im = -c \left( \beta_d + \alpha_u \right) \sin k\Delta x
\end{align}
For stability, we need $|A| \leq 1$ and $\arg(A) < 0$ for $c > 0$, so
\begin{align}
\beta_u - \alpha_d + \left( \beta_d - \alpha_u \right) \cos k\Delta x &\geq 0 \quad \forall k\Delta x \quad \text{and} \\
\beta_d + \alpha_u &> 0 \label{eqn:arg-ineq}
\end{align}
Imposing the additional constraints that $\alpha_u = \beta_u$ and $\alpha_d = \beta_d$:
\begin{align}
|A| &= \exp \left( -c \left( \alpha_u - \alpha_d \right) \left(1 - \cos k\Delta x \right) \right)
%
\intertext{and given $1 - \cos k\Delta x \geq 0$ for well-resolved waves, then}
%
\alpha_u - \alpha_d &\geq 0 \\
%
\shortintertext{which provides a lower bound on $\alpha_u$:}
\alpha_u &\geq \alpha_d \label{eqn:lower-bound}
\end{align}
Additionally, we do not want more damping than an upwind scheme (where $\alpha_u = \beta_u = 1$, $\alpha_d = \beta_d = 0$), having an amplification factor, $A_\mathrm{up}$:
\begin{align}
|A_\mathrm{up}| &= \exp \left( -c \left(1 - \cos k\Delta x \right) \right)
%
\intertext{So we need $|A| \geq |A_\mathrm{up}|$:}
%
-c \left( \alpha_u - \alpha_d \right) \left(1 - \cos k\Delta x \right) &\geq -c \left( 1- \cos k\Delta x \right) \\
\alpha_u - \alpha_d &\leq 1 \\
\alpha_u &\leq 1 + \alpha_d
%
\intertext{which provides an upper bound on $\alpha_u$.  Combining with eqn~\eqref{eqn:lower-bound} we can bound $\alpha_u$ on both sides:}
%
\alpha_d \leq \alpha_u \leq 1 + \alpha_d
\end{align}
Now, assume that $\alpha_u + \alpha_d = 1$ (or $\alpha_d = 1 - \alpha_u$), then
\begin{align}
	1 - \alpha_u < \alpha_u &\leq 1 + (1 - \alpha_u) \\
	0.5 \leq \alpha_u &\leq 1
%
\shortintertext{and, since $\alpha_u + \alpha_d = 1$, then}
%
	0 \leq \alpha_d &\leq 0.5
\end{align}


\subsection*{Three-point analysis}


\bibliographystyle{ametsoc2014}
\bibliography{references}

\section*{Appendix: One-dimensional von Neumann stability analysis}
Two analyses are performed in order to find stability constraints on the weights $\mathbf{w} = \mathbf{\tilde{b}_1^+} \cdot \mathbf{m}$ as appear in equation~\eqref{eqn:weightedPinv}.  The first analysis uses two points to derived separate constraints on the upwind weight $w_u$ and downwind weight $w_d$.  The second analysis uses three points to derive a constraint that considers all weights in a stencil.

\subsection*{Two-point analysis}
We start with the conservation equation for a dependent variable $\phi$ that is discrete-in-space and continuous-in-time
\begin{align}
\frac{\partial \phi_j}{\partial t} &= - u \frac{\phi_R - \phi_L}{\Delta x} \label{eqn:advectionLR} \\
%
\shortintertext{where the left and right fluxes, $\phi_L$ and $\phi_R$ are weighted averages of the neighbouring points.  Assuming that $u$ is positive}
%
\phi_L &= \alpha_u \phi_{j-1} + \alpha_d \phi_j \\
\phi_R &= \beta_u \phi_j + \beta_d \phi_{j+1}
\end{align}
where $\alpha_u$ and $\beta_u$ are the upwind weights and $\alpha_d$ and $\beta_d$ are the downwind weights for the left and right fluxes respectively.  A subscript $j$ denotes the value at a given point $x = j \Delta x$ where $\Delta x$ is a uniform mesh spacing.

At a given time $t = n \Delta t$ at time-level $n$ and with a time-step $\Delta t$, we assume a wave-like solution with an amplification factor $A$, such that
\begin{align}
	\phi_j^{(n)} &= A^n e^{\iu j k \Delta x}
\end{align}
where $\phi_j^{(n)}$ denotes a value of $\phi$ at position $j$ and time-level $n$.  Using this to rewrite the left-hand side of equation~\eqref{eqn:advectionLR}
\begin{align}
\frac{\partial \phi_j}{\partial t} &= \frac{\partial}{\partial t} \left( A^{t / \Delta t} \right) e^{ijk\Delta x} = \frac{\ln A}{\Delta t} A^n e^{ikj\Delta x} \\
%
\shortintertext{hence equation~\eqref{eqn:advectionLR} becomes}
%
\frac{\ln A}{\Delta t} &= - \frac{u}{\Delta x} \left( \beta_u + \beta_d e^{ik\Delta x} - \alpha_u e^{-ik\Delta x} - \alpha_d \right) \\
\ln A &= -c \left( \beta_u - \alpha_d + \beta_d \cos k\Delta x + \iu \beta_d \sin k \Delta x - \alpha_u \cos k\Delta x + \iu \alpha_u \sin k\Delta x \right)
%
\intertext{where the Courant number $c = u \Delta t / \Delta x$.
Let $\Re = \beta_u - \alpha_d + \beta_d \cos k\Delta x - \alpha_u \cos k\Delta x$ and
$\Im = \beta_d \sin k \Delta x + \alpha_u \sin k\Delta x$, then}
%
\ln A &= -c \left( \Re + \iu \Im \right) \\
A &= e^{-c \Re} e^{-\iu c \Im} \\
%
\shortintertext{and the complex modulus and complex argument of $A$ are found to be}
%
|A| &= e^{-c \Re} = \exp \left( -c \left( \beta_u - \alpha_d + \left(\beta_d - \alpha_u \right) \cos k\Delta x \right) \right) \quad \text{and} \\
\arg(A) &= -c \Im = -c \left( \beta_d + \alpha_u \right) \sin k\Delta x
\end{align}
For stability, we need $|A| \leq 1$ and $\arg(A) < 0$ for $c > 0$, so
\begin{align}
\beta_u - \alpha_d + \left( \beta_d - \alpha_u \right) \cos k\Delta x &\geq 0 \quad \forall k\Delta x \quad \text{and} \\
\beta_d + \alpha_u &> 0 \label{eqn:arg-ineq}
\end{align}
Imposing the additional constraints that $\alpha_u = \beta_u$ and $\alpha_d = \beta_d$:
\begin{align}
|A| &= \exp \left( -c \left( \alpha_u - \alpha_d \right) \left(1 - \cos k\Delta x \right) \right)
%
\intertext{and given $1 - \cos k\Delta x \geq 0$ for well-resolved waves, then}
%
\alpha_u - \alpha_d &\geq 0 \\
%
\shortintertext{which provides a lower bound on $\alpha_u$:}
\alpha_u &\geq \alpha_d \label{eqn:lower-bound}
\end{align}
Additionally, we do not want more damping than an upwind scheme (where $\alpha_u = \beta_u = 1$, $\alpha_d = \beta_d = 0$), having an amplification factor, $A_\mathrm{up}$:
\begin{align}
|A_\mathrm{up}| &= \exp \left( -c \left(1 - \cos k\Delta x \right) \right)
%
\intertext{So we need $|A| \geq |A_\mathrm{up}|$:}
%
-c \left( \alpha_u - \alpha_d \right) \left(1 - \cos k\Delta x \right) &\geq -c \left( 1- \cos k\Delta x \right) \\
\alpha_u - \alpha_d &\leq 1 \\
\alpha_u &\leq 1 + \alpha_d
%
\intertext{which provides an upper bound on $\alpha_u$.  Combining with eqn~\eqref{eqn:lower-bound} we can bound $\alpha_u$ on both sides:}
%
\alpha_d \leq \alpha_u \leq 1 + \alpha_d
\end{align}
Now, assume that $\alpha_u + \alpha_d = 1$ (or $\alpha_d = 1 - \alpha_u$), then
\begin{align}
	1 - \alpha_u < \alpha_u &\leq 1 + (1 - \alpha_u) \\
	0.5 \leq \alpha_u &\leq 1
%
\shortintertext{and, since $\alpha_u + \alpha_d = 1$, then}
%
	0 \leq \alpha_d &\leq 0.5
\end{align}


\subsection*{Three-point analysis}


\end{document}
