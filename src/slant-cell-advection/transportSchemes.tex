\section{Transport schemes for arbitrary meshes}
The cubicFit transport scheme is described here for arbitrary two-dimensional meshes and arbitrary, single-layer spherical meshes.  Section \ref{sec:results} compares results using the cubicFit scheme with results using the linearUpwind transport scheme, and so a description of the linearUpwind scheme is also provided here.

\TODO{move the derivation from the advection equation and timestepping details to here since they're common to both cubicFit and linearUpwind}

\TODO{somewhere in here we need to state the multidimensional Courant number and note that the Courant number may vary between cells belonging to the same mesh.  We should then say that the stability limit for both schemes is $\max{\mathrm{Co}} \leq 1$.}

\input{cubicFit}

\subsection{linearUpwind transport scheme}
The linearUpwind scheme is documented here since it provides a baseline accuracy for the experiments in section~\ref{sec:results}.  The approximation for $\phi$ at a face $f$ is calculated using a gradient reconstruction:
\begin{align}
	\phi_F &= \phi_u + \nabla_c\: \phi \cdot \left(\mathbf{x}_f - \mathbf{x}_c \right)
\end{align} 
where $\phi_u$ is the upwind value of $\phi$, and $\mathbf{x}_f$ and $\mathbf{x}_c$ are the position vectors of the face centroid and cell centroid respectively.  \TODO{does the length of this vector change when using the spherical correction?  if so, how do I represent this mathematically?} The gradient $\nabla_c \:\phi$ is calculated using Gauss' theorem:
\begin{align}
	\nabla_c\: \phi = \frac{1}{\mathcal{V}_c} \sum_{f\in\:c} \tilde{\phi_F} \mathbf{S}_f
\end{align}
where $\tilde{\phi_F}$ is linearly interpolated from the two neighbouring cells of face $f$.

\TODO{cite OpenFOAM docs}
\TODO{Explain what linearUpwind does at the lower boundary.  I need to mention that we use a Dirichlet boundary condition of $\phi = \SI{0}{\kilogram\per\meter\cubed}$ at the ground.  Results are somewhat unstable on slanted cells if I use zero gradient boundary condition.  Or I could just use results with zeroGradient and mention the instability.  Using zeroGradient avoids the tracer becoming ``detached'' from the ground.}

