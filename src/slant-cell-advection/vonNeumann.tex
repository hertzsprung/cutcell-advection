\section*{Appendix A: One-dimensional von Neumann stability analysis}
Two analyses are performed in order to find stability constraints on the weights $\mathbf{w} = \mathbf{\tilde{b}_1^+} \cdot \mathbf{m}$ as appear in equation~\eqref{eqn:weightedPinv}.  The first analysis uses a two-cell approximation to derive separate constraints on the upwind weight $w_u$ and downwind weight $w_d$.  The second analysis uses three cells to derive a constraint that considers all weights in a stencil.

\subsection*{Two-cell analysis}
We start with the conservation equation for a dependent variable $\phi$ that is discrete-in-space and continuous-in-time
\begin{align}
\frac{\partial \phi_j}{\partial t} &= - v \frac{\phi_R - \phi_L}{\Delta x} \label{eqn:advectionLR} \\
%
\intertext{where $v$ is the velocity, and the left and right fluxes, $\phi_L$ and $\phi_R$, are weighted averages of the neighbouring cell centres.  Assuming that $v$ is positive}
%
\phi_L &= \alpha_u \phi_{j-1} + \alpha_d \phi_j \\
\phi_R &= \beta_u \phi_j + \beta_d \phi_{j+1}
\end{align}
where $\phi_{j-1}, \phi_j, \phi_{j+1}$ are cell centre values, and $j$ denotes a cell centre position $x = j \Delta x$ where $\Delta x$ is a uniform mesh spacing.
$\alpha_u$ and $\beta_u$ are the upwind weights and $\alpha_d$ and $\beta_d$ are the downwind weights for the left and right fluxes respectively, and $\alpha_u + \alpha_d = 1$ and $\beta_u + \beta_d = 1$.

At a given time $t = n \Delta t$ at time-level $n$ and with a time-step $\Delta t$, we assume a wave-like solution with an amplification factor $A$, such that
\begin{align}
	\phi_j^{(n)} &= A^n e^{\iu j k \Delta x} \label{eqn:vn}
\end{align}
where $\phi_j^{(n)}$ denotes a value of $\phi$ at position $j$ and time-level $n$.  Using this to rewrite the left-hand side of equation~\eqref{eqn:advectionLR}
\begin{align}
\frac{\partial \phi_j}{\partial t} &= \frac{\partial}{\partial t} \left( A^{t / \Delta t} \right) e^{ijk\Delta x} = \frac{\ln A}{\Delta t} A^n e^{ikj\Delta x} \\
%
\shortintertext{hence equation~\eqref{eqn:advectionLR} becomes}
%
\frac{\ln A}{\Delta t} &= - \frac{v}{\Delta x} \left( \beta_u + \beta_d e^{ik\Delta x} - \alpha_u e^{-ik\Delta x} - \alpha_d \right) \\
\ln A &= -c \left( \beta_u - \alpha_d + \beta_d \cos k\Delta x + \iu \beta_d \sin k \Delta x - \alpha_u \cos k\Delta x + \iu \alpha_u \sin k\Delta x \right)
%
\intertext{where the Courant number $c = v \Delta t / \Delta x$.
Let $\Re = \beta_u - \alpha_d + \beta_d \cos k\Delta x - \alpha_u \cos k\Delta x$ and
$\Im = \beta_d \sin k \Delta x + \alpha_u \sin k\Delta x$, then}
%
\ln A &= -c \left( \Re + \iu \Im \right) \\
A &= e^{-c \Re} e^{-\iu c \Im} \\
%
\shortintertext{and the complex modulus of $A$ is}
%
|A| &= e^{-c \Re} = \exp \left( -c \left( \beta_u - \alpha_d + \left(\beta_d - \alpha_u \right) \cos k\Delta x \right) \right) \text{ .}
\end{align}
For stability we need $|A| \leq 1$ and, imposing the additional constraints that $\alpha_u = \beta_u$ and $\alpha_d = \beta_d$, then
\begin{align}
\left( \alpha_u - \alpha_d \right) \left( 1 - \cos k\Delta x \right) &\geq 0 \quad \forall k\Delta x
%
\shortintertext{and, given $0 \leq 1 - \cos k \Delta x \leq 2$, then}
%
\alpha_u - \alpha_d \geq 0 \text{ .} \label{eqn:twopoint-lower}
\end{align}
Additionally, we do not want more damping than a first-order upwind scheme (where $\alpha_u = \beta_u = 1$, $\alpha_d = \beta_d = 0$), having an amplification factor, $A_\mathrm{up}$, so we need $\left\lvert A \right\rvert \geq \left\lvert A_\mathrm{up} \right\rvert$, hence
\begin{align}
	\exp \left( -c \left(\alpha_u - \alpha_d\right) \left( 1 - \cos k\Delta x \right) \right) &\geq \exp \left( -c \left(1 - \cos k\Delta x \right) \right) \quad \forall k\Delta x
%
\shortintertext{therefore}
%
	\alpha_u - \alpha_d &\leq 1 \text{ .} \label{eqn:twopoint-upper}
\end{align}
Now, knowing that $\alpha_u + \alpha_d = 1$ (or $\alpha_d = 1 - \alpha_u$) then, using equations~\eqref{eqn:twopoint-lower} and \eqref{eqn:twopoint-upper},
\begin{align}
	0.5 \leq \alpha_u &\leq 1 \text{ and} \label{eqn:vn:upwind} \\
	0 \leq \alpha_d &\leq 0.5 \label{eqn:vn:downwind} \text{ .}
\end{align}

\subsection*{Three-cell analysis}
We start again from equation~\eqref{eqn:advectionLR} but this time approximate $\phi_L$ and $\phi_R$ using three cell centre values,
\begin{align}
	\phi_L &= \alpha_{uu} \phi_{j-2} + \alpha_u \phi_{j-1} + \alpha_d \phi_j \\
	\phi_R &= \alpha_{uu} \phi_{j-1} + \alpha_u \phi_j + \alpha_d \phi_{j+1}
\end{align}
having used the same weights $\alpha_{uu}$, $\alpha_u$ and $\alpha_d$ for both left and right fluxes.
Substituting equation~\eqref{eqn:vn} into equation~\eqref{eqn:advectionLR} we find
\begin{align}
A = \exp\left( -c \left[ \alpha_{uu} \left( e^{-ik\Delta x} - e^{-2ik\Delta x} \right) + \alpha_u \left( 1 - e^{-ik\Delta x} \right) + \alpha_d \left( e^{ik\Delta x} - 1 \right) \right] \right)
%
\intertext{so that, if the complex modulus $\left\lvert A \right\rvert \leq 1$ then}
%
\alpha_u - \alpha_d + \left( \alpha_{uu} - \alpha_u + \alpha_d \right) \cos k\Delta x - \alpha_{uu} \cos 2k\Delta x \geq 0 \text{ .}
\end{align}
If $k\Delta x = \pi$ then $\cos k\Delta x = -1$ and $\cos 2k\Delta x = 1$ and $\alpha_u - \alpha_d \geq \alpha_{uu}$.  If $k\Delta x = \pi / 2$ then $\cos k\Delta x = 0$ and $\cos 2k\Delta x = -1$ and $\alpha_u - \alpha_d \geq -\alpha_{uu}$.  Hence we find that
\begin{align}
	\alpha_u - \alpha_d &\geq \left\lvert\alpha_{uu}\right\rvert \label{eqn:uuConstraint} \text{ .}
%
	\intertext{When the same analysis is performed with four cells, $\alpha_{uuu}$, $\alpha_{uu}$, $\alpha_u$ and $\alpha_d$, by varying $k \Delta x$ we find that equation~\eqref{eqn:uuConstraint} holds replacing $\left\lvert \alpha_{uu} \right\rvert$ with $\max(\left\lvert\alpha_{uu}\right\rvert, \left\lvert\alpha_{uuu}\right\rvert)$.  Hence, we generalise equation~\eqref{eqn:uuConstraint} to find the final stability constraint}
%
	\alpha_u - \alpha_d &\geq \max_{p\:\in\:P} |\alpha_p|
\end{align}
where the peripheral cells $P$ is the set of all stencil cells except for the upwind cell and downwind cell, and $\alpha_p$ is the weight for a given peripheral cell $p$.
We hypothesise that the three stability constraints (equations~\ref{eqn:vn:upwind}, \ref{eqn:vn:downwind} and \ref{eqn:uuConstraint}) are necessary but not sufficient for a transport scheme on arbitrary meshes.

\revtwo{The stability of the one-dimensional transport equation discretised in space and time could be analysed using existing techniques \citep{baldauf2008}, but we have only analysed the spatial stability of the cubicFit scheme.  Numerical experiments presented in section~\ref{sec:mountainAdvection} demonstrate that the cubicFit scheme is generally insensitive to the time-step, provided that it is below the stability limit.}
