\section{Introduction}

Numerical simulations of atmospheric flows solve equations of motion that result in the transport of momentum, temperature, water species and trace gases.  The numerical representation of Earth's terrain complicates the transport problem because the mesh is necessarily distorted by the modification of the lower boundary.
As new atmospheric models use increasingly fine mesh spacing, meshes are able to resolve steep, small-scale slopes.  Numerical schemes in operational weather forecast models can perform poorly over large mountain ranges, exhibiting small-scale numerical noise in momentum \citep{walko-avissar2008b}, temperature, humidity \citep{schaer2002} and potential vorticity fields \citep{hoinka-zaengl2004}, or even violating the Courant--Friedrich--Lewy stability constraint resulting in so-called `grid-point storms' \citep{webster2003}.
A transport scheme is desired that yields stable and accurate solutions, particularly near the surface where the weather affects us directly.
We present a new transport scheme which is numerically stable on arbitrary meshes and which is generally insensitive to mesh distortions created by steep slopes.  It has a low computational cost since most calculations are not repeated every time-step because they depend upon the mesh geometry only.

There are two main methods for representing terrain in atmospheric models: terrain-following layers and cut cells.  Both methods modify regular quadrilateral meshes to produce distorted meshes with cells that are aligned in vertical columns.  Most operational models use terrain-following layers in which horizontal mesh surfaces are moved upwards to accommodate the terrain.  A vertical decay function is chosen so that mesh surfaces slope less steeply with increasing height.
The most straightforward is the linear decay function used by the basic terrain-following transform \citep{galchen-somerville1975} (also called the sigma coordinate), but many atmospheric models suffer from large numerical errors on such meshes \citep{schaer2002,klemp2011,eckermann2014}.
To reduce such errors, more complex decay functions have been developed so that mesh surfaces are smoother \citep{simmons-burridge1981,schaer2002,leuenberger2010,klemp2011}.

An alternative to terrain-following layers is the cut cell method.  Cut cell meshes are constructed by `cutting' a regular quadrilateral mesh with a piecewise-linear representation of the terrain.  New vertices are created where the terrain intersects mesh edges, and cell volumes that lie beneath the ground are removed.  Cut cell meshes can have arbitrarily small cells that impose severe time-step constraints on explicit transport schemes.  Several techniques have been developed to alleviate this problem, known as the `small-cell problem': small cells can be merged with adjacent cells \citep{yamazaki2016}, cell volumes can be artificially increased \citep{steppeler2002}, or an implicit scheme can be used near the ground with an explicit scheme used aloft \citep{jebens2011}.

Another method for avoiding the small-cell problem was proposed by Shaw and Weller \citep{shaw-weller2016} in which cell vertices are moved vertically so that they are positioned at the terrain surface.  We refer to this alternative method as the slanted cell method in order to distinguish it from the traditional cut cell method.  Slanted cell meshes do not suffer from arbitrarily small cells because the horizontal cell dimensions are not modified by the presence of terrain.

Smoothed terrain-following layers, cut cells and slanted cell methods all reduce the amount of mesh distortion but any mesh that represents sloping terrain must necessarily be distorted, at least near the ground.
Even when distortions are minimal, transport across mesh surfaces tends to be more common near steep slopes, and this misalignment between the flow and mesh surfaces increases numerical errors \citep{leonard1993,schaer2002,shaw-weller2016}.
A huge variety of transport schemes have been developed for atmospheric models, but few are able to account for distortions associated with steep terrain because they treat horizontal and vertical transport separately \citep{kent2014}, resulting in numerical errors called `splitting errors'.
Such errors can be reduced by explicitly accounting for transverse fluxes when combining fluxes \citep{leonard1996}, but splitting errors are still apparent in flows over steep terrain where meshes are highly distorted and metric terms in a terrain-following coordinate transform are large \citep{weller2017}.

Transport schemes are often classified as dimensionally-split or multidimensional.
Dimensionally-split schemes such as \citep{lin-rood1996,katta2015} calculate transport in each dimension separately before the flux contributions are combined.  Such schemes are computationally efficient and allow existing one-dimensional high-order methods to be used.  To use a dimensionally-split scheme over terrain, a terrain-following coordinate transform is required.
Perhaps confusingly, dimensionally-split schemes are sometimes called multidimensional, too, because they use one-dimensional techniques for multidimensional transport.

Unlike dimensionally-split schemes, multidimensional schemes consider transport in two or three dimensions together.
There are several subclasses of multidimensional schemes that include
2D semi-Lagrangian finite volume schemes (also called conservative mesh remapping),
swept-area schemes (also called flux-form semi-Lagrangian, incremental remapping, or forward-in-time),
and method-of-lines schemes (also called Eulerian schemes).
2D semi-Lagrangian finite volume schemes such as \citep{iske-kaeser2004,lauritzen2010} integrate over departure cells that are found by tracing backward the trajectories of cell vertices.  These schemes are conservative because departure cells are constructed so that there are no overlaps or gaps, which requires that cell areas are simply-connected domains \citep{lauritzen2011book}.
Swept area schemes such as \citep{lashley2002,skamarock-menchaca2010,lauritzen2011,thuburn2014} calculate the flux through a cell face by integrating over the upstream area that is swept out over one time-step.  Such schemes differ in their choice of area approximation, sub-grid reconstruction, and spatial integration method.
Because swept area schemes integrate over the reconstructed field, they typically require a matrix-vector multiply per face \citep{thuburn2014,skamarock-menchaca2010}.
Method-of-lines schemes such as \citep{weller2009,skamarock-gassmann2011} use a spatial discretisation to reduce the transport PDE to an ODE that is typically solved using a multi-stage time-stepping method.  
Unlike 2D semi-Lagrangian finite volume schemes, swept area and method-of-lines schemes achieve conservation for non-simply connected domains that can result from small-scale rotational flows \citep{lauritzen2011}.
There are many more types of atmospheric transport schemes, but all can be classified according to their treatment of the three spatial dimensions.  A more comprehensive overview is presented by Lauritzen et al. \cite{lauritzen2014}.

For transport schemes that are ordinarily classified as `multidimensional', a further distinction ought to made between horizontally-multidimensional and three-dimensional schemes.
Multidimensional schemes are almost always horizontally-multidimensional because, while the two horizontal dimensions are considered together, horizontal and vertical transport are still treated separately.
Very few three-dimensional schemes have been suggested for use in atmospheric models \citep[e.g.][]{miura2007,yeh2007,gassmann2013} although such schemes might be expected to be more accurate on distorted meshes with steep terrain.
The multidimensional scheme developed in \citep{weller-shahrokhi2014} is unusual because it has no horizontal--vertical splitting and it has been used in two-dimensional flows on Cartesian $x$--$z$ planes with distorted meshes \citep{shaw-weller2016,weller2017}.

In this paper, we present a new multidimensional method-of-lines scheme, `cubicFit', that improves upon the scheme in \citep{weller-shahrokhi2014} and avoids all splitting errors.  To reconstruct values at cell faces, the scheme fits a multidimensional polynomial over a cubic, upwind-biased stencil using a least-squares approach.  The implementation uses constraints derived from a von Neumann stability analysis to select appropriate polynomial fits for stencils in highly-distorted mesh regions.  Almost all of the least-squares procedure depends upon the mesh geometry only and reconstruction weights can be pre-computed without knowledge of the velocity field or tracer field.  Hence, the cubicFit scheme has a low computational cost that is comparable to dimensionally-split schemes, requiring only $n$ multiplies per cell face per time-step where $n$ is the size of the stencil.

The remainder of this paper is organised as follows.
Section~\ref{sec:transport} starts by discretising the transport equation using a method-of-lines approach before describing the cubicFit transport scheme and a standard multidimensional linear upwind transport scheme.
Section~\ref{sec:results} evaluates the cubicFit scheme using three idealised numerical tests.
The first standard test follows Sch\"ar et al. \citep{schaer2002}, transporting a tracer horizontally above steep mountains on two-dimensional, highly-distorted terrain-following meshes.
The second is a new test case designed to assess numerical accuracy next to a mountainous lower boundary.  In this test, a tracer placed at the ground is transported over steep slopes using terrain-following, cut cell and slanted cell meshes.
The third is a standard test of deformational flow on a spherical Earth, specified by Lauritzen et al. \citep{lauritzen2012}, which we use to assess the cubicFit transport scheme on hexagonal icosahedra and cubed-sphere meshes.
Concluding remarks are made in section~\ref{sec:conclusion}.

