\documentclass{article}
\usepackage{amsmath}
\usepackage{bm}
\usepackage[hidelinks]{hyperref}
\usepackage{natbib}
\usepackage{xcolor}
\usepackage{doi}

\newcommand{\TODO}[1]{\textcolor{purple}{TODO: \emph{#1}}}

\title{Techniques to alleviate the small cell problem}
\author{James Shaw}

\begin{document}
\maketitle

Explicit, finite volume advection schemes must satisfy the Courant--Friedrichs--Lewy (CFL) constraint in order to be stable.  The CFL constraint insists that the maximum timestep is proportional to the volume of the smallest cell.  The cut cell method can create arbitrarily small cells in the vicinity of the terrain that impose severe constraints on the timestep.

There are a variety of techniques that alleviate the timestep constraint for cut cell meshes.  One technique is to combine small cells with neighbouring cells, so that the smallest cell volume and, hence, the maximum timestep, is increased.  \citet{ye1999} identified cells with cell centres below ground, and combined each of them with the cell above.  They applied the technique to a two-dimensional model of viscous incompressible flow.  The technique ensures that the smallest cell volume is at least half the volume of an uncut cell, hence the maximum timestep is also constrained to be at least half the maximum timestep for the equivalent mesh with no cut cells.
\citet{yamazaki2016} extended this cell-combining technique to a three-dimensional, fully-compressible, non-hydrostatic atmospheric model, in which small cells are combined vertically or horizontally depending on the steepness of the terrain slope in the small cell.  \TODO{and? so what?}

\TODO{cell merging has disadvantages: velocity points may not have pressure points on both sides \citep{kirkpatrick2003}, so it is no longer possible to separate horizontal and vertical pressure gradient calculations \citep{walko-avissar2008b}.  \citet{kirkpatrick2003} also says that it is difficult to extend cell merging techniques to three dimensions, but \citet{yamazaki2016} shows this can be done reasonably straightforwardly.}

\bibliographystyle{ametsoc2014}
\bibliography{references}

\end{document}
