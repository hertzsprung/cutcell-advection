\documentclass[times]{elsarticle}
\usepackage{fullpage}
\usepackage{graphicx}
\usepackage{amsmath}
\usepackage{mathtools}
\usepackage{xcolor}
\usepackage{bm}
\usepackage{natbib}
\usepackage[hidelinks]{hyperref}
\usepackage{doi}
\usepackage[final,babel]{microtype}
\usepackage[utf8]{inputenc}
\usepackage[british]{babel}
\usepackage{csquotes}
\usepackage[T1]{fontenc}
\usepackage{siunitx}
\usepackage[font={small}]{caption}

\newcommand{\iu}{{i\mkern1mu}}
\newcommand{\iunit}{\boldsymbol{\hat \imath}}
\newcommand{\junit}{\boldsymbol{\hat \jmath}}
\newcommand{\kunit}{\boldsymbol{\hat k}}
\newcommand{\TODO}[1]{\textcolor{purple}{TODO: \emph{#1}}}

\begin{document}

\begin{frontmatter}
\title{A finite volume transport scheme for atmospheric flows over steep terrain \\ \TODO{(working title)}}
\author[uor]{James Shaw\corref{cor}}
\ead{js102@zepler.net}
\author[uor]{Hilary Weller}

\cortext[cor]{Corresponding author}
\address[uor]{Department of Meteorology, University of Reading, Reading, United Kingdom}

\begin{abstract}
	\TODO{abstract}
\end{abstract}

\begin{keyword}
	\TODO{keywords}
\end{keyword}
\end{frontmatter}

\section{Introduction}

First, we present a multidimensional advection scheme that is computationally cheap and suitable for complex flows on a variety of meshes.  Second, we present a new type of Cartesian mesh, the slanted cell mesh, that avoids the small cell problem associated with cut cell meshes.   We apply the advection scheme to tests over steep orography and show that accurate results are obtained on slanted cell meshes.  Finally, we challenge the multidimensional advection scheme using a test of deformational flow on a geodesic mesh.


\section{Multidimensional advection scheme}

The advection of a dependent variable $\phi$ is given by the conservation equation
\begin{align}
	\frac{\partial \phi}{\partial t} + \nabla \cdot \left( \mathbf{u} \phi \right) = 0 \label{eqn:advection}
\end{align}
where $\mathbf{u}$ is a prescribed wind field.  The time derivative is discretised using a three-stage, second-order Runge-Kutta scheme:
\begin{subequations}
\begin{align}
	\phi^\star &= \phi^{(n)} + \Delta t f(\phi^{(n)}) \\
	\phi^{\star\star} &= \phi^{(n)} + \frac{\Delta t}{2} \left( f(\phi^{(n)}) + f(\phi^\star) \right) \\
	\phi^{(n+1)} &= \phi^{(n)} + \frac{\Delta t}{2} \left( f(\phi^{(n)}) + f(\phi^{\star\star}) \right)
\end{align}
\end{subequations}
where \(f(\phi^{(n)}) = - \nabla \cdot (\mathbf{u} \phi^{(n)})\) at time level \(n\).

Using the finite volume method, the wind field is prescribed at face centroids and the dependent variable is stored at cell centroids.  The divergence term in equation~\eqref{eqn:advection} is discretised using Gauss's theorem:
\begin{align}
	\nabla \cdot \left( \mathbf{u} \phi \right) \approx \frac{1}{\mathcal{V}_c} \sum_{f \in c} \mathbf{u}_f \cdot \mathbf{S}_f \phi_F
\end{align}
where $\mathcal{V}_c$ is the cell volume, $\mathbf{u}_f$ is a wind vector prescribed at a face, ${\mathbf{S}_f}$ is the surface area vector with a direction outward normal to the face and a magnitude equal to the face area, and $\sum_{f \in c}$ denotes a summation over all faces $f$ belonging to cell $c$.  The value of the dependent variable at the face, $\phi_F$, is approximated by a least squares fit over a stencil of surrounding cell centre values.

\begin{figure}
	\centering
	\includegraphics{../fig-upwind-stencil/fig-upwind-stencil.pdf}
	\caption{Upwind-biased stencils for faces far away from the boundaries of two-dimensional (a) triangular, (b) rectangular and (c) hexagon meshes.  The stencil is used to fit a multidimensional polynomial to cell centre values, $\phi_c$, marked by grey circles, in order to approximate the value $\phi_F$ at the face centroid marked by an open circle.  $\phi_u$ and $\phi_d$ are the values at the centroids of the upwind and downwind cells neighbouring the target face, drawn with a heavy line.  The wind vector $\mathbf{u}_f$ is prescribed at face $f$ and determines the choice of stencil at each timestep.}
	\label{fig:interiorStencils}
\end{figure}

To introduce the approximation method, we will consider how an approximate value is calculated for a face that is far away from the boundaries of a two-dimensional uniform rectangular mesh.  For any mesh, every interior face connects two adjacent cells.  The wind direction at the face determines which of the two adjacent cells is the upwind cell.  Since the stencil is upwind-biased, two stencils must be constructed for every interior face, and the appropriate stencil is chosen for each face depending on the wind direction at that face for every timestep.

The upwind-biased stencil for a face $f$ is shown in figure~\ref{fig:interiorStencils}b.  The wind at the face, $\mathbf{u}_f$, is blowing from the upwind cell $c_u$ to the downwind cell $c_d$.
To obtain an approximate value at $f$, a polynomial least squares fit is calculated using the stencil values.
The stencil has \num{4} points in $x$ and \num{3} points in $y$, leading to a natural choice of polynomial that is cubic in $x$ and quadratic in $y$:
\begin{align}
	\phi = a_1 + a_2 x + a_3 y + a_4 x^2 + a_5 xy + a_6 y^2 + a_7 x^3 + a_8 x^2 y + a_9 x y^2 \label{eqn:fullPoly}
\end{align}
A least squares approach is needed because the system of equations is overconstrained, with \num{12} stencil values but only \num{9} polynomial terms.  If the stencil geometry is expressed in a local coordinate system with the face centroid as the origin, then the approximated value $\phi_F$ is equal to the constant term $a_1$.

The remainder of this section generalises the the approximation technique for arbitrary meshes, explaining the methods for constructing stencils, performing a least squares fit with a suitable polynomial, and ensuring numerical stability of the advection scheme.

\subsection{Stencil construction}
For every interior face, two stencils are constructed, one for each of the possible upwind cells.  For a given face $f$ and upwind cell $c_u$, we find those faces that are connected to $c_u$ and `oppose' face $f$.  These are called the \textit{opposing faces}.
The opposing faces for face $f$ and upwind cell $c_u$ are determined as follows.
Defining $G$ to be the set of other faces bounding cell $c_u$, we calculate the `opposedness', $\mathrm{Opp}$, between faces $f$ and $g \in G$, defined as
\begin{align}
	\mathrm{Opp}(f, g) \equiv - \frac{\mathbf{S}_f \cdot \mathbf{S}_g}{|\mathbf{S}_f|^2} \label{eqn:opp}
\end{align}
where $\mathbf{S}_f$ and $\mathbf{S}_g$ are the surface area vectors pointing outward from cell $c_u$ for faces $f$ and $g$ respectively.
Using the fact that $\mathbf{a} \cdot \mathbf{b} = |\mathbf{a}|\:|\mathbf{b}| \cos(\theta)$ we can rewrite equation~\eqref{eqn:opp} as
\begin{align}
	\mathrm{Opp}(f, g) = - \frac{|\mathbf{S}_g|}{|\mathbf{S}_f|} \cos(\theta)
\end{align}
where $\theta$ is the angle between faces $f$ and $g$.  In this form, it can be seen that $\mathrm{Opp}$ is a measure of the area of $g$ and how closely it parallels face $f$.

The set of opposing faces, $\mathrm{OF}$, is a subset of $G$, comprising those faces with $\mathrm{Opp} \geq 0.5$, and the face with the maximum opposedness.  Expressed in set notation, this is
\begin{align}
	\mathrm{OF}(f,c_u) \equiv \{ g : \mathrm{Opp}(f, g) \geq 0.5 \} \cup \{ g : \max(\mathrm{Opp}(f, g)) \} 
\end{align}
On a two-dimensional rectangular mesh, there is always one opposing face that is exactly parallel to the face $f$.

Once the opposing faces have been determined, the set of internal and external cells must be found.  The \textit{internal cells} are those cells that are connected to the opposing faces.  Note that $c_u$ is always an internal cell.  The \textit{external cells} are those cells that share vertices with the internal cells.  Note that $c_d$ is always an external cell.  Having found these two sets of cells, the stencil is constructed to comprise all internal and external cells.

\begin{figure}
	\centering
	\includegraphics{../fig-double-upwind-stencil/fig-double-upwind-stencil.pdf}
	%
	\caption{A thirteen-cell, upwind-biased stencil for face $f$ connecting the pentagonal upwind cell, $c_u$, and the downwind cell $c_d$.  The dashed lines denote the two faces of cell $c_u$ that oppose $f$, and black circles mark the centroids of the internal cells that are connected to these two opposing faces.  The stencil is extended outwards by including cells that share vertices with the three internal cells, where black squares mark these vertices.  The local coordinate system $(x, y)$ has its origin at the centroid of face $f$, marked by an open circle, with $x$ normal to $f$ and $y$ perpendicular.}
	\label{fig:double-upwind-stencil}
\end{figure}

Figure~\ref{fig:double-upwind-stencil} illustrates a stencil construction for face $f$ connecting upwind cell $c_u$ and downwind cell $c_d$.  The two opposing faces are denoted by thick dashed lines and the centres of the three adjoining internal cells are marked by black circles.  The stencil is extended outwards by including the external cells that share vertices with the internal cells, marked by black squares.  The resultant stencil contains 13 cells.


\subsection{Least squares fit}
To approximate the value at a face $f$, a least squares fit is calculated from a stencil of surrounding cell centre values.  First, we will show how a polynomial least squares fit is calculated for a face on a rectangular mesh.  Second, we will make modifications to the least squares fit that are necessary for numerical stability.  Finally, we will describe how the approach is applicable to faces of arbitrary meshes.  

For faces that are far away from the boundaries of a rectangular mesh, we fit the multidimensional polynomial given by equation~\eqref{eqn:fullPoly} that has nine unknown coefficients, $\mathbf{a} = a_1 \ldots a_9$, using the twelve cell centre values from the upwind-biased stencil, $\bm{\phi} = \phi_1 \ldots \phi_{12}$.  This yields a matrix equation
\begin{align}
	\begin{bmatrix}
		1 & x_1 & y_1 & x_1^2 & x_1 y_1 & y_1^2 & x_1^3 & x_1^2 y_1 & x_1 y_1^2 \\
		1 & x_2 & y_2 & x_2^2 & x_2 y_2 & y_2^2 & x_2^3 & x_2^2 y_2 & x_2 y_2^2 \\
		\vdots & \vdots & \vdots & \vdots & \vdots & \vdots & \vdots & \vdots & \vdots \\
		1 & x_{12} & y_{12} & x_{12}^2 & x_{12} y_{12} & y_{12}^2 & x_{12}^3 & x_{12}^2 y_{12} & x_{12} y_{12}^2 \\
	\end{bmatrix}
	\begin{bmatrix}
		a_1 \\
		a_2 \\
		\vdots \\
		a_9
	\end{bmatrix}
	=
	\begin{bmatrix}
		\phi_1 \\
		\phi_2 \\
		\vdots \\
		\phi_{12}
	\end{bmatrix}
\end{align}
which can be written as
\begin{align}
	\mathbf{B} \mathbf{a} = \bm{\phi} \label{eqn:unweightedLeastSquares}
\end{align}
The rectangular matrix $\mathbf{B}$ has one row for each cell in the stencil and one column for each term in the polynomial.  $\mathbf{B}$ is constructed using only the mesh geometry and is called the \textit{stencil matrix}.
A local coordinate system is established in which $x$ is normal to the face $f$ and $y$ is perpendicular to $x$.
The coordinates $(x_i, y_i)$ give the position of the centroid of the $i$th cell in the stencil.
The unknown coefficients $\mathbf{a}$ are calculated using the pseudo-inverse of $\mathbf{B}^+$ found by singular value decomposition:
\begin{align}
	\mathbf{a} = \mathbf{B}^+ \bm{\phi}
%
\intertext{Recall that the approximate value $\phi_F$ is equal to the constant coefficient $a_1$, which is calculated as} 
%
	a_1 = \begin{bmatrix}
		b_{1,1}^+ \\
		b_{1,2}^+ \\
		\vdots \\
		b_{1,12}^+
	\end{bmatrix}
	\cdot
	\begin{bmatrix}
		\phi_1 \\
		\phi_2 \\
		\vdots \\
		\phi_{12}
	\end{bmatrix}
\end{align}
where $b_{1,1}^+ \ldots b_{1,12}^+$ are the elements of the first row of $\mathbf{B}^+$.

The least squares fit presented above was unweighted, with equal weight placed on all stencil cell values.  \citet{lashley2002} showed that a weighted least squares fit is necessary for numerical stability, with greater weight being placed on the cells connected to face $f$.  We assign a weight to each stencil value, $\mathbf{w} = w_1 \ldots w_{12}$ and multiply by equation~\eqref{eqn:unweightedLeastSquares} to obtain
\begin{align}
	\mathbf{\tilde{B}} \mathbf{a} = \mathbf{w} \cdot \bm{\phi}
\end{align}
where $\mathbf{\tilde{B}} = \mathbf{W} \mathbf{B}$ and $\mathbf{W} = \mathrm{diag}(\mathbf{w})$.  The constant coefficient $a_1$ is calculated from the pseudo-inverse, $\mathbf{\tilde{B}}^+$:
\begin{align}
	a_1 = \mathbf{\tilde{b}_1^+} \cdot \mathbf{w} \cdot \mathbf{\phi}
\end{align}
where $\mathbf{\tilde{b}_1^+} = \tilde{b}_{1,1}^+ \ldots \tilde{b}_{1,12}^+$ are the elements of the first row of $\mathbf{\tilde{B}}^+$.

For faces of a non-rectangular mesh, or faces that are near a boundary, the number of stencil cells and number of polynomial terms may differ: a stencil will have two or more cells and, for two-dimensional meshes, its polynomial will have between one and nine terms.  Additionally, the polynomial cannot have more terms than its stencil has cells because this would lead to an underconstrained system of equations.  The procedure for choosing suitable polynomials is discussed next.

\subsection{Polynomial generation}
\begin{figure}
	\centering
	\caption{\TODO{2x3 and 3x3 meshes}}
	\label{fig:boundaryStencils}
\end{figure}

The majority of faces on a uniform two-dimensional mesh have stencils with more than nine cells.  For example, a triangular mesh has 19 points (figure~\ref{fig:interiorStencils}a), a rectangular mesh has 12 points (figure~\ref{fig:interiorStencils}b), and a hexagonal mesh has 10 points (figure~\ref{fig:interiorStencils}c).
In all three cases, constructing a system of equations using the nine-term polynomial in equation~\eqref{eqn:fullPoly} leads to an overconstrained problem that can be solved using least squares.  However, this is not true for faces near boundaries: stencils that have fewer than nine cells (figure~\ref{fig:boundaryStencils}a) would result in an underconstrained problem, and stencils that have exactly nine cells may lack sufficient information to constrain high-order terms.  For example, the stencil in figure~\ref{fig:boundaryStencils}b lacks sufficient information to fit the $x^3$ term.  In such cases, it becomes necessary to perform a least squares fit using a polynomial with fewer terms.

For every stencil, we find a set of \textit{candidate polynomials} that do not result in an underconstrained problem.  A candidate polynomial has between one and nine terms and includes a combination of the terms in equation~\eqref{eqn:fullPoly}.  There are two additional constraints that a candidate polynomial satisfy.

First, high-order terms may be included in a candidate polynomial only if the lower-order terms are also included.
Let
\begin{align}
	M(x, y) = { x^i y^j : i,j \geq 0 \text{ and } i+j \leq 3 }
\end{align}
be the set of all monomials of degree at most \num{3} in $x, y$.
A subset $S$ of $M(x,y)$ is ``dense'' if, whenever $x^a y^b$ and $x^c y^d$ are in $S$ with $a \leq c$ and $b \leq d$, then $x^i y^j$ is also in $S$ for all $a < i < c$, $b < j < d$.

Second, a candidate polynomial must have a stencil matrix $\mathbf{B}$ that is full rank.  The matrix is considered full rank if its smallest singular value is greater than \num{1e-9}.  Using a polynomial with all nine terms and the stencil in figure~\ref{fig:boundaryStencils}b results in a rank-deficient matrix and so the nine-term polynomial would not be a candidate polynomial.

The candidate polynomials are all the dense subsets of $M(x,y)$ that have a stencil matrix that is full rank.  The final stage of the advection scheme selects a candidate polynomial and ensures that the least squares fit is numerically stable.

\subsection{Stabilisation procedure}
% stability constraints
% reweighting

Stability constriants:
\begin{align}
	0.5 \leq u \leq 1 \\
	0 \leq d \leq 0.5 \\
	u - d \geq \max(|p|)
\end{align}




\section{Results}
\TODO{should I use RK4 or will RK2 suffice?}
\TODO{somewhere mention that the second-order convergence is a limitation of the divergence discretisation.  With more DoF a higher order should be achievable.}

\subsection{Horizontal transport with mesh refinement}
Coping with sudden changes in mesh spacing is necessary for some types of mesh refinement and mesh adaptivity.  We should use the horizontal advection test from \citet{schaer2002} on a BTF mesh with a 2x/4x/8x refined mesh in part of the mountainous region.  The tracer will have to pass into and out of this refined region.  This test will also help to familiarise the reader with this standard test.  We will modify parts of this test in the following subsection in order to test advection over the lower boundary.

\begin{figure}
	\caption{\TODO{horizontal advection error contours for linearUpwind and cubicFit}}
\end{figure}

\subsection{Transport over a mountainous lower boundary}
A two-dimensional transport test over mountains was developed in \citep{schaer2002} to study the effect of terrain-following coordinate transformations on numerical accuracy.  In this standard test, a tracer is positioned aloft and transported horizontally over wave-shaped terrain.  This test presents no particular challenge on cut cell meshes because there is no wind and zero tracer density near the ground \citep{good2014}.
Here we present a variation of this standard test case that challenges transport schemes on all mesh types: positioning the tracer near the ground and modifying the wind field so that it is tangential to terrain-following coordinate surfaces allows us to assess the accuracy of the cubicFit scheme near the lower boundary.

The domain is defined on an $x$--$z$ plane that is \SI{301}{\kilo\meter} wide and \SI{25}{\kilo\meter} high as measured between parallel boundary edges.  The domain is subdivided into a $301 \times 50$ mesh such that $\Delta x = \SI{1}{\kilo\meter}$ and $\Delta z = \SI{500}{\meter}$.

The terrain is wave-shaped, specified by the surface height $h$ such that
\begin{subequations}
\begin{align}
   h(x) &= h^\star \cos^2 ( \alpha x )
%
\intertext{where}
%
   h^\star(x) &= \left\{ \begin{array}{l l}
       h_0 \cos^2 ( \beta x ) & \quad \text{if $| x | < a$} \\
	0 & \quad \text{otherwise}
    \end{array} \right.
\end{align}
\end{subequations}
where $a = \SI{25}{\kilo\meter}$ is the mountain envelope half-width, $h_0 = \SI{3}{\kilo\meter}$ is the maximum mountain height, $\lambda = \SI{8}{\kilo\meter}$ is the wavelength, \(\alpha = \pi / \lambda\) and \(\beta = \pi / (2a)\).
Basic terrain following, cut cell and slanted cell meshes are constructed by modifying the uniform $301 \times 50$ mesh using this terrain profile.  The details of the various mesh generation methods were given in section~\ref{sec:meshes}.

\TODO{describe wind field}

A tracer with density $\phi$ is positioned upwind of the mountain at the ground.  It has the shape
\begin{align}
	\phi(x, z) &= \phi_0 \left\{ \begin{array}{l l}
		\cos^2 \left( \frac{\pi r}{2} \right) & \quad \text{if $r \leq 1$} \\
		0 & \quad \text{otherwise}
	\end{array} \right.
%
\intertext{with radius $r$ given by}
%
	r &= \sqrt{
		\left( \frac{x - x_0}{A_x} \right)^2 + 
		\left( \frac{z - z_0}{A_z} \right)^2
	}
\end{align}
where $A_x = \SI{25}{\kilo\meter}$, $A_z = \SI{10}{\kilo\meter}$ are the horizontal and vertical half-widths respectively, and $\phi_0 = \SI{1}{\kilogram\per\meter\cubed}$ is the maximum density of the tracer.  At $t = \SI{0}{\second}$, the tracer is centred at $(x_0, z_0) = (\SI{-50}{\kilo\meter}, \SI{0}{\kilo\meter})$ so that the tracer is upwind of the mountain and centred at the ground.
Tests are integrated forward in time for \SI{10000}{\second}.

\TODO{explain how we calculate the analytic solution}

\begin{itemize}
	\item Compare cubicFit with linearUpwind
	\item Compare errors on BTF, cut cells and slanted cells using a small timestep
	\item Show maximum timesteps for various mesh spacings using Courant number close to one
\end{itemize}

\begin{figure}
	\centering
	\includegraphics{../fig-mountainAdvection-meshes/fig-mountainAdvection-meshes.pdf}
	\caption{\TODO{BTF, cut cell and slanted cell meshes used for the slug over a mountain test}}
\end{figure}

\begin{figure}
	\centering
	\includegraphics{../fig-mountainAdvection-error/fig-mountainAdvection-error.pdf}
	\caption{\TODO{evolution of the slug over a mountain at $t=0$, $t=T/2$ and $t=T$.  mountain advection error contours for (left-to-right) BTF, cut cells and slanted cells; linearUpwind (top) and cubicFit (bottom).  Tracer contours 0.1.  Error contours 0.01.} \\
	\TODO{I could overlay l2 and linf errors onto these plots.  Might be nicer than tabulating them separately.}}
\end{figure}

\begin{figure}
	\centering
	\includegraphics{../fig-mountainAdvection-maxdt/fig-mountainAdvection-maxdt.pdf}
	\caption{\TODO{mountain advection maximum timesteps for BTF, cut cells and slanted cells for various mesh spacings.  Demonstrates first that cubicFit has no problems near the limit of stability and, second, that slanted cells scale predictably with mesh spacing.}}
\end{figure}



\subsection{Deformational flow on a sphere}
To ensure that the cubicFit transport scheme is suitable for complex flows on a variety of meshes, we use a standard test of deformational flow on a spherical Earth \citep{lauritzen2012}.  
The standard test in \citep{lauritzen2012} comprised six elements
\begin{enumerate}
\item a convergence test using a Gaussian-shaped tracer
\item a ``minimal'' resolution test using a cosine-shaped tracer
\item a test of filament preservation
\item a test using a ``rough'' slotted cylinder tracer
\item a test of correlation preservation between two tracers
\item a test using a divergent velocity field
\end{enumerate}
We assess the cubicFit scheme using only tests 1, 2 and 6.  We do not consider filament preservation or the transport of a ``rough'' slotted cylinder because no shape-preserving filter has yet been developed for cubicFit.  \TODO{why is tracer correlation out of scope for this paper?}
Results are compared between linearUpwind and cubicFit schemes using icoshedral meshes and cubed-sphere meshes.

\TODO{how much detail do I need about OpenFOAM's global Cartesian coordinates, lack of 2D meshes and our correction for spherical geometry?}


\subsubsection{Numerical order of convergence using Gaussian hills}
% 

\begin{figure}
	\centering
	\includegraphics{../fig-deformationSphere-initialTracer/fig-deformationSphere-initialTracer.pdf}
	\caption{\TODO{evolution of deformational flow test cases for Gaussian hills with plots at $t=0$, $t=T/2$ and $t=T$.  The analytic solution at $t=T$ is identical to the initial condition.  Cosine bells initial condition also plotted.  This figure is supposed to give a sense of what `should' happen, so plot at a high resolution using whichever mesh gives better results.}}
\end{figure}

\begin{figure}
	\centering
	\includegraphics{../fig-deformationSphere-gaussiansConvergence/fig-deformationSphere-gaussiansConvergence.pdf}
	\caption{\TODO{deformational flow l2 and linf convergence plots comparing cubed sphere and hexagons, cubicFit and linearUpwind.  This figure is comparable to \citet{lauritzen2012} figure 4.}}
\end{figure}

\subsubsection{``Minimal'' resolution using cosine bells}

\begin{figure}
	\centering
	\includegraphics{../fig-deformationSphere-cosBellsConvergence/fig-deformationSphere-cosBellsConvergence.pdf}
	\caption{\TODO{$\ell_2$ convergence for non-divergent deformational flow using Cosine bells.  Used to find ``minimal'' resolution.  Plot for hexagons and cubed sphere, cubicFit and linearUpwind.  Plot a heavy line for minimal resolution, as in \citet{lauritzen2012} figure 5.}}
\end{figure}

\subsubsection{Transport under divergent flow conditions using cosine bells}

\begin{figure}
	\centering
	\includegraphics{../fig-deformationSphere-divergentTracer/fig-deformationSphere-divergentTracer.pdf}
	\caption{\TODO{divergent flow at $t=T/2$ and $t=T$ comparing cubed sphere and hexagons, cubicFit and linearUpwind.  Corresponds to \citet{lauritzen2012} figure 9.}}
\end{figure}



\section{Conclusions}

The advection scheme is
\begin{itemize}
	\item suitable for complex flows on a variety of meshes
	\item computationally cheap at runtime, with more expensive computations depending only on the mesh geometry
	\item \TODO{convergence}
	\item stable for Courant numbers up to 1
\end{itemize}

\section{Acknowledgements}
\TODO{Supervisors, funding bodies.  ASAM group for the mesh generator---I should ask permission to use cut cell meshes in this paper.  Dr Tristan Pryer.  Dr Shing Hing Man.}

\section*{Appendix A: One-dimensional von Neumann stability analysis}
Two analyses are performed in order to find stability constraints on the weights $\mathbf{w} = \mathbf{\tilde{b}_1^+} \cdot \mathbf{m}$ as appear in equation~\eqref{eqn:weightedPinv}.  The first analysis uses two points to derive separate constraints on the upwind weight $w_u$ and downwind weight $w_d$.  The second analysis uses three points to derive a constraint that considers all weights in a stencil.

\subsection*{Two-point analysis}
We start with the conservation equation for a dependent variable $\phi$ that is discrete-in-space and continuous-in-time
\begin{align}
\frac{\partial \phi_j}{\partial t} &= - u \frac{\phi_R - \phi_L}{\Delta x} \label{eqn:advectionLR} \\
%
\intertext{where the left and right fluxes, $\phi_L$ and $\phi_R$, are weighted averages of the neighbouring points.  Assuming that $u$ is positive}
%
\phi_L &= \alpha_u \phi_{j-1} + \alpha_d \phi_j \\
\phi_R &= \beta_u \phi_j + \beta_d \phi_{j+1}
\end{align}
where $\alpha_u$ and $\beta_u$ are the upwind weights and $\alpha_d$ and $\beta_d$ are the downwind weights for the left and right fluxes respectively, and $\alpha_u + \alpha_d = 1$ and $\beta_u + \beta_d = 1$.  A subscript $j$ denotes the value at a given point $x = j \Delta x$ where $\Delta x$ is a uniform mesh spacing.

At a given time $t = n \Delta t$ at time-level $n$ and with a time-step $\Delta t$, we assume a wave-like solution with an amplification factor $A$, such that
\begin{align}
	\phi_j^{(n)} &= A^n e^{\iu j k \Delta x} \label{eqn:vn}
\end{align}
where $\phi_j^{(n)}$ denotes a value of $\phi$ at position $j$ and time-level $n$.  Using this to rewrite the left-hand side of equation~\eqref{eqn:advectionLR}
\begin{align}
\frac{\partial \phi_j}{\partial t} &= \frac{\partial}{\partial t} \left( A^{t / \Delta t} \right) e^{ijk\Delta x} = \frac{\ln A}{\Delta t} A^n e^{ikj\Delta x} \\
%
\shortintertext{hence equation~\eqref{eqn:advectionLR} becomes}
%
\frac{\ln A}{\Delta t} &= - \frac{u}{\Delta x} \left( \beta_u + \beta_d e^{ik\Delta x} - \alpha_u e^{-ik\Delta x} - \alpha_d \right) \\
\ln A &= -c \left( \beta_u - \alpha_d + \beta_d \cos k\Delta x + \iu \beta_d \sin k \Delta x - \alpha_u \cos k\Delta x + \iu \alpha_u \sin k\Delta x \right)
%
\intertext{where the Courant number $c = u \Delta t / \Delta x$.
Let $\Re = \beta_u - \alpha_d + \beta_d \cos k\Delta x - \alpha_u \cos k\Delta x$ and
$\Im = \beta_d \sin k \Delta x + \alpha_u \sin k\Delta x$, then}
%
\ln A &= -c \left( \Re + \iu \Im \right) \\
A &= e^{-c \Re} e^{-\iu c \Im} \\
%
\shortintertext{and the complex modulus of $A$ is}
%
|A| &= e^{-c \Re} = \exp \left( -c \left( \beta_u - \alpha_d + \left(\beta_d - \alpha_u \right) \cos k\Delta x \right) \right) \text{ .}
\end{align}
For stability we need $|A| \leq 1$ and, imposing the additional constraints that $\alpha_u = \beta_u$ and $\alpha_d = \beta_d$, then
\begin{align}
\left( \alpha_u - \alpha_d \right) \left( 1 - \cos k\Delta x \right) &\geq 0 \quad \forall k\Delta x
%
\shortintertext{and, given $0 \leq 1 - \cos k \Delta x \leq 2$, then}
%
\alpha_u - \alpha_d \geq 0 \text{ .} \label{eqn:twopoint-lower}
\end{align}
Additionally, we do not want more damping than a first-order upwind scheme (where $\alpha_u = \beta_u = 1$, $\alpha_d = \beta_d = 0$), having an amplification factor, $A_\mathrm{up}$, so we need $|A| \geq |A_\mathrm{up}|$, hence
\begin{align}
	\exp \left( -c \left(\alpha_u - \alpha_d\right) \left( 1 - \cos k\Delta x \right) \right) &\geq \exp \left( -c \left(1 - \cos k\Delta x \right) \right) \quad \forall k\Delta x
%
\shortintertext{therefore}
%
	\alpha_u - \alpha_d &\leq 1 \text{ .} \label{eqn:twopoint-upper}
\end{align}
Now, knowing that $\alpha_u + \alpha_d = 1$ (or $\alpha_d = 1 - \alpha_u$) then, using equations~\eqref{eqn:twopoint-lower} and \eqref{eqn:twopoint-upper},
\begin{align}
	0.5 \leq \alpha_u &\leq 1 \text{,} \label{eqn:vn:upwind} \\
	0 \leq \alpha_d &\leq 0.5 \label{eqn:vn:downwind} \text{ .}
\end{align}

\subsection*{Three-point analysis}
We start again from equation~\eqref{eqn:advectionLR} but this time approximate $\phi_L$ and $\phi_R$ using three points,
\begin{align}
	\phi_L &= \alpha_{uu} \phi_{j-2} + \alpha_u \phi_{j-1} + \alpha_d \phi_j \\
	\phi_R &= \alpha_{uu} \phi_{j-1} + \alpha_u \phi_j + \alpha_d \phi_{j+1}
\end{align}
having used the same weights $\alpha_{uu}$, $\alpha_u$ and $\alpha_d$ for both left and right fluxes.
Substituting equation~\eqref{eqn:vn} into equation~\eqref{eqn:advectionLR} we find
\begin{align}
A = \exp\left( -c \left[ \alpha_{uu} \left( e^{-ik\Delta x} - e^{-2ik\Delta x} \right) + \alpha_u \left( 1 - e^{-ik\Delta x} \right) + \alpha_d \left( e^{ik\Delta x} - 1 \right) \right] \right)
%
\intertext{so that, if the complex modulus $|A| \leq 1$ then}
%
\alpha_u - \alpha_d + \left( \alpha_{uu} - \alpha_u + \alpha_d \right) \cos k\Delta x - \alpha_{uu} \cos 2k\Delta x \geq 0 \text{ .}
\end{align}
If $k\Delta x = \pi$ then $\cos k\Delta x = -1$ and $\cos 2k\Delta x = 1$ and $\alpha_u - \alpha_d \geq \alpha_{uu}$.  If $k\Delta x = \pi / 2$ then $\cos k\Delta x = 0$ and $\cos 2k\Delta x = -1$ and $\alpha_u - \alpha_d \geq -\alpha_{uu}$.  Hence we find that
\begin{align}
	\alpha_u - \alpha_d &\geq |\alpha_{uu}| \label{eqn:uuConstraint} \text{ .}
%
\intertext{When the same analysis is performed with four points, $\alpha_{uuu}$, $\alpha_{uu}$, $\alpha_u$ and $\alpha_d$, by varying $k \Delta x$ we find that equation~\eqref{eqn:uuConstraint} still holds.
We also find that the same condition holds replacing $\alpha_{uu}$ with $\alpha_{uuu}$.  Hence, we generalise equation~\eqref{eqn:uuConstraint} to find the final stability constraint}
%
	\alpha_u - \alpha_d &\geq \max_{p\:\in\:P} |\alpha_p|
\end{align}
where the peripheral cells $P$ is the set of all stencil cells except for the upwind cell and downwind cell, and $\alpha_p$ is the weight for a given peripheral cell $p$.
We hypothesise that the three stability constraints (equations~\ref{eqn:vn:upwind}, \ref{eqn:vn:downwind} and \ref{eqn:uuConstraint}) are necessary but not sufficient for a transport scheme on arbitrarily-structured meshes.


\bibliographystyle{ametsoc2014}
\bibliography{references}

\end{document}
