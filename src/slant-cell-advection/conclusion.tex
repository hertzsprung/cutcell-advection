\section{Conclusion}
\label{sec:conclusion}

Atmospheric models are using increasingly fine horizontal mesh spacings that resolve steep slopes in terrain resulting in highly-distorted meshes, increased numerical errors and numerical instabilities.
We have presented a new multidimensional method-of-lines transport scheme, cubicFit, that applies constraints derived from a von Neumann stability analysis to make the scheme stable over steep terrain on \revone{highly-distorted, arbitrary meshes}.
The scheme has a low computational cost at runtime, requiring only $n$ multiplies per face per \revtwo{time-stage} using a stencil with $n$ cells.  Stability constraint calculations are pre-computed during model initialisation since they depend upon the mesh geometry only.

The cubicFit scheme was compared to a multidimensional linear upwind scheme using three idealised numerical tests.
The first test transported a tracer horizontally above steep slopes on highly-distorted, two-dimensional terrain-following meshes.  The cubicFit scheme was second-order convergent regardless of mesh distortions.
The second test transported a tracer over a mountainous lower boundary using terrain-following, cut cell and slanted cell meshes.
The cubicFit scheme was generally insensitive to the type of mesh and less sensitive to terrain steepness compared to the multidimensional linear upwind scheme.
The cubicFit scheme maintained accuracy up to the stability limit of a Courant number of \TODO{nnn}.
The third test evaluated the transport schemes in a standard deformational flow field on \revone{hexagonal-icosahedral meshes} and cubed-sphere meshes.
In all tests, compared to the multidimensional linear upwind scheme, the cubicFit transport scheme was more stable and more accurate.

