\documentclass[times]{elsarticle}
\usepackage{fullpage}
\usepackage{graphicx}
\usepackage{amsmath}
\usepackage{mathtools}
\usepackage{xcolor}
\usepackage{bm}
\usepackage{natbib}
\usepackage[hidelinks]{hyperref}
\usepackage{doi}
\usepackage[final,babel]{microtype}
\usepackage[utf8]{inputenc}
\usepackage[british]{babel}
\usepackage{csquotes}
\usepackage[T1]{fontenc}
\usepackage{siunitx}
\usepackage[font={small}]{caption}

\newcommand{\iu}{{i\mkern1mu}}
\newcommand{\iunit}{\boldsymbol{\hat \imath}}
\newcommand{\junit}{\boldsymbol{\hat \jmath}}
\newcommand{\kunit}{\boldsymbol{\hat k}}
\newcommand{\TODO}[1]{\textcolor{purple}{TODO: \emph{#1}}}

\begin{document}

\begin{frontmatter}
\title{An Eulerian finite volume transport scheme for atmospheric flows on arbitrary meshes over steep terrain \\ \TODO{(working title)}}
\author[uor]{James Shaw\corref{cor}}
\ead{js102@zepler.net}
\author[uor]{Hilary Weller}

\cortext[cor]{Corresponding author}
\address[uor]{Department of Meteorology, University of Reading, Reading, United Kingdom}

\begin{abstract}
The inclusion of terrain in atmospheric models gives rise to mesh distortions and nonuniformities near the lower boundary that can degrade the accuracy of transport schemes.  In addition, accuracy may be compromised because of stringent time-to-solution constraints on operational weather forecast models.

This paper presents an Eulerian finite volume transport scheme, ``cubicFit'', that uses a multidimensional polynomial, least squares reconstruction method.  Constraints derived from a von Neumann stability analysis are imposed during model initialisation to remove high-order polynomial terms in distorted regions of the mesh, or near boundaries where there is insufficient information.  This technique achieves stable, non-oscillatory solutions on arbitrarily distorted, nonuniform meshes.  The least squares reconstruction method is computationally cheap because most calculations depend upon the mesh only.

The cubicFit scheme is evaluated using two, idealised numerical tests of atmospheric flow.  The first test assesses accuracy near the ground by transporting a tracer over steep terrain on severely distorted terrain-following meshes and cut cell meshes.  The second test deforms a tracer in non-divergent and divergent flows on hexagonal icosahedra and cubed-sphere meshes.
In all tests, cubicFit is largely insensitive to mesh distortions, and cubicFit results are more accurate than those obtained using a standard linear upwind transport scheme.
\end{abstract}

\begin{keyword}
	\TODO{keywords}
\end{keyword}
\end{frontmatter}

\section{Introduction}
% models having problems over steep slopes:
% schaer2002, Alps (MAP), small-scale disturbances to theta and humidity in mid/upper troposphere
Numerical simulations of atmospheric flows solve equations of motion that result in the transport of momentum, temperature, water species and trace gases.  The numerical representation of Earth's terrain complicates the transport problem because the mesh is necessarily distorted by the modification of the lower boundary.
As new atmospheric models use increasingly fine mesh spacing, meshes are able to resolve steep, small-scale slopes.  Numerical schemes in operational weather forecast models can perform poorly over large mountain ranges, exhibiting small-scale numerical noise in temperature, humidity \citep{schaer2002} and potential vorticity fields \citep{hoinka-zaengl2004}, or even violating the Courant--Friedrich--Lewy stability constraint resulting in so-called `grid-point storms' \citep{webster2003}.
Further, operational weather forecasts must balance numerical accuracy and computational efficiency to meet strict time-to-solution constraints.
Hence, an efficient transport scheme is desired that yields accurate solutions, particularly near the surface where the weather affects us directly.
We present a new transport scheme that is generally insensitive to mesh distortions created by steep slopes, and achieves computational efficiency because most calculations are not repeated every time-step because they depend upon the mesh geometry only.

There are two main methods for representing terrain in atmospheric models: terrain-following layers and cut cells.  Both methods modify regular quadrilateral meshes to produce irregular meshes with cells that are aligned in vertical columns.  Most operational models use terrain-following layers in which horizontal mesh surfaces are moved upwards to accommodate the terrain.  A vertical decay function is chosen so that mesh surfaces slope less steeply with increasing height.
The most straightforward is the linear decay function used by the basic terrain-following transform \citep{galchen-somerville1975} (also called the sigma coordinate), but many atmospheric models suffer from large numerical errors on such meshes \citep{schaer2002,klemp2011,eckermann2014}.
To reduce such errors, more complex decay functions have been developed so that mesh surfaces are smoother \citep{simmons-burridge1981,schaer2002,leuenberger2010,klemp2011}.

An alternative to terrain-following layers is the cut cell method.  Cut cell meshes are constructed by `cutting' a regular quadrilateral mesh with a piecewise-linear representation of the terrain.  New vertices are created where the terrain intersects mesh edges, and cell volumes that lie beneath the ground are removed.  Cut cell meshes can have arbitrarily small cells that impose severe timestep constraints on explicit transport schemes.  Several techniques have been developed to allieviate this problem, known as the `small-cell problem': small cells can be merged with adjacent cells \citep{yamazaki2016}, cell volumes can be artificially increased \citep{steppeler2002}, or an implicit scheme can be used near the ground with an explicit scheme used aloft \citep{jebens2011}.

Another method for avoiding the small-cell problem was proposed by \citep{shaw-weller2016} in which cell vertices are moved vertically so that they are positioned at the terrain surface.  In this paper the method is referred to as the slanted cell method in order to distinguish it from the traditional cut cell method.  Slanted cell meshes do not suffer from arbitrarily small cells because the horizontal cell dimensions are not modified by the presence of terrain.

Smoothed terrain-following layers, cut cells and slanted cell methods all reduce the amount of mesh distortion but the presence of terrain means that any mesh must necessarily be distorted, at least near the ground.
Even when distortions are minimal, transport across mesh surfaces tends to be more common near steep slopes, and this misalignment between the flow and mesh surfaces increases numerical errors \citep{schaer2002,shaw-weller2016}.
\TODO{segue into talking about transport schemes...}

%As well as vertical distortions over sloping terrain, transport schemes must also deal with horizontal mesh distortions on the sphere.  A wide variety of spherical meshes are used in global models \citep{staniforth-thuburn2012}.  The latitude--longitude mesh is in widespread use because it is orthogonal and can be represented by a computationally rectangular mesh.  However, small cells near the poles constrain the timestep for explicit methods, and parallel computation becomes inefficient.  Because of these problems there has been increased interest in alternative, quasi-uniform meshes.

%The quasi-uniform cubed-sphere mesh retains the quadrilateral cells found on the latitude--longitude mesh, and improves uniformity at the expense of orthogonality.  Starting with a cube with each face being a Cartesian mesh, the cubed-sphere mesh is constructed by a gnomic projection of each face onto the sphere.
%Another class of quasi-uniform spherical mesh is constructed by successive refinement of an icosahedron.  The resultant mesh can be modified in different ways to reduce skewness and hence improve numerical accuracy \citep{heikes-randall1995b,tomita2002}.  A thorough review of spherical meshes for atmospheric models is presented in \citep{staniforth-thuburn2012}.

Transport schemes may be classified as dimensionally-split, 2D multidimensional, or 3D multidimensional.  2D multidimensional schemes are often simply called `multidimensional' schemes since very few 3D multidimensional schemes have been used in atmospheric models.
Perhaps confusingly, dimensionally-split schemes are sometimes called multidimensional, too, because they use one-dimensional techniques for multidimensional transport.

Dimensionally-split schemes such as \citep{lin-rood1996,putman-lin2007,katta2015} calculate transport in each dimension separately before the flux contributions are combined.  Such schemes are computationally efficient and allow existing one-dimensional high-order methods to be used.  To use a dimensionally-split scheme over terrain, a terrain-following coordinate transform is typically required.
  Dimensionally-split schemes can suffer from `splitting errors' in which the tracer is artificially distorted when the velocity field is misaligned with the grid \citep{leonard1993}.  Errors can be reduced by explicitly accounting for transverse fluxes when combining fluxes \citep{leonard1996}, but splitting errors are nevertheless apparent in flows over steep terrain where meshes are highly distorted and metric terms in the coordinate transform are large \citep{weller2017}.

In 2D multidimensional schemes the horizontal dimensions are considered together.  Such schemes are extended for three-dimensional transport by using splitting techniques in the vertical dimension.
There are several subclasses of 2D multidimensional schemes that include
2D semi-Lagrangian finite volume schemes (also called conservative mesh remapping),
swept-area schemes (also called flux-form semi-Lagrangian, incremental remapping, or forward-in-time),
and method-of-lines schemes.
2D semi-Lagrangian finite volume schemes such as \citep{iske-kaeser2004,lauritzen2010} integrate over departure cells that are found by tracing backward the trajectories of cell vertices.  These schemes are conservative because departure cells are constructed so that there are no overlaps or gaps, which requires that cell areas are simply-connected domains \citep{lauritzen2011book}.
Swept area schemes such as \citep{lashley2002,skamarock-menchaca2010,lauritzen2011,thuburn2014} calculate the flux through a cell face by integrating over the upstream area that is swept out over one time-step.  Such schemes differ in their choice of area approximation, sub-grid reconstruction, and spatial integration method.
Because swept area schemes integrate over the reconstructed field, they typically require a matrix-vector multiply per face \citep{thuburn2014,skamarock-menchaca2010}.
Method-of-lines schemes such as \citep{weller2009,skamarock-gassmann2011} use a spatial discretisation to reduce the transport PDE to an ODE that is typically solved using a multi-stage timestepping method.  
Unlike 2D semi-Lagrangian finite volume schemes, swept area and method-of-lines schemes achieve conservation for non-simply connected domains that can result from small-scale rotational flows \citep{lauritzen2011}.

Very few 3D multidimensional schemes have been used in atmospheric models \citep[e.g.][]{gassmann2013} although such schemes might be expected to be more accurate on horizontally non-orthogonal, spherical meshes with steep terrain.
%yeh2007 discusses generalisation to 3D?
Additionally, the multidimensional method-of-lines schemes developed in \citep{weller-shahrokhi2014} has been used in two-dimensional flows on Cartesian $x$--$z$ planes with distorted meshes \citep{shaw-weller2016,weller2017}.
There are many more types of atmospheric transport schemes, but all can be classified according to their treatment of the three spatial dimensions.  A more comprehensive overview is presented in \cite{lauritzen2014}.

In this paper, we present a new multidimensional method-of-lines scheme, `cubicFit', that improves upon the scheme in \citep{weller-shahrokhi2014}.  To reconstruct face values, the scheme fits a multidimensional polynomial over a cubic, upwind-biased stencil using a least-squares approach.  The implementation uses constraints derived from a von Neumann stability analysis to select appropriate polynomial fits for stencils in highly-distorted mesh regions.  The least-squares procedure depends upon the mesh geometry only, hence the scheme is computationally efficient, requiring only $n$ multiplies per cell face per time-step where $n$ is the size of the stencil.
\TODO{outline contents of subsequent sections}


%\TODO{talk about accuracy near the ground: what is there in the literature about this?  there's a bunch of stuff that's not met-specific}
% walko-avissar Weak oscillations generated near the lower boundary in OLAM 
% almgren1997 says that losing 2nd order accuracy at boundary should not be a problem for atmospheric flows over orography -- **do we think this is (still) true?**
% ye1999 achieve 2nd order near the boundary but use dimensional splitting
% zaengl2012: ICON, noisy small-scale circulations with weak/no ambient flow, see zaengl2002, zaengl2004, klemp2011


% something about why we want accuracy near the ground for weather forecasts: clouds can form near the surface, fog, pollution etc
% describe existing classes of transport schemes used in atmospheric models (use lauritzen2014)
% existing uses of least squares fit




cubicFit motivation
\begin{itemize}
	\item finite volume is desirable because it's inherently conservative, \TODO{what else?  FV3 do it, ECMWF are doing it}
	\item a transport scheme for arbitrary meshes is desirable because different meshes offer different trade-offs (\TODO{e.g. provide example for hexagonal icosahedra vs cubed-sphere and citation}).  supporting arbitrary meshes lets us defer the choice of mesh.
	\item predicting meteorological variables at or near the surface is important because it's where we live, clouds can form near the surface, fog, pollution etc $\Rightarrow$ accurate advection is desirable
\end{itemize}

\TODO{why isn't linearUpwind good enough?  what other schemes compete with cubicFit and what can cubicFit offer in comparison with these competitors?}

\subsection{Review of existing techniques}
\textbf{What types of transport scheme are used in atmospheric models?} \\
\citep{lauritzen2014} provide a useful overview.

\textbf{How are weighted least-squares polynomial reconstructions (LSPR) used elsewhere?}
\begin{itemize}
	\item OpenFOAM has a variety of surface interpolation methods that use WLSPR, but I don't know what contexts they're used in
	\item \citep{lashley2002} developed a swept-area transport scheme that uses WLSPR.  For time-varying velocity fields, a matrix-vector multiply must be calculated for every face at every timestep.
	\item \citep{skamarock-menchaca2010} also developed a swept-area scheme using WLSPR closely following \citep{lashley2002}.
	\item \citep{skamarock-gassmann2011} use LSPR to calculate second derivatives that are used in high-order finite volume flux calculations.  Like us, coefficients are precomputed and only vector-vector multiplies are performed at runtime.  Also like us, they use multi-stage Runge-Kutta timestepping.
	\item \citep{jayantha-turner2003} use WLSPR to approximate fluxes of $\nabla \phi$ at faces in the 2D unsteady orthotropic diffusion equation
	\item Moving least squares reconstructions are applied to shallow-water equations \citep{cuetofelgueroso2006} and viscous Navier-Stokes \citep{cuetofelgueroso2007}.  They are ``moving'' because the local stencil is moved to the position of reconstruction.  They also treat cell-centre averages as point values, just as we do.
	\citep{cuetofelgueroso2007} also note that most calculations can be precomputed for static meshes.
	However, much of the description and mathematics is beyond me right now.  
	\item Another study used a least-squares reconstruction for viscous fluxes in the advection--diffusion equation \citep{olliviergooch-vanaltena2002}.  They describe a technique for constraining the fit near curved boundaries.
\end{itemize}

\textbf{Have von Neumann analyses ever been used in online numerical codes before?} \\
I've yet to find an instance where this has occurred.


\TODO{2nd order schemes that are more accurate than centred linear approximations: lipscomb-ringler2005, yeh2007, miura2007}

\section{Types of meshes}
\TODO{might ditch this section and just describe things in the intro/results sections}

\subsection{Meshing a spherical Earth}

\TODO{Describe hexagonal icosahedral meshes (cite one or more of \citep{thuburn2014,heikes-randall1995a,heikes-randall1995b})} \\
\TODO{Describe cubed-sphere mesh (cite someone?)}

Following appendix A9 in \citep{lauritzen2014}, the average equatorial spacing $\Delta \lambda$ is used as a measure of mesh spacing.  It is defined as
\begin{align}
	\Delta \lambda = \ang{360} \frac{\overline{\Delta x}}{2 \pi R_e}
\end{align}
where $\overline{\Delta x}$ is the mean distance between cell centres and $R_e = \SI{6.3712e6}{\meter}$ is the radius of the Earth.

\section{Transport schemes for arbitrary meshes}
The cubicFit transport scheme is described here for arbitrary two-dimensional meshes and arbitrary, single-layer spherical meshes.  Section \ref{sec:results} compares results using the cubicFit scheme with results using the linearUpwind transport scheme, and so a description of the linearUpwind scheme is also provided here.

\TODO{move the derivation from the advection equation and timestepping details to here since they're common to both cubicFit and linearUpwind}

\TODO{somewhere in here we need to state the multidimensional Courant number and note that the Courant number may vary between cells belonging to the same mesh.  We should then say that the stability limit for both schemes is $\max{\mathrm{Co}} \leq 1$.}

\input{cubicFit}

\subsection{linearUpwind transport scheme}
\TODO{describe OpenFOAM's linearUpwind scheme and cite something? OpenFOAM docs? OpenFOAM github?}
\TODO{I need to mention that we use a Dirichlet boundary condition of $\phi = \SI{0}{\kilogram\per\meter\cubed}$ at the ground.  Results are unstable if I use  zero gradient boundary condition.}

\begin{figure}
	\caption{\TODO{I'll probably need a figure that shows linearUpwind's stencil and shows how it approximates $\phi_F$}}
\end{figure}



\section{Results}
\label{sec:results}

\TODO{somewhere mention that the second-order convergence is a limitation of the divergence discretisation.  With more DoF a higher order should be achievable.} \\
\TODO{should I do eigenmode analysis?  this would prove stability for particular meshes for arbitrary wind fields} \\
\TODO{somewhere define error norms}

\subsection{Transport over a mountainous lower boundary}
This test is a modification of the \citet{schaer2002} horizontal advection test.  The mountain height is raised, the wind field is aligned with the the terrain-following surfaces, and the tracer is moved downward so that it is advected over the ground.

\begin{itemize}
	\item Compare cubicFit with linearUpwind
	\item Compare errors on BTF, cut cells and slanted cells using a small timestep
	\item Show maximum timesteps for various mesh spacings using Courant number close to one
\end{itemize}

\begin{figure}
	\centering
	\includegraphics{../fig-mountainAdvection-meshes/fig-mountainAdvection-meshes.pdf}
	\caption{\TODO{BTF, cut cell and slanted cell meshes used for the slug over a mountain test}}
\end{figure}

\begin{figure}
	\centering
	\includegraphics{../fig-mountainAdvection-error/fig-mountainAdvection-error.pdf}
	\caption{\TODO{evolution of the slug over a mountain at $t=0$, $t=T/2$ and $t=T$.  mountain advection error contours for (left-to-right) BTF, cut cells and slanted cells; linearUpwind (top) and cubicFit (bottom).  Tracer contours 0.1.  Error contours 0.01.} \\
	\TODO{I could overlay l2 and linf errors onto these plots.  Might be nicer than tabulating them separately.}}
\end{figure}

\begin{figure}
	\centering
	\includegraphics{../fig-mountainAdvection-maxdt/fig-mountainAdvection-maxdt.pdf}
	\caption{\TODO{mountain advection maximum timesteps for BTF, cut cells and slanted cells for various mesh spacings.  Demonstrates first that cubicFit has no problems near the limit of stability and, second, that slanted cells scale predictably with mesh spacing.}}
\end{figure}



\subsection{Deformational flow on a sphere}

\TODO{how much detail do I need about OpenFOAM's global Cartesian coordinates, lack of 2D meshes and our correction for spherical geometry?}

Tests on cubed sphere and hexagons, again comparing cubicFit against linearUpwind.  \citet{lauritzen2012} had six classes of test and we will reproduce three of them:
\begin{enumerate}
	\item convergence tests with Gaussian hills
	\item minimal resolution test with cosine bell
	\item cosine bell in divergent flow
\end{enumerate}
We will not consider filament preservation, a "rough" distribution with a slotted cylinder, or correlation preservation.

\subsubsection{Numerical order of convergence using Gaussian hills}
\begin{figure}
	\centering
	\includegraphics{../fig-deformationSphere-initialTracer/fig-deformationSphere-initialTracer.pdf}
	\caption{\TODO{evolution of deformational flow test cases for Gaussian hills with plots at $t=0$, $t=T/2$ and $t=T$.  The analytic solution at $t=T$ is identical to the initial condition.  Cosine bells initial condition also plotted.  This figure is supposed to give a sense of what `should' happen, so plot at a high resolution using whichever mesh gives better results.}}
\end{figure}

\begin{figure}
	\centering
	\includegraphics{../fig-deformationSphere-gaussiansConvergence/fig-deformationSphere-gaussiansConvergence.pdf}
	\caption{\TODO{deformational flow l2 and linf convergence plots comparing cubed sphere and hexagons, cubicFit and linearUpwind.  This figure is comparable to \citet{lauritzen2012} figure 4.}}
\end{figure}

\subsubsection{``Minimal'' resolution using cosine bells}

\begin{figure}
	\centering
	\includegraphics{../fig-deformationSphere-cosBellsConvergence/fig-deformationSphere-cosBellsConvergence.pdf}
	\caption{\TODO{$\ell_2$ convergence for non-divergent deformational flow using Cosine bells.  Used to find ``minimal'' resolution.  Plot for hexagons and cubed sphere, cubicFit and linearUpwind.  Plot a heavy line for minimal resolution, as in \citet{lauritzen2012} figure 5.}}
\end{figure}

\subsubsection{Transport under divergent flow conditions using cosine bells}

\begin{figure}
	\centering
	\includegraphics{../fig-deformationSphere-divergentTracer/fig-deformationSphere-divergentTracer.pdf}
	\caption{\TODO{divergent flow at $t=T/2$ and $t=T$ comparing cubed sphere and hexagons, cubicFit and linearUpwind.  Corresponds to \citet{lauritzen2012} figure 9.}}
\end{figure}



\section{Conclusions}

The advection scheme is
\begin{itemize}
	\item suitable for complex flows on a variety of meshes
	\item computationally cheap at runtime, with more expensive computations depending only on the mesh geometry
	\item \TODO{convergence}
	\item stable for Courant numbers up to 1
\end{itemize}

\section{Acknowledgements}
\TODO{Supervisors, funding bodies.  ASAM group for the mesh generator---I should ask permission to use cut cell meshes in this paper.  Dr Tristan Pryer.  Dr Shing Hing Man.}

\section*{Appendix: One-dimensional von Neumann stability analysis}
Two analyses are performed in order to find stability constraints on the weights $\mathbf{w} = \mathbf{\tilde{b}_1^+} \cdot \mathbf{m}$ as appear in equation~\eqref{eqn:weightedPinv}.  The first analysis uses two points to derived separate constraints on the upwind weight $w_u$ and downwind weight $w_d$.  The second analysis uses three points to derive a constraint that considers all weights in a stencil.

\subsection*{Two-point analysis}
We start with the conservation equation for a dependent variable $\phi$ that is discrete-in-space and continuous-in-time
\begin{align}
\frac{\partial \phi_j}{\partial t} &= - u \frac{\phi_R - \phi_L}{\Delta x} \label{eqn:advectionLR} \\
%
\shortintertext{where the left and right fluxes, $\phi_L$ and $\phi_R$ are weighted averages of the neighbouring points.  Assuming that $u$ is positive}
%
\phi_L &= \alpha_u \phi_{j-1} + \alpha_d \phi_j \\
\phi_R &= \beta_u \phi_j + \beta_d \phi_{j+1}
\end{align}
where $\alpha_u$ and $\beta_u$ are the upwind weights and $\alpha_d$ and $\beta_d$ are the downwind weights for the left and right fluxes respectively.  A subscript $j$ denotes the value at a given point $x = j \Delta x$ where $\Delta x$ is a uniform mesh spacing.

At a given time $t = n \Delta t$ at time-level $n$ and with a time-step $\Delta t$, we assume a wave-like solution with an amplification factor $A$, such that
\begin{align}
	\phi_j^{(n)} &= A^n e^{\iu j k \Delta x}
\end{align}
where $\phi_j^{(n)}$ denotes a value of $\phi$ at position $j$ and time-level $n$.  Using this to rewrite the left-hand side of equation~\eqref{eqn:advectionLR}
\begin{align}
\frac{\partial \phi_j}{\partial t} &= \frac{\partial}{\partial t} \left( A^{t / \Delta t} \right) e^{ijk\Delta x} = \frac{\ln A}{\Delta t} A^n e^{ikj\Delta x} \\
%
\shortintertext{hence equation~\eqref{eqn:advectionLR} becomes}
%
\frac{\ln A}{\Delta t} &= - \frac{u}{\Delta x} \left( \beta_u + \beta_d e^{ik\Delta x} - \alpha_u e^{-ik\Delta x} - \alpha_d \right) \\
\ln A &= -c \left( \beta_u - \alpha_d + \beta_d \cos k\Delta x + \iu \beta_d \sin k \Delta x - \alpha_u \cos k\Delta x + \iu \alpha_u \sin k\Delta x \right)
%
\intertext{where the Courant number $c = u \Delta t / \Delta x$.
Let $\Re = \beta_u - \alpha_d + \beta_d \cos k\Delta x - \alpha_u \cos k\Delta x$ and
$\Im = \beta_d \sin k \Delta x + \alpha_u \sin k\Delta x$, then}
%
\ln A &= -c \left( \Re + \iu \Im \right) \\
A &= e^{-c \Re} e^{-\iu c \Im} \\
%
\shortintertext{and the complex modulus and complex argument of $A$ are found to be}
%
|A| &= e^{-c \Re} = \exp \left( -c \left( \beta_u - \alpha_d + \left(\beta_d - \alpha_u \right) \cos k\Delta x \right) \right) \quad \text{and} \\
\arg(A) &= -c \Im = -c \left( \beta_d + \alpha_u \right) \sin k\Delta x
\end{align}
For stability, we need $|A| \leq 1$ and $\arg(A) < 0$ for $c > 0$, so
\begin{align}
\beta_u - \alpha_d + \left( \beta_d - \alpha_u \right) \cos k\Delta x &\geq 0 \quad \forall k\Delta x \quad \text{and} \\
\beta_d + \alpha_u &> 0 \label{eqn:arg-ineq}
\end{align}
Imposing the additional constraints that $\alpha_u = \beta_u$ and $\alpha_d = \beta_d$:
\begin{align}
|A| &= \exp \left( -c \left( \alpha_u - \alpha_d \right) \left(1 - \cos k\Delta x \right) \right)
%
\intertext{and given $1 - \cos k\Delta x \geq 0$ for well-resolved waves, then}
%
\alpha_u - \alpha_d &\geq 0 \\
%
\shortintertext{which provides a lower bound on $\alpha_u$:}
\alpha_u &\geq \alpha_d \label{eqn:lower-bound}
\end{align}
Additionally, we do not want more damping than an upwind scheme (where $\alpha_u = \beta_u = 1$, $\alpha_d = \beta_d = 0$), having an amplification factor, $A_\mathrm{up}$:
\begin{align}
|A_\mathrm{up}| &= \exp \left( -c \left(1 - \cos k\Delta x \right) \right)
%
\intertext{So we need $|A| \geq |A_\mathrm{up}|$:}
%
-c \left( \alpha_u - \alpha_d \right) \left(1 - \cos k\Delta x \right) &\geq -c \left( 1- \cos k\Delta x \right) \\
\alpha_u - \alpha_d &\leq 1 \\
\alpha_u &\leq 1 + \alpha_d
%
\intertext{which provides an upper bound on $\alpha_u$.  Combining with eqn~\eqref{eqn:lower-bound} we can bound $\alpha_u$ on both sides:}
%
\alpha_d \leq \alpha_u \leq 1 + \alpha_d
\end{align}
Now, assume that $\alpha_u + \alpha_d = 1$ (or $\alpha_d = 1 - \alpha_u$), then
\begin{align}
	1 - \alpha_u < \alpha_u &\leq 1 + (1 - \alpha_u) \\
	0.5 \leq \alpha_u &\leq 1
%
\shortintertext{and, since $\alpha_u + \alpha_d = 1$, then}
%
	0 \leq \alpha_d &\leq 0.5
\end{align}


\subsection*{Three-point analysis}


\bibliographystyle{ametsoc2014}
\bibliography{references}

\end{document}
