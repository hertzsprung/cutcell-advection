\subsection{Deformational flow on a sphere}
To ensure that the cubicFit transport scheme is suitable for complex flows on a variety of meshes, we use a standard test of deformational flow on a spherical Earth \citep{lauritzen2012}.  
The standard test in \citep{lauritzen2012} comprised six elements
\begin{enumerate}
\item a convergence test using a Gaussian-shaped tracer
\item a ``minimal'' resolution test using a cosine-shaped tracer
\item a test of filament preservation
\item a test using a ``rough'' slotted cylinder tracer
\item a test of correlation preservation between two tracers
\item a test using a divergent velocity field
\end{enumerate}
We assess the cubicFit scheme using only tests 1, 2 and 6.  We do not consider filament preservation or the transport of a ``rough'' slotted cylinder because no shape-preserving filter has yet been developed for cubicFit.  \TODO{why is tracer correlation out of scope for this paper?}
Results are compared between linearUpwind and cubicFit schemes using icoshedral meshes and cubed-sphere meshes.

\TODO{how much detail do I need about OpenFOAM's global Cartesian coordinates, lack of 2D meshes and our correction for spherical geometry?}


\subsubsection{Numerical order of convergence using Gaussian hills}
% 

\begin{figure}
	\centering
	\includegraphics{../fig-deformationSphere-initialTracer/fig-deformationSphere-initialTracer.pdf}
	\caption{\TODO{evolution of deformational flow test cases for Gaussian hills with plots at $t=0$, $t=T/2$ and $t=T$.  The analytic solution at $t=T$ is identical to the initial condition.  Cosine bells initial condition also plotted.  This figure is supposed to give a sense of what `should' happen, so plot at a high resolution using whichever mesh gives better results.}}
\end{figure}

\begin{figure}
	\centering
	\includegraphics{../fig-deformationSphere-gaussiansConvergence/fig-deformationSphere-gaussiansConvergence.pdf}
	\caption{\TODO{deformational flow l2 and linf convergence plots comparing cubed sphere and hexagons, cubicFit and linearUpwind.  This figure is comparable to \citet{lauritzen2012} figure 4.}}
\end{figure}

\subsubsection{``Minimal'' resolution using cosine bells}

\begin{figure}
	\centering
	\includegraphics{../fig-deformationSphere-cosBellsConvergence/fig-deformationSphere-cosBellsConvergence.pdf}
	\caption{\TODO{$\ell_2$ convergence for non-divergent deformational flow using Cosine bells.  Used to find ``minimal'' resolution.  Plot for hexagons and cubed sphere, cubicFit and linearUpwind.  Plot a heavy line for minimal resolution, as in \citet{lauritzen2012} figure 5.}}
\end{figure}

\subsubsection{Transport under divergent flow conditions using cosine bells}

\begin{figure}
	\centering
	\includegraphics{../fig-deformationSphere-divergentTracer/fig-deformationSphere-divergentTracer.pdf}
	\caption{\TODO{divergent flow at $t=T/2$ and $t=T$ comparing cubed sphere and hexagons, cubicFit and linearUpwind.  Corresponds to \citet{lauritzen2012} figure 9.}}
\end{figure}

