\documentclass[times]{elsarticle}
\usepackage{fullpage}
\usepackage{graphicx}
\usepackage{amsmath}
\usepackage{mathtools}
\usepackage{xcolor}
\usepackage{bm}
\usepackage{natbib}
\usepackage[hidelinks]{hyperref}
\usepackage{doi}
\usepackage[final,babel]{microtype}
\usepackage[utf8]{inputenc}
\usepackage[british]{babel}
\usepackage{csquotes}
\usepackage[T1]{fontenc}
\usepackage{siunitx}
\usepackage[font={small}]{caption}

\newcommand{\iu}{{i\mkern1mu}}
\newcommand{\iunit}{\boldsymbol{\hat \imath}}
\newcommand{\junit}{\boldsymbol{\hat \jmath}}
\newcommand{\kunit}{\boldsymbol{\hat k}}
\newcommand{\TODO}[1]{\textcolor{purple}{TODO: \emph{#1}}}

\begin{document}

\begin{frontmatter}
\title{An Eulerian finite volume transport scheme for atmospheric flows on arbitrary meshes over steep terrain}
\author[uor]{James Shaw\corref{cor}}
\ead{js102@zepler.net}
\author[uor]{Hilary Weller}
\author[uor]{John Methven}
\author[mo]{Terry Davies}

\cortext[cor]{Corresponding author}
\address[uor]{Department of Meteorology, University of Reading, Reading, United Kingdom}
\address[mo]{Met Office, Exeter, United Kingdom}

\begin{abstract}
The inclusion of terrain in atmospheric models gives rise to mesh distortions near the lower boundary that can degrade the accuracy and challenge the stability of transport schemes \TODO{apart from linearUpwind results, what evidence do I have that other transport schemes struggle with stability over steep slopes?}.
In addition, accuracy may be compromised because of stringent time-to-solution constraints on operational weather forecast models.

This paper presents an Eulerian finite volume transport scheme, ``cubicFit'', that uses a multidimensional polynomial, least squares reconstruction method.  Constraints derived from a von Neumann stability analysis are imposed during model initialisation to remove high-order terms from the reconstruction polynomial in distorted regions of the mesh, or near boundaries where there is insufficient information.
This technique achieves stable, non-oscillatory solutions on arbitrarily distorted, non-uniform meshes.  The least squares reconstruction method has a low computational cost because most calculations depend upon the mesh only, with just one vector multiply per face needed per time-step.

The cubicFit scheme is evaluated using two, idealised numerical tests of atmospheric flow.  The first test assesses accuracy near the ground by transporting a tracer over steep terrain on severely distorted terrain-following meshes and cut-cell meshes.  The second test deforms a tracer in non-divergent and divergent flows on hexagonal icosahedra and cubed-sphere meshes.
In all tests, cubicFit is largely insensitive to mesh distortions, and cubicFit results are more accurate than those obtained using a standard linear upwind transport scheme.
\end{abstract}

\begin{keyword}
	\TODO{keywords}
\end{keyword}
\end{frontmatter}

\section{Introduction}
\section{Introduction}

Numerical simulations of atmospheric flows solve equations of motion that result in the transport of momentum, temperature, water species and trace gases.  The numerical representation of Earth's terrain complicates the transport problem because the mesh is necessarily distorted by the modification of the lower boundary.
As new atmospheric models use increasingly fine mesh spacing, meshes are able to resolve steep, small-scale slopes.  Numerical schemes in operational weather forecast models can perform poorly over large mountain ranges, exhibiting small-scale numerical noise in momentum \citep{walko-avissar2008b}, temperature, humidity \citep{schaer2002} and potential vorticity fields \citep{hoinka-zaengl2004}, or even violating the Courant--Friedrich--Lewy stability constraint resulting in so-called `grid-point storms' \citep{webster2003}.
A transport scheme is desired that yields stable and accurate solutions, particularly near the surface where the weather affects us directly.
We present a new transport scheme which is numerically stable on arbitrary meshes and which is generally insensitive to mesh distortions created by steep slopes.  It has a low computational cost since most calculations are not repeated every time-step because they depend upon the mesh geometry only.

There are two main methods for representing terrain in atmospheric models: terrain-following layers and cut cells.  Both methods modify regular quadrilateral meshes to produce distorted meshes with cells that are aligned in vertical columns.  Most operational models use terrain-following layers in which horizontal mesh surfaces are moved upwards to accommodate the terrain.  A vertical decay function is chosen so that mesh surfaces slope less steeply with increasing height.
The most straightforward is the linear decay function used by the basic terrain-following transform \citep{galchen-somerville1975} (also called the sigma coordinate), but many atmospheric models suffer from large numerical errors on such meshes \citep{schaer2002,klemp2011,eckermann2014}.
To reduce such errors, more complex decay functions have been developed so that mesh surfaces are smoother \citep{simmons-burridge1981,schaer2002,leuenberger2010,klemp2011}.

An alternative to terrain-following layers is the cut cell method.  Cut cell meshes are constructed by `cutting' a regular quadrilateral mesh with a piecewise-linear representation of the terrain.  New vertices are created where the terrain intersects mesh edges, and cell volumes that lie beneath the ground are removed.  Cut cell meshes can have arbitrarily small cells that impose severe timestep constraints on explicit transport schemes.  Several techniques have been developed to allieviate this problem, known as the `small-cell problem': small cells can be merged with adjacent cells \citep{yamazaki2016}, cell volumes can be artificially increased \citep{steppeler2002}, or an implicit scheme can be used near the ground with an explicit scheme used aloft \citep{jebens2011}.

Another method for avoiding the small-cell problem was proposed by \citep{shaw-weller2016} in which cell vertices are moved vertically so that they are positioned at the terrain surface.  In this paper the method is referred to as the slanted cell method in order to distinguish it from the traditional cut cell method.  Slanted cell meshes do not suffer from arbitrarily small cells because the horizontal cell dimensions are not modified by the presence of terrain.

Smoothed terrain-following layers, cut cells and slanted cell methods all reduce the amount of mesh distortion but any mesh that represents sloping terrain must necessarily be distorted, at least near the ground.
Even when distortions are minimal, transport across mesh surfaces tends to be more common near steep slopes, and this misalignment between the flow and mesh surfaces increases numerical errors \citep{leonard1993,schaer2002,shaw-weller2016}.
A huge variety of transport schemes have been developed for atmospheric models, but few are able to account for distortions associated with steep terrain because they treat horizontal and vertical transport separately \citep{kent2014}, resulting in numerical errors called `splitting errors'.
Such errors can be reduced by explicitly accounting for transverse fluxes when combining fluxes \citep{leonard1996}, but splitting errors are still apparent in flows over steep terrain where meshes are highly distorted and metric terms in a terrain-following coordinate transform are large \citep{weller2017}.

Transport schemes are often classified as dimensionally-split or multidimensional.
Dimensionally-split schemes such as \citep{lin-rood1996,katta2015} calculate transport in each dimension separately before the flux contributions are combined.  Such schemes are computationally efficient and allow existing one-dimensional high-order methods to be used.  To use a dimensionally-split scheme over terrain, a terrain-following coordinate transform is required.
Perhaps confusingly, dimensionally-split schemes are sometimes called multidimensional, too, because they use one-dimensional techniques for multidimensional transport.

Unlike dimensionally-split schemes, multidimensional schemes consider transport in two or three dimensions together.
There are several subclasses of multidimensional schemes that include
2D semi-Lagrangian finite volume schemes (also called conservative mesh remapping),
swept-area schemes (also called flux-form semi-Lagrangian, incremental remapping, or forward-in-time),
and method-of-lines schemes (also called Eulerian schemes).
2D semi-Lagrangian finite volume schemes such as \citep{iske-kaeser2004,lauritzen2010} integrate over departure cells that are found by tracing backward the trajectories of cell vertices.  These schemes are conservative because departure cells are constructed so that there are no overlaps or gaps, which requires that cell areas are simply-connected domains \citep{lauritzen2011book}.
Swept area schemes such as \citep{lashley2002,skamarock-menchaca2010,lauritzen2011,thuburn2014} calculate the flux through a cell face by integrating over the upstream area that is swept out over one time-step.  Such schemes differ in their choice of area approximation, sub-grid reconstruction, and spatial integration method.
Because swept area schemes integrate over the reconstructed field, they typically require a matrix-vector multiply per face \citep{thuburn2014,skamarock-menchaca2010}.
Method-of-lines schemes such as \citep{weller2009,skamarock-gassmann2011} use a spatial discretisation to reduce the transport PDE to an ODE that is typically solved using a multi-stage timestepping method.  
Unlike 2D semi-Lagrangian finite volume schemes, swept area and method-of-lines schemes achieve conservation for non-simply connected domains that can result from small-scale rotational flows \citep{lauritzen2011}.
There are many more types of atmospheric transport schemes, but all can be classified according to their treatment of the three spatial dimensions.  A more comprehensive overview is presented in \cite{lauritzen2014}.

For transport schemes that are ordinarily classified as `multidimensional', a further distinction ought to made between horizontally-multidimensional and three-dimensional schemes.
Multidimensional schemes are almost always horizontally-multidimensional because, while the two horizontal dimensions are considered together, horizontal and vertical transport are still treated separately.
Very few three-dimensional schemes have been suggested for use in atmospheric models \citep[e.g.][]{miura2007,yeh2007,gassmann2013} although such schemes might be expected to be more accurate on distorted meshes with steep terrain.
The multidimensional scheme developed in \citep{weller-shahrokhi2014} is unusual because it has no horizontal--vertical splitting and it has been used in two-dimensional flows on Cartesian $x$--$z$ planes with distorted meshes \citep{shaw-weller2016,weller2017}.

In this paper, we present a new multidimensional method-of-lines scheme, `cubicFit', that improves upon the scheme in \citep{weller-shahrokhi2014} and avoids all splitting errors.  To reconstruct values at cell faces, the scheme fits a multidimensional polynomial over a cubic, upwind-biased stencil using a least-squares approach.  The implementation uses constraints derived from a von Neumann stability analysis to select appropriate polynomial fits for stencils in highly-distorted mesh regions.  Almost all of the least-squares procedure depends upon the mesh geometry only and reconstruction weights can be pre-computed without knowledge of the velocity field or tracer field.  Hence, the cubicFit scheme has a low computational cost that is comparable to dimensionally-split schemes, requiring only $n$ multiplies per cell face per time-step where $n$ is the size of the stencil.

The remainder of this paper is organised as follows.
Section~\ref{sec:transport} starts by discretising the transport equation using a method-of-lines approach, before describing the cubicFit transport scheme and a standard multidimensional linear upwind transport scheme.
Section~\ref{sec:results} evaluates the cubicFit scheme using three idealised numerical tests.
The first standard test follows Sch\"ar et al. \citep{schaer2002}, transporting a tracer horizontally above steep mountains on two-dimensional, highly-distorted terrain-following meshes.
The second is a new test case designed to assess numerical accuracy next to a mountainous lower boundary.  In this test, a tracer placed at the ground is transported over steep slopes using terrain-following, cut cell and slanted cell meshes.
The third is a standard test of deformational flow on a spherical Earth, specified by Lauritzen et al. \citep{lauritzen2012}, which we use to assess the cubicFit transport scheme on hexagonal icosahedra and cubed-sphere meshes.
Concluding remarks are made in section~\ref{sec:conclusion}.



\section{Transport schemes for arbitrary meshes}
\label{sec:transport}

The transport of a dependent variable $\phi$ in a prescribed wind field $\mathbf{u}$ is given by the equation
\begin{align}
	\frac{\partial \phi}{\partial t} + \nabla \cdot \left( \mathbf{u} \phi \right) = 0 \label{eqn:advection}
\end{align}
In the mass continuity equation, $\phi = \rho$ where $\rho$ is the air density.  In the continuity equation for tracer density $\phi = \rho q$ where $q$ is the tracer mixing ratio.  For the special case of a non-divergent velocity field then $\rho$ is constant hence $\phi = q$.
The time derivative is discretised using a two-stage, second-order Heun method,
\begin{subequations}
\begin{align}
	\phi^\star &= \phi^{(n)} + \Delta t f(\phi^{(n)}) \\
	\phi^{(n+1)} &= \phi^{(n)} + \frac{\Delta t}{2} \left[ f(\phi^{(n)} + f(\phi^{\star}) \right]
\end{align}
\end{subequations}
where \(f(\phi^{(n)}) = - \nabla \cdot (\mathbf{u} \phi^{(n)})\) at time level \(n\).  The same time-stepping method is used for both cubicFit and linearUpwind schemes.

Using the finite volume method, the wind field is prescribed at face centroids and the dependent variable is stored at cell centroids.  The divergence term in equation~\eqref{eqn:advection} is discretised using Gauss's theorem:
\begin{align}
	\nabla \cdot \left( \mathbf{u} \phi \right) \approx \frac{1}{\mathcal{V}_c} \sum_{f \in\:c} \mathbf{u}_f \cdot \mathbf{S}_f \phi_F \label{eqn:gauss-div}
\end{align}
where subscript $f$ denotes a value stored at a face and subscript $F$ denotes a value approximated at a face from surrounding values stored at cell centres.  $\mathcal{V}_c$ is the cell volume, $\mathbf{u}_f$ is a velocity vector prescribed at a face, ${\mathbf{S}_f}$ is the surface area vector with a direction outward normal to the face and a magnitude equal to the face area, $\phi_F$ is an approximation of the dependent variable at the face, and $\sum_{f \in\:c}$ denotes a summation over all faces $f$ bordering cell $c$.
Note that equation~\eqref{eqn:gauss-div} is a second-order approximation of the divergence term that limits the cubicFit transport scheme to second-order accuracy.

This discretisation is applicable to arbitrary meshes.  A neccesary condition for stability is given by the multidimensional Courant number,
\begin{align}
	\mathrm{Co}_c = \frac{\Delta t}{2 \mathcal{V}_c} \sum_{f \in\: c} \mathbf{u} \cdot \mathbf{S}_f
\end{align}
such that $\mathrm{Co}_c \leq 1$ for all cells $c$ in the domain.  Hence, stability is constrained by the maximum Courant number of any cell in the domain.

The accurate approximation of the dependent variable at the face, $\phi_F$, is key to the overall accuracy of the transport scheme. The cubicFit and linearUpwind schemes differ in their approximations of $\phi_F$, and these approximation methods described next.


\subsection{Cubic fit transport scheme}

\begin{figure}
	\centering
	\includegraphics{../fig-interior-stencils/fig-interior-stencils.pdf}
	\caption{Upwind-biased stencils for faces far away from the boundaries of two-dimensional (a) rectangular and (b) hexagon meshes.  The stencil is used to fit a multidimensional polynomial to cell centre values, $\phi_c$, marked by grey circles, in order to approximate the value $\phi_F$ at the face centroid marked by an open circle.  $\phi_u$ and $\phi_d$ are the values at the centroids of the upwind and downwind cells neighbouring the target face, drawn with a heavy line.  The velocity vector $\mathbf{u}_f$ is prescribed at face $f$ and determines the choice of stencil at each time-step.}
	\label{fig:interiorStencils}
\end{figure}

The cubicFit scheme approximates the value of the dependent variable at the face, $\phi_F$, using a least squares fit over a stencil of surrounding known values.
To introduce the approximation method, we will consider how an approximate value is calculated for a face that is far away from the boundaries of a two-dimensional uniform rectangular mesh.
For any mesh, every interior face connects two adjacent cells.  The velocity direction at the face determines which of the two adjacent cells is the upwind cell.  Since the stencil is upwind-biased and asymmetric, two stencils must be constructed for every interior face, and the appropriate stencil is chosen depending on the velocity direction at each face for every time-step.

The upwind-biased stencil for a face $f$ is shown in figure~\ref{fig:interiorStencils}a.  The wind at the face, $\mathbf{u}_f$, is blowing from the upwind cell $c_u$ to the downwind cell $c_d$.
To obtain an approximate value at $f$, a polynomial least squares fit is calculated using the stencil values.
The stencil has \num{4} points in $x$ and \num{3} points in $y$, leading to a natural choice of polynomial that is cubic in $x$ and quadratic in $y$,
\begin{align}
	\phi = a_1 + a_2 x + a_3 y + a_4 x^2 + a_5 xy + a_6 y^2 + a_7 x^3 + a_8 x^2 y + a_9 x y^2 \label{eqn:fullPoly} \text{ .}
\end{align}
A least squares approach is needed because the system of equations is overconstrained, with \num{12} stencil values but only \num{9} polynomial terms.  The stencil geometry is expressed in a local coordinate system with the face centroid as the origin so that the approximated value $\phi_F$ is equal to the constant coefficient $a_1$.
\revtwo{The stencil is upwind-biased to improve numerical stability, and the multidimensional cubic polynomial is chosen to improve accuracy in the direction of flow \citep{leonard1993}.}

The remainder of this subsection generalises the approximation technique for arbitrary meshes and describes the methods for constructing stencils, performing a least squares fit with a suitable polynomial, and ensuring numerical stability of the transport scheme.

\subsubsection{Stencil construction}
For every interior face, two stencils are constructed, one for each of the possible upwind cells.
Stencils are not constructed for boundary faces because values of $\phi$ at boundaries are calculated from prescribed boundary conditions.
For a given interior face $f$ and upwind cell $c_u$, we find those faces that are connected to $c_u$ and `oppose' face $f$.  These are called the \textit{opposing faces}.
The opposing faces for face $f$ and upwind cell $c_u$ are determined as follows.
Defining $G$ to be the set of faces other than $f$ that border cell $c_u$, we calculate the `opposedness', $\mathrm{Opp}$, between faces $f$ and $g \in G$, defined as
\begin{align}
	\mathrm{Opp}(f, g) \equiv - \frac{\mathbf{S}_f \cdot \mathbf{S}_g}{|\mathbf{S}_f|^2} \label{eqn:opp}
\end{align}
where $\mathbf{S}_f$ and $\mathbf{S}_g$ are the surface area vectors pointing outward from cell $c_u$ for faces $f$ and $g$ respectively.
Using the fact that $\mathbf{a} \cdot \mathbf{b} = |\mathbf{a}|\:|\mathbf{b}| \cos(\theta)$ we can rewrite equation~\eqref{eqn:opp} as
\begin{align}
	\mathrm{Opp}(f, g) = - \frac{|\mathbf{S}_g|}{|\mathbf{S}_f|} \cos(\theta)
\end{align}
where $\theta$ is the angle between faces $f$ and $g$.  In this form, it can be seen that $\mathrm{Opp}$ is a measure of the relative area of $g$ and how closely it parallels face $f$.

The set of opposing faces, $\mathrm{OF}$, is a subset of $G$, comprising those faces with $\mathrm{Opp} \geq 0.5$, and the face with the maximum opposedness.  Expressed in set notation, this is
\begin{align}
	\mathrm{OF}(f,c_u) \equiv \{ g : \mathrm{Opp}(f, g) \geq 0.5 \} \cup \{ g : \max_{g\:\in\:G}(\mathrm{Opp}(f, g)) \} \text{ .}
\end{align}
On a rectangular mesh, there is always one opposing face $g$, and it is exactly parallel to the face $f$ such that $\mathrm{Opp}(f, g) = 1$.

Once the opposing faces have been determined, the set of internal and external cells must be found.  The \textit{internal cells} are those cells that are connected to the opposing faces.  Note that $c_u$ is always an internal cell.  The \textit{external cells} are those cells that share vertices with the internal cells.  Note that $c_d$ is always an external cell.  Finally, the \textit{stencil boundary faces} are boundary faces having Dirichlet boundary conditions\footnote{Boundary faces with Neumann boundary conditions would require extrapolated boundary values to be calculated.
This would create a feedback loop in which boundary values are extrapolated from interior values, then interior values are transported using stencils that include boundary values.  
We have not considered how such an extrapolation \revtwo{could be made} consistent with the multidimensional polynomial reconstruction.
Using an inconsistent extrapolation, we found that the feedback loop is able to generate slow-growing numerical instabilities near some highly-distorted cells.
Hence, boundary faces with Neumann boundary conditions are excluded from the set of stencil boundary faces.} that share a vertex with the internal cells.
Having found these three sets, the stencil is constructed to comprise all internal cells, external cells and stencil boundary faces.


\begin{figure}
	\centering
	\includegraphics{../fig-double-upwind-stencil/fig-double-upwind-stencil.pdf}
	%
	\caption{A fourteen-point, upwind-biased stencil for face $f$ connecting the pentagonal upwind cell, $c_u$, and the downwind cell $c_d$.  The dashed lines denote the two faces of cell $c_u$ that oppose $f$, and black circles mark the centroids of the internal cells that are connected to these two opposing faces.  The stencil is extended outwards by including cells that share vertices with the three internal cells, where black squares mark these vertices.  Four stencil boundary faces, marked by black triangles, are also included.
The local coordinate system $(x, y)$ has its origin at the centroid of face $f$, marked by an open circle, with $x$ normal to $f$ and $y$ perpendicular to $x$.}
	\label{fig:double-upwind-stencil}
\end{figure}

Figure~\ref{fig:double-upwind-stencil} illustrates a stencil construction for face $f$ connecting upwind cell $c_u$ and downwind cell $c_d$.  The two opposing faces are denoted by thick dashed lines and the centres of the three adjoining internal cells are marked by black circles.  The stencil is extended outwards by including the external cells that share vertices with the internal cells, where the vertices are marked by black squares.  A boundary at the far left has \revone{Dirichlet} boundary conditions, and so the four stencil boundary faces are also included in the stencil, where the boundary face centres are marked by black triangles.  The resultant stencil contains fourteen points.


\subsubsection{Least squares fit}
To approximate the value of $\phi$ at a face $f$, a least squares fit is calculated from a stencil of surrounding known values.  First, we will show how a polynomial least squares fit is calculated for a face on a rectangular mesh.  Second, we will make modifications to the least squares fit that are necessary for numerical stability.  

For faces that are far away from the boundaries of a rectangular mesh, we fit the multidimensional polynomial given by equation~\eqref{eqn:fullPoly} that has nine unknown coefficients, $\mathbf{a} = a_1 \ldots a_9$, using the twelve cell centre values from the upwind-biased stencil, $\bm{\phi} = \phi_1 \ldots \phi_{12}$.  This yields a matrix equation
\begin{align}
	\begin{bmatrix}
		1 & x_1 & y_1 & x_1^2 & x_1 y_1 & y_1^2 & x_1^3 & x_1^2 y_1 & x_1 y_1^2 \\
		1 & x_2 & y_2 & x_2^2 & x_2 y_2 & y_2^2 & x_2^3 & x_2^2 y_2 & x_2 y_2^2 \\
		\vdots & \vdots & \vdots & \vdots & \vdots & \vdots & \vdots & \vdots & \vdots \\
		1 & x_{12} & y_{12} & x_{12}^2 & x_{12} y_{12} & y_{12}^2 & x_{12}^3 & x_{12}^2 y_{12} & x_{12} y_{12}^2 \\
	\end{bmatrix}
	\begin{bmatrix}
		a_1 \\
		a_2 \\
		\vdots \\
		a_9
	\end{bmatrix}
	=
	\begin{bmatrix}
		\phi_1 \\
		\phi_2 \\
		\vdots \\
		\phi_{12}
	\end{bmatrix}
\end{align}
which can be written as
\begin{align}
	\mathbf{B} \mathbf{a} = \bm{\phi} \label{eqn:unweightedLeastSquares} \text{ .}
\end{align}
The rectangular matrix $\mathbf{B}$ has one row for each cell in the stencil and one column for each term in the polynomial.  $\mathbf{B}$ is called the \textit{stencil matrix}, and it is constructed using only the mesh geometry.
A local coordinate system is established in which $x$ is normal to the face $f$ and $y$ is perpendicular to $x$.
The coordinates $(x_i, y_i)$ give the position of the centroid of the $i$th cell in the stencil.
A two-dimensional stencil is also used for the tests on spherical meshes in section~\ref{sec:deformationSphere}.  In these tests, \revthree{cell centres are projected perpendicular to a tangent plane is defined at the face centre.  Previous studies found that results were largely insensitive to the projection method \citep{skamarock-gassmann2011,lashley2002}.}

The unknown coefficients $\mathbf{a}$ are calculated using the pseudo-inverse, $\mathbf{B}^+$, found by singular value decomposition,
\begin{align}
	\mathbf{a} = \mathbf{B}^+ \bm{\phi} \text{ .}
%
\intertext{Recall that the approximate value $\phi_F$ is equal to the constant coefficient $a_1$, which is a weighted mean of $\bm{\phi}$,} 
%
	a_1 = \begin{bmatrix}
		b_{1,1}^+ \\
		b_{1,2}^+ \\
		\vdots \\
		b_{1,12}^+
	\end{bmatrix}
	\cdot
	\begin{bmatrix}
		\phi_1 \\
		\phi_2 \\
		\vdots \\
		\phi_{12}
	\end{bmatrix} \label{eqn:cubicFit:weighted-sum}
\end{align}
where the weights $b_{1,1}^+ \ldots b_{1,12}^+$ are the elements of the first row of $\mathbf{B}^+$.
Note that the majority of the least squares fit procedure depends on the mesh geometry only.  An implementation may precompute the pseudo-inverse for each stencil during model initialisation, and only the first row needs to be stored.  Since each face has two possible stencils depending on the orientation of the velocity relative to the face, the implementation stores two sets of weights for each face.
Knowledge of the values of $\bm{\phi}$ is only required to calculate the weighted mean given by equation~\eqref{eqn:cubicFit:weighted-sum}, which is evaluated once per face per \revtwo{time-stage}.

In the least squares fit presented above, all stencil values contributed equally to the polynomial fit.
It is necessary for numerical stability that the polynomial fits the cells connected to face $f$ more closely than other cells in the stencil, as shown by \citep{lashley2002,skamarock-menchaca2010}.
To achieve this, we allow each cell to make an unequal contribution to the least squares fit.
We assign an integer \textit{multiplier} to each cell in the stencil, $\mathbf{m} = m_1 \ldots m_{12}$, and multiply equation~\eqref{eqn:unweightedLeastSquares} to obtain
\begin{align}
	\mathbf{\tilde{B}} \mathbf{a} = \mathbf{m} \cdot \bm{\phi}
\end{align}
where $\mathbf{\tilde{B}} = \mathbf{M} \mathbf{B}$ and $\mathbf{M} = \mathrm{diag}(\mathbf{m})$.  The constant coefficient $a_1$ is calculated from the pseudo-inverse, $\mathbf{\tilde{B}}^+$,
\begin{align}
	a_1 = \mathbf{\tilde{b}_1^+} \cdot \mathbf{m} \cdot \bm{\phi} \label{eqn:weightedPinv}
\end{align}
where $\mathbf{\tilde{b}_1^+} = \tilde{b}_{1,1}^+ \ldots \tilde{b}_{1,12}^+$ are the elements of the first row of $\mathbf{\tilde{B}}^+$.
Again, $a_1$ is a weighted mean of $\bm{\phi}$, where the weights are now $\mathbf{\tilde{b}_1^+} \cdot \mathbf{m}$.  Values for $\mathbf{m}$ are chosen so that the cells connected to face $f$ make a greater contribution to the least squares fit, as discussed later in section~\ref{sec:stabilisation}.

For faces of a non-rectangular mesh, or faces that are near a boundary, the number of stencil points and number of polynomial terms may differ: a stencil will have one or more cells and, for two-dimensional meshes, its polynomial will have between one and nine terms.  Additionally, the polynomial cannot have more terms than its stencil has cells because this would lead to an underconstrained system of equations.  The procedure for choosing suitable polynomials is discussed next.

\subsubsection{Polynomial generation}
\begin{figure}
	\centering
	\includegraphics{../fig-boundary-stencils/fig-boundary-stencils.pdf}
	\caption{Upwind-biased stencils for faces near the lower boundary of a rectangular $x$--$z$ mesh, with (a) a $3 \times 2$ stencil for the face immediately adjacent to the lower boundary, and (b) a $3 \times 3$ stencil for the face immediately adjacent to the face in (a).  Each stencil belongs to the face marked by a thick line.  The local coordinate system is shown, having an $x$ direction normal to the face a $y$ direction tangent to the face.  For both stencils, attempting a least squares fit using the nine-term polynomial in equation~\eqref{eqn:fullPoly} would result in an underconstrained problem.
Here we have assumed that the lower boundary has a Neumann boundary condition and so \revone{lower} boundary faces are excluded from the stencils.}
	\label{fig:boundaryStencils}
\end{figure}

The majority of faces on a uniform two-dimensional mesh have stencils with more than nine cells.  For example, a rectangular mesh has 12 points (figure~\ref{fig:interiorStencils}a), and a hexagonal mesh has 10 points (figure~\ref{fig:interiorStencils}b).
In both cases, constructing a system of equations using the nine-term polynomial in equation~\eqref{eqn:fullPoly} leads to an overconstrained problem that can be solved using least squares.  However, this is not true for faces near boundaries: stencils that have fewer than nine cells (figure~\ref{fig:boundaryStencils}a) would result in an underconstrained problem, and stencils that have exactly nine cells may lack sufficient information to constrain high-order terms.  For example, the stencil in figure~\ref{fig:boundaryStencils}b lacks sufficient information to fit the $x^3$ term.  In such cases, it becomes necessary to perform a least squares fit using a polynomial with fewer terms.

For every stencil, we find a set of \textit{candidate polynomials} that do not result in an underconstrained problem.
In two dimensions, a candidate polynomial has some combination of between one and nine terms from equation~\eqref{eqn:fullPoly}.  There are two additional constraints that a candidate polynomial must satisfy.

First, high-order terms may be included in a candidate polynomial only if the lower-order terms are also included.
More precisely, let
\begin{align}
	M(x, y) = { x^i y^j : i,j \geq 0 \text{ and } i+j \leq 3 }
\end{align}
be the set of all monomials of degree at most \num{3} in $x, y$.
A subset $S$ of $M(x,y)$ is ``dense'' if, whenever $x^a y^b$ and $x^c y^d$ are in $S$ with $a \leq c$ and $b \leq d$, then $x^i y^j$ is also in $S$ for all $a < i < c$, $b < j < d$.
For example, the polynomial $\phi = a_1 + a_2 x + a_3 y + a_4 xy + a_5 x^2 + a_6 x^2 y$ is a dense subset of $M(x,y)$, but $\phi = a_1 + a_2 x + a_3 y + a_4 x^2 y$ is not because $x^2 y$ can be included only if $xy$ and $x^2$ are also included.

Second, a candidate polynomial must have a stencil matrix $\mathbf{B}$ that is full rank.  The matrix is considered full rank if its smallest singular value is greater than \num{1e-9}.
Using a polynomial with all nine terms and the stencil in figure~\ref{fig:boundaryStencils}b results in a rank-deficient matrix and so the nine-term polynomial is not a candidate polynomial.

The candidate polynomials are all the dense subsets of $M(x,y)$ that have a stencil matrix that is full rank.  The final stage of the cubicFit transport scheme selects a candidate polynomial and ensures that the least squares fit is numerically stable.

\subsubsection{Stabilisation procedure}
\label{sec:stabilisation}
So far, we have constructed a stencil and found a set of candidate polynomials.  Applying a least squares fit to any of these candidate polynomials avoids creating an underconstrained problem.  The final stage of the transport scheme chooses a suitable candidate polynomial and appropriate multipliers $\mathbf{m}$ so that the fit is numerically stable.

The approximated value $\phi_F$ is equal to $a_1$ which is calculated from equation~\eqref{eqn:weightedPinv}.  The value of $a_1$ is a weighted mean of $\bm{\phi}$ where $\mathbf{w} = \mathbf{\tilde{b}_1^+} \cdot \mathbf{m}$ are the weights.
If the cell centre values $\bm{\phi}$ are assumed to approximate a smooth field then we expect $\phi_F$ to be close to the values of $\phi_u$ and $\phi_d$, and expect $\phi_F$ to be insensitive to small changes in $\bm{\phi}$.  When the weights $\mathbf{w}$ have large magnitude then this is no longer true: $\phi_F$ becomes sensitive to small changes in $\bm{\phi}$ which can result in large, \revtwo{numerically unstable} departures from the smooth field $\bm{\phi}$.

\revtwo{To avoid numerical instabilities, an idealised, one-dimensional von Neumann analysis was performed, presented in appendix A.  The analysis is used to impose three stability constraints on the weights $\mathbf{w}$},
\begin{subequations}
\label{eqn:stability}
\begin{align}
	0.5 \leq w_u \leq 1 \label{eqn:stabilityU} \\
	0 \leq w_d \leq 0.5 \label{eqn:stabilityD} \\
	w_u - w_d \geq \max_{p\:\in\:P}(|w_p|)
\end{align}
\end{subequations}
where $w_u$ and $w_d$ are the weights for the upwind and downwind cells respectively.  The \textit{peripheral points} $P$ are the cells in the stencil that are not the upwind or downwind cells, and $w_p$ is the weight for a given peripheral point $p$.
 The upwind, downwind and peripheral weights sum to one such that $w_u + w_d + \sum_{p \in P} w_p = 1$.

The stabilisation procedure comprises three steps.  In the first step, the set of candidate polynomials is sorted in preference order so that candidates with more terms are preferred over those with fewer terms.
If there are multiple candidates with the same number of terms, the minimum singular value of $\mathbf{B}$ is calculated for each candidate, and an ordering is imposed such that the candidate with the larger minimum singular value is preferred.  This ordering ensures that the preferred candidate is the highest-order polynomial with the most information content.

In the second step, the most-preferred polynomial is taken from the list of candidates and the multipliers are assigned so that the upwind cell and downwind cell have multipliers $m_u = 2^{10}$ and $m_d = 2^{10}$ respectively, and all peripheral points have multipliers $m_p = 1$.  These multipliers are very similar to those used by \citep{lashley2002}, leading to a well-conditioned matrix $\mathbf{\tilde{B}}$ and a least squares fit in which the polynomial passes almost exactly through the upwind and downwind cell centre values.

In the third step, we calculate the weights $\mathbf{w}$ and evaluate them against the stability constraints given in equation~\eqref{eqn:stability}.  If any constraint is violated, the value of $m_d$ is halved and the constraints are evaluated with the new weights.  This step is repeated until the weights satisfy the stability constraints, or $m_d$ becomes smaller than one.  In practice, the constraints are satisified when $m_d$ is either small (between 1 and 4) or equal to $2^{10}$.  The upwind multiplier $m_u$ is fixed at $2^{10}$ and the peripheral multipliers $m_p$ are fixed at \num{1}.  If the constraints are still not satisfied, then we start again from the second step with the next polynomial in the candidate list. 

Finally, if no stable weights are found for any candidate polynomial, we revert to an upwind scheme such that $w_u = 1$ and all other weights are zero.  In our experiments we have not encountered any stencil for which this last resort is required.

\begin{figure}
	\centering
	\includegraphics{../fig-stabilisation/fig-stabilisation.pdf}
%
	\caption{\revone{One-dimensional least squares fits with} a stencil of five points using (a) a cubic polynomial with multipliers $m_u = 1024$, $m_d = 1024$ and $m_p = 1$, (b) a quadratic polynomial with the same multipliers, and (c) a quadratic polynomial with multipliers $m_u = 1024$, $m_d = 1$ and $m_p = 1$.  Notice that the curves in (a) and (b) fit almost exactly through the upwind and downwind points immediately adjacent to the $y$-axis, but in (c) the curve fits almost exactly only through the upwind point immediately to the left of the $y$-axis.  The point data are labelled with their respective weights.  Points that have failed one of the stability constraints in equation~\eqref{eqn:stability} are marked in red with italicised labels.  The upwind point is located at $(-1, 1.8)$ and the downwind point at $(0.62, 1.9)$, and the peripheral points are at $(-2.8, 2.4)$, $(-1.6, 2.7)$ and $(-1.2, 2.2)$.  The stabilisation procedure (section~\ref{sec:stabilisation}) calculates weights using only $x$ positions, and values of $\phi$ are included here for illustration only.}
	\label{fig:oscillatory1D}
\end{figure}

To illustrate the stabilisation procedure, figure~\ref{fig:oscillatory1D}a presents a one-dimensional example of a cubic polynomial fitted through five points, with the weight at each point printed beside it.
The stabilisation procedure only uses the $x$ positions of these points and does not use the values of $\phi$ themselves.  The $\phi$ values are included here for illustration only.
Hence, for a given set of $x$ positions, the same set of weights are chosen irrespective of the $\phi$ values.

For a one-dimensional cubic polynomial fit, the list of candidate polynomials in preference order is
\begin{align}
	\phi &= a_1 + a_2 x + a_3 x^2 + a_4 x^3 \label{eqn:cubicCandidate} \text{ ,} \\
	\phi &= a_1 + a_2 x + a_3 x^2 \label{eqn:quadCandidate} \text{ ,} \\
	\phi &= a_1 + a_2 x \text{ ,} \\
	\phi &= a_1 \text{ .}
\end{align}
We begin with the cubic equation~\eqref{eqn:cubicCandidate}.  The multipliers are chosen so that the polynomial passes almost exactly through the upwind and downwind points that are immediately to the left and right of the $y$-axis respectively.
The constraint on the upwind point is violated because $w_u = 1.822 > 1$ (equation~\ref{eqn:stabilityU}).  Reducing the downwind multiplier does not help to satisfy the constraint, so we start again with the quadratic equation~\eqref{eqn:quadCandidate}, and the new fit is presented in figure~\ref{fig:oscillatory1D}b.
Again, the multipliers are chosen to force the polynomial through the upwind and downwind points, but this violates the constraint on the downwind point because $w_d = 0.502 > 0.5$ (equation~\ref{eqn:stabilityD}).  This time, however, stable weights are found by reducing \revtwo{$m_d$} to one (figure~\ref{fig:oscillatory1D}c) and these are the weights that will be used to approximate $\phi_F$, where the polynomial intercepts the $y$-axis.


\subsection{Multidimensional linear upwind transport scheme}
The multidimensional linear upwind scheme, called ``linearUpwind'' hereafter, is documented here since it provides a baseline accuracy for the experiments in section~\ref{sec:results}.  The approximation of $\phi_F$ is calculated using a gradient reconstruction,
\begin{align}
	\phi_F &= \phi_u + \nabla_c\: \phi \cdot \left(\mathbf{x}_f - \mathbf{x}_c \right)
\end{align} 
where $\phi_u$ is the upwind value of $\phi$, and $\mathbf{x}_f$ and $\mathbf{x}_c$ are the position vectors of the face centroid and cell centroid respectively.
\TODO{the length of this vector should change when using the spherical correction, but doesn't.  don't worry about this for now}
The gradient $\nabla_c \:\phi$ is calculated using Gauss' theorem:
\begin{align}
	\nabla_c\: \phi = \frac{1}{\mathcal{V}_c} \sum_{f\in\:c} \tilde{\phi_F} \mathbf{S}_f \label{eqn:linearUpwind-grad}
\end{align}
where $\tilde{\phi_F}$ is linearly interpolated from the two neighbouring cells of face $f$.  For cells adjacent to boundaries having zero gradient boundary conditions, the boundary value is set to be equal to the cell centre value before equation~\eqref{eqn:linearUpwind-grad} is evaluated.
This implementation of the multidimensional linear upwind scheme is included in the OpenFOAM software distribution \citep{openfoam}.



\section{Results}
\label{sec:results}

\TODO{somewhere mention that the second-order convergence is a limitation of the divergence discretisation.  With more DoF a higher order should be achievable.} \\
\TODO{I might need to estimate the number of floating-point ops to enable a comparison between linearUpwind and cubicFit in terms of computational expense versus numerical accuracy}

\subsection{Transport over a mountainous lower boundary}
A two-dimensional transport test over mountains was developed in \citep{schaer2002} to study the effect of terrain-following coordinate transformations on numerical accuracy.  In this standard test, a tracer is positioned aloft and transported horizontally over wave-shaped terrain.  This test presents no particular challenge on cut cell meshes because there is no wind and zero tracer density near the ground \citep{good2014}.
Here we present a variation of this standard test case that challenges transport schemes on all mesh types: positioning the tracer near the ground and modifying the wind field so that it is tangential to terrain-following coordinate surfaces allows us to assess the accuracy of the cubicFit scheme near the lower boundary.

The domain is defined on an $x$--$z$ plane that is \SI{301}{\kilo\meter} wide and \SI{25}{\kilo\meter} high as measured between parallel boundary edges.  The domain is subdivided into a $301 \times 50$ mesh such that $\Delta x = \SI{1}{\kilo\meter}$ and $\Delta z = \SI{500}{\meter}$.

The terrain is wave-shaped, specified by the surface height $h$ such that
\begin{subequations}
\begin{align}
   h(x) &= h^\star \cos^2 ( \alpha x )
%
\intertext{where}
%
   h^\star(x) &= \left\{ \begin{array}{l l}
       h_0 \cos^2 ( \beta x ) & \quad \text{if $| x | < a$} \\
	0 & \quad \text{otherwise}
    \end{array} \right.
\end{align}
\end{subequations}
where $a = \SI{25}{\kilo\meter}$ is the mountain envelope half-width, $h_0 = \SI{3}{\kilo\meter}$ is the maximum mountain height, $\lambda = \SI{8}{\kilo\meter}$ is the wavelength, \(\alpha = \pi / \lambda\) and \(\beta = \pi / (2a)\).
Basic terrain following, cut cell and slanted cell meshes are constructed by modifying the uniform $301 \times 50$ mesh using this terrain profile.  The details of the various mesh generation methods were given in section~\ref{sec:meshes}.

\TODO{describe wind field}

A tracer with density $\phi$ is positioned upwind of the mountain at the ground.  It has the shape
\begin{align}
	\phi(x, z) &= \phi_0 \left\{ \begin{array}{l l}
		\cos^2 \left( \frac{\pi r}{2} \right) & \quad \text{if $r \leq 1$} \\
		0 & \quad \text{otherwise}
	\end{array} \right.
%
\intertext{with radius $r$ given by}
%
	r &= \sqrt{
		\left( \frac{x - x_0}{A_x} \right)^2 + 
		\left( \frac{z - z_0}{A_z} \right)^2
	}
\end{align}
where $A_x = \SI{25}{\kilo\meter}$, $A_z = \SI{10}{\kilo\meter}$ are the horizontal and vertical half-widths respectively, and $\phi_0 = \SI{1}{\kilogram\per\meter\cubed}$ is the maximum density of the tracer.  At $t = \SI{0}{\second}$, the tracer is centred at $(x_0, z_0) = (\SI{-50}{\kilo\meter}, \SI{0}{\kilo\meter})$ so that the tracer is upwind of the mountain and centred at the ground.
Tests are integrated forward in time for \SI{10000}{\second}.

\TODO{explain how we calculate the analytic solution}

\begin{itemize}
	\item Compare cubicFit with linearUpwind
	\item Compare errors on BTF, cut cells and slanted cells using a small timestep
	\item Show maximum timesteps for various mesh spacings using Courant number close to one
\end{itemize}

\begin{figure}
	\centering
	\includegraphics{../fig-mountainAdvection-meshes/fig-mountainAdvection-meshes.pdf}
	\caption{\TODO{BTF, cut cell and slanted cell meshes used for the slug over a mountain test}}
\end{figure}

\begin{figure}
	\centering
	\includegraphics{../fig-mountainAdvection-error/fig-mountainAdvection-error.pdf}
	\caption{\TODO{evolution of the slug over a mountain at $t=0$, $t=T/2$ and $t=T$.  mountain advection error contours for (left-to-right) BTF, cut cells and slanted cells; linearUpwind (top) and cubicFit (bottom).  Tracer contours 0.1.  Error contours 0.01.} \\
	\TODO{I could overlay l2 and linf errors onto these plots.  Might be nicer than tabulating them separately.}}
\end{figure}

\begin{figure}
	\centering
	\includegraphics{../fig-mountainAdvection-maxdt/fig-mountainAdvection-maxdt.pdf}
	\caption{\TODO{mountain advection maximum timesteps for BTF, cut cells and slanted cells for various mesh spacings.  Demonstrates first that cubicFit has no problems near the limit of stability and, second, that slanted cells scale predictably with mesh spacing.}}
\end{figure}



\subsection{Deformational flow on a sphere}
To ensure that the cubicFit transport scheme is suitable for complex flows on a variety of meshes, we use a standard test of deformational flow on a spherical Earth \citep{lauritzen2012}.  
The standard test in \citep{lauritzen2012} comprised six elements
\begin{enumerate}
\item a convergence test using a Gaussian-shaped tracer
\item a ``minimal'' resolution test using a cosine-shaped tracer
\item a test of filament preservation
\item a test using a ``rough'' slotted cylinder tracer
\item a test of correlation preservation between two tracers
\item a test using a divergent velocity field
\end{enumerate}
We assess the cubicFit scheme using only tests 1, 2 and 6.  We do not consider filament preservation or the transport of a ``rough'' slotted cylinder because no shape-preserving filter has yet been developed for cubicFit.  \TODO{why is tracer correlation out of scope for this paper?}
Results are compared between linearUpwind and cubicFit schemes using icoshedral meshes and cubed-sphere meshes.

\TODO{how much detail do I need about OpenFOAM's global Cartesian coordinates, lack of 2D meshes and our correction for spherical geometry?}


\subsubsection{Numerical order of convergence using Gaussian hills}
% 

\begin{figure}
	\centering
	\includegraphics{../fig-deformationSphere-initialTracer/fig-deformationSphere-initialTracer.pdf}
	\caption{\TODO{evolution of deformational flow test cases for Gaussian hills with plots at $t=0$, $t=T/2$ and $t=T$.  The analytic solution at $t=T$ is identical to the initial condition.  Cosine bells initial condition also plotted.  This figure is supposed to give a sense of what `should' happen, so plot at a high resolution using whichever mesh gives better results.}}
\end{figure}

\begin{figure}
	\centering
	\includegraphics{../fig-deformationSphere-gaussiansConvergence/fig-deformationSphere-gaussiansConvergence.pdf}
	\caption{\TODO{deformational flow l2 and linf convergence plots comparing cubed sphere and hexagons, cubicFit and linearUpwind.  This figure is comparable to \citet{lauritzen2012} figure 4.}}
\end{figure}

\subsubsection{``Minimal'' resolution using cosine bells}

\begin{figure}
	\centering
	\includegraphics{../fig-deformationSphere-cosBellsConvergence/fig-deformationSphere-cosBellsConvergence.pdf}
	\caption{\TODO{$\ell_2$ convergence for non-divergent deformational flow using Cosine bells.  Used to find ``minimal'' resolution.  Plot for hexagons and cubed sphere, cubicFit and linearUpwind.  Plot a heavy line for minimal resolution, as in \citet{lauritzen2012} figure 5.}}
\end{figure}

\subsubsection{Transport under divergent flow conditions using cosine bells}

\begin{figure}
	\centering
	\includegraphics{../fig-deformationSphere-divergentTracer/fig-deformationSphere-divergentTracer.pdf}
	\caption{\TODO{divergent flow at $t=T/2$ and $t=T$ comparing cubed sphere and hexagons, cubicFit and linearUpwind.  Corresponds to \citet{lauritzen2012} figure 9.}}
\end{figure}



\section{Conclusions}

The advection scheme is
\begin{itemize}
	\item suitable for complex flows on a variety of meshes
	\item computationally cheap at runtime, with more expensive computations depending only on the mesh geometry
	\item \TODO{convergence}
	\item stable for Courant numbers up to 1
\end{itemize}

\section{Acknowledgements}
\TODO{Supervisors, funding bodies.  ASAM group for the mesh generator---I should ask permission to use cut cell meshes in this paper.  Dr Tristan Pryer.  Dr Shing Hing Man.}

\section*{Appendix A: One-dimensional von Neumann stability analysis}
Two analyses are performed in order to find stability constraints on the weights $\mathbf{w} = \mathbf{\tilde{b}_1^+} \cdot \mathbf{m}$ as appear in equation~\eqref{eqn:weightedPinv}.  The first analysis uses two points to derive separate constraints on the upwind weight $w_u$ and downwind weight $w_d$.  The second analysis uses three points to derive a constraint that considers all weights in a stencil.

\subsection*{Two-point analysis}
We start with the conservation equation for a dependent variable $\phi$ that is discrete-in-space and continuous-in-time
\begin{align}
\frac{\partial \phi_j}{\partial t} &= - u \frac{\phi_R - \phi_L}{\Delta x} \label{eqn:advectionLR} \\
%
\intertext{where the left and right fluxes, $\phi_L$ and $\phi_R$, are weighted averages of the neighbouring points.  Assuming that $u$ is positive}
%
\phi_L &= \alpha_u \phi_{j-1} + \alpha_d \phi_j \\
\phi_R &= \beta_u \phi_j + \beta_d \phi_{j+1}
\end{align}
where $\alpha_u$ and $\beta_u$ are the upwind weights and $\alpha_d$ and $\beta_d$ are the downwind weights for the left and right fluxes respectively, and $\alpha_u + \alpha_d = 1$ and $\beta_u + \beta_d = 1$.  A subscript $j$ denotes the value at a given point $x = j \Delta x$ where $\Delta x$ is a uniform mesh spacing.

At a given time $t = n \Delta t$ at time-level $n$ and with a time-step $\Delta t$, we assume a wave-like solution with an amplification factor $A$, such that
\begin{align}
	\phi_j^{(n)} &= A^n e^{\iu j k \Delta x} \label{eqn:vn}
\end{align}
where $\phi_j^{(n)}$ denotes a value of $\phi$ at position $j$ and time-level $n$.  Using this to rewrite the left-hand side of equation~\eqref{eqn:advectionLR}
\begin{align}
\frac{\partial \phi_j}{\partial t} &= \frac{\partial}{\partial t} \left( A^{t / \Delta t} \right) e^{ijk\Delta x} = \frac{\ln A}{\Delta t} A^n e^{ikj\Delta x} \\
%
\shortintertext{hence equation~\eqref{eqn:advectionLR} becomes}
%
\frac{\ln A}{\Delta t} &= - \frac{u}{\Delta x} \left( \beta_u + \beta_d e^{ik\Delta x} - \alpha_u e^{-ik\Delta x} - \alpha_d \right) \\
\ln A &= -c \left( \beta_u - \alpha_d + \beta_d \cos k\Delta x + \iu \beta_d \sin k \Delta x - \alpha_u \cos k\Delta x + \iu \alpha_u \sin k\Delta x \right)
%
\intertext{where the Courant number $c = u \Delta t / \Delta x$.
Let $\Re = \beta_u - \alpha_d + \beta_d \cos k\Delta x - \alpha_u \cos k\Delta x$ and
$\Im = \beta_d \sin k \Delta x + \alpha_u \sin k\Delta x$, then}
%
\ln A &= -c \left( \Re + \iu \Im \right) \\
A &= e^{-c \Re} e^{-\iu c \Im} \\
%
\shortintertext{and the complex modulus of $A$ is}
%
|A| &= e^{-c \Re} = \exp \left( -c \left( \beta_u - \alpha_d + \left(\beta_d - \alpha_u \right) \cos k\Delta x \right) \right) \text{ .}
\end{align}
For stability we need $|A| \leq 1$ and, imposing the additional constraints that $\alpha_u = \beta_u$ and $\alpha_d = \beta_d$, then
\begin{align}
\left( \alpha_u - \alpha_d \right) \left( 1 - \cos k\Delta x \right) &\geq 0 \quad \forall k\Delta x
%
\shortintertext{and, given $0 \leq 1 - \cos k \Delta x \leq 2$, then}
%
\alpha_u - \alpha_d \geq 0 \text{ .} \label{eqn:twopoint-lower}
\end{align}
Additionally, we do not want more damping than a first-order upwind scheme (where $\alpha_u = \beta_u = 1$, $\alpha_d = \beta_d = 0$), having an amplification factor, $A_\mathrm{up}$, so we need $|A| \geq |A_\mathrm{up}|$, hence
\begin{align}
	\exp \left( -c \left(\alpha_u - \alpha_d\right) \left( 1 - \cos k\Delta x \right) \right) &\geq \exp \left( -c \left(1 - \cos k\Delta x \right) \right) \quad \forall k\Delta x
%
\shortintertext{therefore}
%
	\alpha_u - \alpha_d &\leq 1 \text{ .} \label{eqn:twopoint-upper}
\end{align}
Now, knowing that $\alpha_u + \alpha_d = 1$ (or $\alpha_d = 1 - \alpha_u$) then, using equations~\eqref{eqn:twopoint-lower} and \eqref{eqn:twopoint-upper},
\begin{align}
	0.5 \leq \alpha_u &\leq 1 \text{,} \label{eqn:vn:upwind} \\
	0 \leq \alpha_d &\leq 0.5 \label{eqn:vn:downwind} \text{ .}
\end{align}

\subsection*{Three-point analysis}
We start again from equation~\eqref{eqn:advectionLR} but this time approximate $\phi_L$ and $\phi_R$ using three points,
\begin{align}
	\phi_L &= \alpha_{uu} \phi_{j-2} + \alpha_u \phi_{j-1} + \alpha_d \phi_j \\
	\phi_R &= \alpha_{uu} \phi_{j-1} + \alpha_u \phi_j + \alpha_d \phi_{j+1}
\end{align}
having used the same weights $\alpha_{uu}$, $\alpha_u$ and $\alpha_d$ for both left and right fluxes.
Substituting equation~\eqref{eqn:vn} into equation~\eqref{eqn:advectionLR} we find
\begin{align}
A = \exp\left( -c \left[ \alpha_{uu} \left( e^{-ik\Delta x} - e^{-2ik\Delta x} \right) + \alpha_u \left( 1 - e^{-ik\Delta x} \right) + \alpha_d \left( e^{ik\Delta x} - 1 \right) \right] \right)
%
\intertext{so that, if the complex modulus $|A| \leq 1$ then}
%
\alpha_u - \alpha_d + \left( \alpha_{uu} - \alpha_u + \alpha_d \right) \cos k\Delta x - \alpha_{uu} \cos 2k\Delta x \geq 0 \text{ .}
\end{align}
If $k\Delta x = \pi$ then $\cos k\Delta x = -1$ and $\cos 2k\Delta x = 1$ and $\alpha_u - \alpha_d \geq \alpha_{uu}$.  If $k\Delta x = \pi / 2$ then $\cos k\Delta x = 0$ and $\cos 2k\Delta x = -1$ and $\alpha_u - \alpha_d \geq -\alpha_{uu}$.  Hence we find that
\begin{align}
	\alpha_u - \alpha_d &\geq |\alpha_{uu}| \label{eqn:uuConstraint} \text{ .}
%
\intertext{When the same analysis is performed with four points, $\alpha_{uuu}$, $\alpha_{uu}$, $\alpha_u$ and $\alpha_d$, by varying $k \Delta x$ we find that equation~\eqref{eqn:uuConstraint} still holds.
We also find that the same condition holds replacing $\alpha_{uu}$ with $\alpha_{uuu}$.  Hence, we generalise equation~\eqref{eqn:uuConstraint} to find the final stability constraint}
%
	\alpha_u - \alpha_d &\geq \max_{p\:\in\:P} |\alpha_p|
\end{align}
where the peripheral cells $P$ is the set of all stencil cells except for the upwind cell and downwind cell, and $\alpha_p$ is the weight for a given peripheral cell $p$.
We hypothesise that the three stability constraints (equations~\ref{eqn:vn:upwind}, \ref{eqn:vn:downwind} and \ref{eqn:uuConstraint}) are necessary but not sufficient for a transport scheme on arbitrarily-structured meshes.


\section*{Appendix B: Mesh geometry on a spherical Earth}

The cubicFit transport scheme is implemented using the OpenFOAM CFD library.  Unlike many atmospheric models that use spherical coordinates, OpenFOAM uses global, three-dimensional Cartesian coordinates.  In order to perform the experiments on a spherical Earth presented in section~\ref{sec:deformationSphere}, it is necessary for velocity fields and mesh geometries to be expressed in these global Cartesian coordinates.

\subsection*{Velocity field specification}
The non-divergent velocity field in section~\ref{sec:deformationSphere} is specified as a streamfunction $\Psi(\lambda, \theta)$.  Instead of calculating velocity vectors, the flux $\mathbf{u}_f \cdot \mathbf{S}_f$ through a face $f$ is calculated directly from the streamfunction,
\begin{align}
	\mathbf{u}_f \cdot \mathbf{S}_f	= \sum_{e\:\in\:f} \mathbf{e} \cdot \mathbf{x}_e \Psi(e) \label{eqn:nondiv-spherical-flux}
\end{align}
where $e \in f$ denotes the edges $e$ of face $f$, $\mathbf{e}$ is the edge vector joining its two vertices, $\mathbf{x}_e$ is the position vector of the edge midpoint, and $\Psi(e)$ is the streamfunction evaluated at the same position.
Edge vectors are directed in a counter-clockwise orientation.

\subsection*{Spherical mesh construction}

Since OpenFOAM does not support two-dimensional spherical meshes, instead, we construct meshes that have a single layer of cells that are \SI{2000}{\meter} deep, having an inner radius $r_1 = R_e - \SI{1000}{\meter}$ and an outer radius $r_2 = R_e + \SI{1000}{\meter}$.
By default, OpenFOAM meshes comprise polyhedral cells with straight edges and flat faces.  This is problematic for spherical meshes because face areas and cell volumes are too small.
For tests on a spherical Earth, we override the default configuration and calculate our own face areas, cell volumes, face centres and cell centres that account for the spherical geometry.  

A face is assumed to be either a surface face or radial face.
A surface face has any number of vertices, all of equal radius.
A radial faces have four vertices with two different radii, $r_1$ and $r_2$, and two different horizontal coordinates, $(\lambda_1, \theta_1)$ and $(\lambda_2, \theta_2)$.
A radial face centre is modified so that it has a radius $R_e$.  The latitudinal and longitudinal components of a radial face centre need no modification. \TODO{although the code does recalculate this anyway!}
The face area $A_f$ for a radial face $f$ is the area of the annular sector,
\begin{align}
	A_f = \frac{d}{2} \left\lvert r_2^2 - r_1^2 \right\rvert
\end{align}
where $d$ is the great-circle distance between $(\lambda_1, \theta_1)$ and $(\lambda_2, \theta_2)$.

To calculate the centre of a surface face $f$, a new vertex is created that is positioned at the mean of the face vertices.  Note that this centre position, $\mathbf{\tilde{c}}_f$, is used in intermediate calculations and it is not the face centre position.
Next, the surface face is subdivided into spherical triangles that share this new vertex.
The face centre direction and radius are calculated separately.  The face centre direction $\mathbf{\hat{r}}$ is the mean of the spherical triangle centres weighted by their solid angle
\begin{align}
	\mathbf{\hat{r}} = \frac
	{\sum_{t\:\in\:f}{\Omega_t \left(\mathbf{x}_{t,1} + \mathbf{x}_{t,2} + \mathbf{\tilde{c}}_f \right)}}
	{\left\lvert \sum_{t\:\in\:f}{\Omega_t \left(\mathbf{x}_{t,1} + \mathbf{x}_{t,2} + \mathbf{\tilde{c}}_f \right)} \right\rvert} \label{eqn:face-centre-dir}
\end{align}
where $t\in f$ denotes the spherical triangles $t$ of face $f$, $\Omega_t$ is spherical triangle's solid angle which is calculated using l'Huilier's theorem, $\mathbf{x}_{t,1}$ and $\mathbf{x}_{t,2}$ are the positions of the vertices shared by the face $f$ and spherical triangle $t$, and $\mathbf{\tilde{c}}_f$ is the position of the centre vertex shared by all spherical triangles of face $f$.
The face centre radius $r$ is the mean radius of the face vertices, again weighted by the solid angle of each spherical triangle,
\begin{align}
	r = \frac
	{\sum_{t\in f}{\Omega_t \left(\left\lvert \mathbf{x}_{t,1} \right\rvert + \left\lvert \mathbf{x}_{t,2} \right\rvert \right)/2}}
	{\Omega_f} \label{eqn:face-centre-mag}
\end{align}
where the solid angle $\Omega_f$ of face $f$ is the sum of the solid angles of the constituent spherical triangles,
\begin{align}
	\Omega_f = \sum_{t\in f}{\Omega_t}
\end{align}
We use equations~\eqref{eqn:face-centre-dir} and \eqref{eqn:face-centre-mag} to calculate the centre $\mathbf{c}_f$ of the face,
\begin{align}
	\mathbf{c}_f = r\:\mathbf{\hat{r}}
\end{align}
The area vector $\mathbf{S}_f$ of the surface face $f$ is the sum of the spherical triangle areas \citep{vanbrummelen2013},
\begin{align}
	\mathbf{S}_f = r^2 \Omega_f \mathbf{\hat{r}}
\end{align}
Cell centres and cell volumes are corrected by considering faces that are not normal to the sphere such that
\begin{align}
	\frac{\left(\mathbf{S}_f \cdot \mathbf{c}_f\right)^2}{\left\lvert \mathbf{S}_f \right\rvert^2 \left\lvert \mathbf{c}_f \right\rvert^2} > 0 \label{eqn:surface-faces}
\end{align}
\TODO{in this case we use a more general method for identifying surface faces than we used for face correction.  we should use the general method in all cases and update this description accordingly.  raised \href{https://trello.com/c/anBhIy60/812-change-cell-centre-spherical-correction-to-be-consistent-with-edge-and-face-centres} for this.}
Let $\mathcal{F}$ be the set of faces satisfying equation~\eqref{eqn:surface-faces}.  Then, the cell volume $\mathcal{V}_c$ is
\begin{align}
	\mathcal{V}_c = \frac{1}{3} \sum_{f\:\in\:\mathcal{F}} \mathbf{S}_f \cdot \mathbf{c}_f
\end{align}
which can be thought of as the area $A$ integrated between $r_1$ and $r_2$ such that 
$\int_0^R{A(r)\:\mathrm{d}r} = \int_{r_1}^{r_2}{r^2 \Omega\:\mathrm{d}r} = \frac{1}{3} \Omega \left( r_2^3 - r_1^3 \right)$.
\TODO{this will be amended in due course --- The cell centre position $\mathbf{c}_c$ is
\begin{align}
	\mathbf{c}_c = \frac{\sqrt{\frac{1}{3} \left(r_1^2 + r_1 r_2 + r_2^2\right)}\sum_{f\in\mathcal{F}} \mathbf{c}_f}{\left\lvert \sum_{f\in\mathcal{F}} \mathbf{c}_f \right\rvert}
\end{align}}

Edges can be classified in a similar manner to faces where surface edges are tangent to the sphere and radial faces are normal to the sphere.  The edge midpoints, $\mathbf{c}_e$, are used to calculate the face flux for non-divergent velocity fields (equation~\ref{eqn:nondiv-spherical-flux}).
For transport tests, corrections to edge midpoints are unnecessary.  Due to the choice of $r_1$ and $r_2$ during mesh construction, the midpoint of a radial edge is at a radial distance of $R_e$ which is necessary for the correct calculation of non-divergent velocity fields.
The position of surface edge midpoints is unimportant because these edges do not contribute to the face flux since $\mathbf{e} \cdot \mathbf{c}_e = 0$.
Edge lengths are the straight-line distance between the two vertices and not the great-circle distance.  Again, the edge lengths are not corrected because it makes no difference to the face flux calculation.



\bibliographystyle{elsarticle-num}
\bibliography{references}

\end{document}
