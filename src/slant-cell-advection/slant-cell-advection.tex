\documentclass{article}
\usepackage{graphicx}

\newcommand{\TODO}[1]{\textcolor{purple}{TODO: \emph{#1}}}
\begin{document}
\begin{figure}
	\centering
	\includegraphics{../fig-grid-generation/fig-grid-generation.pdf}
\end{figure}
test
\end{document}
