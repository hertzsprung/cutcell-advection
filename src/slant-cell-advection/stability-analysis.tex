\documentclass{article}
\usepackage{fullpage}
\usepackage{amsmath}
\usepackage{tikz}
\usepackage{bm}
\usepackage[hidelinks]{hyperref}
\usepackage{siunitx}

\newcommand{\Co}{C}
\newcommand{\vect}{\bm}
\newcommand{\iu}{{i\mkern1mu}}
\newcommand{\TODO}[1]{\textcolor{purple}{TODO: \emph{#1}}}

\title{Von Neumann stability analysis for \texttt{cubicUpwindCPCFit} advection scheme}
\author{James Shaw, Hilary Weller}

\begin{document}
\maketitle

The stabilisation procedure for the \texttt{cubicUpwindCPCFit} advection scheme comprises several steps.  One of the steps selects a suitable subset of polynomial terms using a constraint
\begin{align}
	-c_p + 0.01 \leq \min(c_u, c_d)\:\forall\:p
\end{align}
where $c_i$ is the coefficient associated with cell $i$, $c_u$ is the upwind coeficient, $c_d$ is the downwind coefficient, and $p$ is the set of peripheral points.  Collectively, the upwind and downwind points are called the central points, and the upwind and downwind coefficients are called the central coefficients.  Peripheral points are those points that are not central points.
This constraint ensures that peripheral cells do not have very negative values.  The magic constant \num{0.01} was found to improve accuracy for a few particular stencils near the lower boundary of BTF meshes.

This constraint is present in the current codebase\footnote{\url{https://github.com/hertzsprung/AtmosFOAM/commit/ffaf71}} and supercedes a previous, stronger constraint\footnote{\url{https://github.com/hertzsprung/AtmosFOAM/commit/208899}}:
\begin{align}
	|c_p| \leq \min(c_u, c_d)
\end{align}
This constraint ensures that peripheral coefficients are smaller than any central coefficient.

This document presents two analyses that begin to justify these constraints.  Section~\ref{sec:matrix} analyses the matrix inverse for a one-dimensional quadratic equation.  Section~\ref{sec:neumann} performs a von Neumann analysis of a one-dimensional linear advection scheme with three points.  Finally, we note the similarity between the two results.

\section{Matrix inversion of a quadratic equation}
\label{sec:matrix}
An interpolating quadratic equation, $\phi = a_1 + a_2 x + a_3 x^2$, is found for three data, $\phi_{uu}, \phi_{u}, \phi_{d}$, that are known at points $x_{uu}, x_{u}, x_{d}$.  This forms a matrix equation
\begin{align}
\underbrace{
\begin{bmatrix}
1 & x_{uu} & x_{uu}^2 \\
1 & x_u & x_u^2 \\
1 & x_d & x_d^2
\end{bmatrix}}_{\vect{B}}
\underbrace{
\begin{bmatrix}
a_1 \\
a_2 \\
a_3
\end{bmatrix}}_{\vect{a}}
= 
\underbrace{
\begin{bmatrix}
\phi_{uu} \\
\phi_{u} \\
\phi_{d}
\end{bmatrix}
}_{\vect{\phi}}
\end{align}
which can be inverted to solve for $\vect{a}$, that is $\vect{a} = \vect{B^{-1}} \vect{\phi}$.
In fact, we only need to calculate $a_1$:
\begin{align}
a_1 = 
\begin{bmatrix}
\underbrace{\dfrac{x_u x_d^2 - x_u^2 x_d}{\det \vect{B}}}_{c_{uu}} &
\underbrace{\dfrac{x_{uu}^2 x_d - x_{uu} x_d^2}{\det \vect{B}}}_{c_u} &
\underbrace{\dfrac{x_{uu} x_u^2 - x_{uu}^2 x_u}{\det \vect{B}}}_{c_d}
\end{bmatrix}
\begin{bmatrix}
\phi_{uu} \\
\phi_{u} \\
\phi_{d}
\end{bmatrix}
\end{align}
We choose $x_{uu}, x_u, x_d$ such that $x_{uu} < x_u < 0$ and $x_d > 0$ and we wish to find the constraint involving $x_{uu}$, $x_u$ and $x_d$ such that $|c_{uu}| \leq |c_u|$ and $c_{uu} \leq |c_d|$.
To do so, we choose four different values for $x_{uu}$ between \num{-2} and \num{-1}, and allow $-1 < x_u < 0$ and $0 < x_d < 1$.  Figure~\ref{fig:matrix} highlights in orange those regions in which the constraint is satisfied.  By inspection, we find that the constraint is
\begin{align}
	x_u - x_d \geq x_{uu} \label{eqn:matrix-constraint}
\end{align}

\begin{figure}
	\centering
	\input{../fig-matrix-constraint/matrix-constraint}
	%
	\caption{Regions in which $c_{uu} < c_u$ and $c_{uu} < c_d$}
	\label{fig:matrix}
\end{figure}

\section{Von Neumann analysis of a three-point, 1D linear advection scheme}
\label{sec:neumann}

Start with the flux form equation, discretised in space, continuous in time:
\begin{align}
\frac{\partial \phi_j}{\partial t} &= - u \frac{\phi_R - \phi_L}{\Delta x} \label{eqn:advection} \\
	\phi_L &= \alpha_{uu} \phi_{j-2} + \alpha_u \phi_{j-1} + \alpha_d \phi_j \\
	\phi_R &= \alpha_{uu} \phi_{j-1} + \alpha_u \phi_j + \alpha_d \phi_{j+1}
\end{align}
We perform a Von Neumann stability analysis with perfect time discretisation
\begin{align}
\phi_j^n &= A^n e^{ijk\Delta x} \label{eqn:vn} \\
t &= n \Delta t \\
\frac{\partial \phi_j}{\partial t} &= \frac{\partial}{\partial t} \left( A^{t / \Delta t} \right) e^{ijk\Delta x} \\
&= \frac{\ln A}{\Delta t} A^n e^{ikj\Delta x} \\
\frac{\ln A}{\Delta t} &= -\frac{u }{\Delta x} \left( \alpha_{uu} \left( e^{-ik\Delta x} - e^{-2ik\Delta x} \right) + \alpha_u \left( 1 - e^{-ik\Delta x} \right) + \alpha_d \left( e^{ik\Delta x} - 1 \right) \right) \\
A &= \exp\left( -\Co \left[ \alpha_{uu} \left( e^{-ik\Delta x} - e^{-2ik\Delta x} \right) + \alpha_u \left( 1 - e^{-ik\Delta x} \right) + \alpha_d \left( e^{ik\Delta x} - 1 \right) \right] \right)
%
\intertext{where $\Co = u \Delta t / \Delta x$ is the Courant number.  Stability requires $|A| \leq 1$.  Let $z = \alpha_{uu} \left( e^{-ik\Delta x} - e^{-2ik\Delta x} \right) + \alpha_u \left( 1 - e^{-ik\Delta x} \right) + \alpha_d \left( e^{ik\Delta x} - 1 \right)$ and $\Re$ and $\Im$ are the real and imaginary parts of $z$ respectively, then}
%
| e^{-\Co z}| &\leq 1 \\
| e^{-\Co \left( \Re + i \Im \right)}| &\leq 1 \\
| e^{-\Co \Re} e^{-i \Im} | &\leq 1 \\
| e^{-\Co \Re} | &\leq 1 
\end{align}
So $\Re \geq 0$, then
\begin{align}
\alpha_{uu} \left( \cos k\Delta x - \cos 2k\Delta x \right) + \alpha_u \left( 1 - \cos k \Delta x \right) + \alpha_d \left( \cos k\Delta x - 1 \right) &\geq 0 \\
\alpha_u - \alpha_d + \left( \alpha_{uu} - \alpha_u + \alpha_d \right) \cos k\Delta x - \alpha_{uu} \cos 2k\Delta x &\geq 0
\end{align}
Now we want to find inequalities that constrain $\alpha_u$, $\alpha_{uu}$ and $\alpha_d$.
If $\cos k\Delta x = -1$ and $\cos 2k\Delta x = 1$ then $\alpha_u \geq \alpha_{uu} + \alpha_d$.
So we have the constraint
\begin{align}
\alpha_u - \alpha_d \geq \alpha_{uu}
\end{align}
This constraint has the same form as equation~\eqref{eqn:matrix-constraint}.

\section*{Von Neumann analysis of a two-point, 1D linear advection scheme}

Start with the flux form equation, discretised in space, continuous in time:
\begin{align}
\frac{\partial \phi_j}{\partial t} &= - u \frac{\phi_R - \phi_L}{\Delta x} \label{eqn:advection} \\
\phi_L &= \alpha_u \phi_{j-1} + \alpha_d \phi_j \\
\phi_R &= \beta_u \phi_j + \beta_d \phi_{j+1}
\end{align}

Von Neumann stability analysis with perfect time discretisation
\begin{align}
\phi_j^n &= A^n e^{ijk\Delta x} \label{eqn:vn} \\
t &= n \Delta t \\
\frac{\partial \phi_j}{\partial t} &= \frac{\partial}{\partial t} \left( A^{t / \Delta t} \right) e^{ijk\Delta x} \\
&= \frac{\ln A}{\Delta t} A^n e^{ikj\Delta x} \\
\frac{\ln A}{\Delta t} &= - \frac{u}{\Delta x} \left( \beta_u + \beta_d e^{ik\Delta x} - \alpha_u e^{-ik\Delta x} - \alpha_d \right) \\
\ln A &= - c \left( \beta_u - \alpha_d + \beta_d e^{ik\Delta x} - \alpha_u e^{-ik\Delta x} \right) \\
      &= -c \left( \beta_u - \alpha_d + \beta_d \cos k\Delta x + \iu \beta_d \sin k \Delta x - \alpha_u \cos k\Delta x + \iu \alpha_u \sin k\Delta x \right)
%
\intertext{let $\Re = \beta_u - \alpha_d + \beta_d \cos k\Delta x - \alpha_u \cos k\Delta x$ and
%
$\Im = \beta_d \sin k \Delta x + \alpha_u \sin k\Delta x$, then}
\ln A &= -c \left( \Re + \iu \Im \right) \\
A &= e^{-c \Re} e^{-\iu c \Im} \\
|A| &= e^{-c \Re} = \exp \left( -c \left( \beta_u - \alpha_d + \left(\beta_d - \alpha_u \right) \cos k\Delta x \right) \right) \\
\arg(A) &= -c \Im = -c \left( \beta_d + \alpha_u \right) \sin k\Delta x
%
\intertext{For stability we need $|A| \leq 1$ and $\arg(A) < 0$ for $c > 0$, so}
%
\beta_u - \alpha_d + \left( \beta_d - \alpha_u \right) \cos k\Delta x &\geq 0 \quad \forall k\Delta x \quad \text{and} \\
\beta_d + \alpha_u &> 0 \label{eqn:arg-ineq}
\end{align}
\TODO{not sure what to make of these inequalities, Hilary got a little further but I didn't follow all of it.  But no matter, let's continue\ldots}

Imposing the additional constraints that $\alpha_u = \beta_u$ and $\alpha_d = \beta_d$:
\begin{align}
|A| &= \exp \left( -c \left( \alpha_u - \alpha_d \right) \left(1 - \cos k\Delta x \right) \right)
%
\intertext{and given $1 - \cos k\Delta x \geq 0$ for well-resolved waves}
%
\alpha_u - \alpha_d &\geq 0 \\
\alpha_u &\geq \alpha_d
%
\intertext{and from eqn \eqref{eqn:arg-ineq}}
%
\alpha_d + \alpha_u &> 0 \\
\alpha_u &> - \alpha_d \quad \text{hence} \\
\alpha_u &> |\alpha_d| \label{eqn:lower-bound}
\end{align}

Additionally, we do not want more damping than an upwind scheme (where $\alpha_u = \beta_u = 1$, $\alpha_d = \beta_d = 0$), having an amplification factor, $A_\mathrm{up}$:
\begin{align}
|A_\mathrm{up}| &= \exp \left( -c \left(1 - \cos k\Delta x \right) \right)
%
\intertext{So we need $|A| \geq |A_\mathrm{up}|$:}
%
-c \left( \alpha_u - \alpha_d \right) \left(1 - \cos k\Delta x \right) &\geq -c \left( 1- \cos k\Delta x \right) \\
\alpha_u - \alpha_d &\leq 1 \\
\alpha_u &\leq 1 + \alpha_d
%
\intertext{which provides an upper bound on $\alpha_u$.  Combining with eqn~\eqref{eqn:lower-bound} we can bound $\alpha_u$ on both sides:}
%
|\alpha_d| < \alpha_u \leq 1 + \alpha_d
\end{align}
Now, assume that $\alpha_u + \alpha_d = 1$ (or $\alpha_d = 1 - \alpha_u$), then
\begin{align}
	1 - \alpha_u < \alpha_u &\leq 1 + (1 - \alpha_u) \\
	0.5 < \alpha_u &\leq 1
%
\intertext{we use only the lower bound \TODO{why?} to obtain}
%
|\alpha_d| < \alpha_u \leq 1 + \alpha_d\quad\text{and}\quad0.5 < \alpha_u
\end{align}

\end{document}
