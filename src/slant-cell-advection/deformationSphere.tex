\subsection{Deformational flow on a sphere}
\label{sec:deformationSphere}
The tests so far have used flows that are mostly uniform on meshes that are based on rectangular cells.
To ensure that the cubicFit transport scheme is suitable for complex flows on a variety of meshes, we use a standard test of deformational flow on a spherical Earth, as specified by Lauritzen et al. \citep{lauritzen2012}.  
Results are compared between linearUpwind and cubicFit schemes using hexagonal icosahedra and cubed-spheres.
Hexagonal-icosahedral meshes are constructed by successive refinement of a regular icosahedron following the approach by \citep{thuburn2014}.
Figure~\ref{fig:sphere-meshes}a shows an example of such a mesh that has been refined three times.
Cubed-sphere meshes are constructed using an equi-distant gnomic projection of a cube having a uniform Cartesian mesh on each panel \citep{staniforth-thuburn2012}.
Figure~\ref{fig:sphere-meshes}b shows an example of a cubed-sphere mesh having panels with $8 \times 8$ cells.

\begin{figure}
	\centering
	\includegraphics{../fig-deformationSphere-meshes/fig-deformationSphere-meshes.pdf}
	\caption{Illustrations of coarse meshes of a spherical Earth using (a) a hexagonal icosahedron that has been refined three times, and (b) a cubed-sphere with each panel having $8 \times 8$ cells.}
	\label{fig:sphere-meshes}
\end{figure}

Following appendix A9 in \citep{lauritzen2014}, the average equatorial spacing $\Delta \lambda$ is used as a measure of mesh spacing.  It is defined as
\begin{align}
	\Delta \lambda = \ang{360} \frac{\overline{\Delta x}}{2 \pi R_e}
\end{align}
where $\overline{\Delta x}$ is the mean distance between cell centres and $R_e = \SI{6.3712e6}{\meter}$ is the radius of the Earth.

The deformational flow test in \citep{lauritzen2012} comprised six elements:
\begin{enumerate}
\item a convergence test using a Gaussian-shaped tracer
\item a ``minimal'' resolution test using a cosine-shaped tracer
\item a test of filament preservation
\item a test using a ``rough'' slotted cylinder tracer
\item a test of correlation preservation between two tracers
\item a test using a divergent velocity field
\end{enumerate}
We assess the cubicFit scheme using the first two tests only.  We do not consider filament preservation, correlation preservation, or the transport of a ``rough'' slotted cylinder because no shape-preserving filter has yet been developed for cubicFit.  Stable results were obtained when testing cubicFit using a divergent velocity field, but no further analysis is made here.  \TODO{provide results as supplementary material?}

The first deformational flow test uses a $C^\infty$ initial tracer that is transported in a non-divergent, time-varying rotational velocity field.
The velocity field deforms two Gaussian `hills' of tracer into thin vortical filaments.  Half-way through the integration the rotation reverses so that the filaments become circular hills once again.  The analytic solution at the end of integration is identical to the initial condition.
A rotational flow is superimposed on a time-invariant background flow in order to avoid error cancellation.
The non-divergent velocity field is defined by the streamfunction $\Psi$,
\begin{align}
	\Psi(\lambda, \theta, t) = \frac{10 R_e}{T} \sin^2 \left(\lambda'\right) \cos^2 \left(\theta\right) \cos \left( \frac{\pi t}{T} \right) - \frac{2 \pi R_e}{T} \sin\left(\theta\right)
\end{align}
where $\lambda$ is a longitude, $\theta$ is a latitude, $\lambda' = \lambda - 2 \pi t / T$, and $T = \SI{1.0368e6}{\second}$ is the duration of integration.  The time-step is chosen such that the maximum Courant number is about 0.4.

The initial tracer density $\phi$ is defined as the sum of two Gaussian hills,
\begin{align}
	\phi = \phi_1(\lambda, \theta) + \phi_2(\lambda, \theta)
\end{align}
An individual hill $\phi_i$ is given by
\begin{align}
	\phi_i(\lambda, \theta) = \phi_0 \exp\left( -b \left( \frac{|\mathbf{x} - \mathbf{x}_i|}{R_e} \right)^2 \right)
\end{align}
where $\phi_0 = \SI{0.95}{\kilo\gram\per\meter\cubed}$ and $b = 5$.  The Cartesian position vector $\mathbf{x} = (x,y,z)$ is related to the spherical coordinates $(\lambda, \theta)$ by
\begin{align}
	(x,y,z) = (R_e \cos \theta \cos \lambda, R_e \cos \theta \sin \lambda, R_e \sin \theta) \label{eqn:spherical-cartesian}
\end{align}
The centre of hill $i$ is positioned at $\mathbf{x}_i$.  In spherical coordinates, two hills are centred at
\begin{align}
	(\lambda_1,\theta_1) &= (5 \pi /6, 0) \\
	(\lambda_2,\theta_2) &= (7 \pi /6, 0)
\end{align}

\begin{figure}
	\centering
	\includegraphics{../fig-deformationSphere-initialTracer/fig-deformationSphere-initialTracer.pdf}
	\caption{Tracer fields for the deformational flow test using initial Gaussian hills.  The tracer is deformed by the velocity field before the rotation reverses to return the tracer to its original distribution: (a) the initial tracer distribution at $t = \SI{0}{\second}$; (b) by $t=T/2$ the Gaussian hills are stretched into a thin S-shaped filament; (c) at $t=T$ the tracer resembles the initial Gaussian hills except for some distortion and diffusion due to numerical errors.}
	\label{fig:deformationSphere-evolution}
\end{figure}

The results in figure~\ref{fig:deformationSphere-evolution} are obtained using the cubicFit scheme on a hexagonal-icosahedral mesh with $\Delta \lambda = \ang{0.271}$.  The initial Gaussian hills are shown in figure~\ref{fig:deformationSphere-evolution}a.  At $t=T/2$ the tracer has been deformed into an S-shaped filament (figure~\ref{fig:deformationSphere-evolution}b).  By $t=T$ the tracer has almost returned to its original distribution except for some slight distortion and diffusion that are the result of numerical errors (figure~\ref{fig:deformationSphere-evolution}c).

\begin{figure}
	\centering
	\includegraphics{../fig-deformationSphere-gaussiansConvergence/fig-deformationSphere-gaussiansConvergence.pdf}
%
	\caption{Numerical convergence of the deformational flow test on the sphere using initial Gaussian hills.  $\ell_2$ errors (equation~\ref{eqn:l2-error}) and $\ell_\infty$ errors (equation~\ref{eqn:linf-error}) are marked at mesh spacings between \ang{8.61} and \ang{0.271} using linearUpwind and cubicFit transport schemes on hexagonal icosahedra and cubed-sphere meshes.}
	\label{fig:deformationSphere-gaussian-convergence}
\end{figure}

To determine the order of convergence and relative accuracy of the linearUpwind and cubicFit schemes, the same test was performed at a variety of mesh spacings betweeen $\Delta \lambda = \ang{8.61}$ and $\Delta \lambda = \ang{0.271}$ on hexagonal icosahedra and cubed-sphere meshes.  The results are shown in figure~\ref{fig:deformationSphere-gaussian-convergence}.
The solution is slow to converge at coarse resolutions, but both linearUpwind and cubicFit schemes achieve second-order accuracy at smaller mesh spacings.  This behaviour agrees with the results from Lauritzen et al. \citep{lauritzen2012}.  For any given mesh type and mesh spacing, cubicFit is more accurate than linearUpwind.  The linearUpwind scheme achieves more accurate results on hexagonal icosahedra compared to cubed-sphere meshes, but cubicFit is less sensitive to the mesh type.

A slightly more challenging variation of the same test is performed using a quasi-smooth tracer field is defined as the sum of two cosine bells,
\begin{align}
	\phi =
	\begin{cases}
		b + c \phi_1(\lambda, \theta) & \quad \text{if $r_1 < r$} \\
		b + c \phi_2(\lambda, \theta) & \quad \text{if $r_2 < r$} \\
		b			      & \quad \text{otherwise}
	\end{cases}
\end{align}
The velocity field is the same as before.  This test is used to determine the ``minimal'' resolution, which is specified by Lauritzen et al. \citep{lauritzen2012} as the coarsest mesh spacing for which $\ell_2 \approx 0.033$.
\TODO{minimal resolution for cubicFit on hex mesh is about $\Delta \lambda = \ang{0.3}$}
\TODO{and linearUpwind? and perhaps compare to similar schemes.  mention that \citep{lauritzen2014} has collated minimal resolutions for a number of other schemes.}

