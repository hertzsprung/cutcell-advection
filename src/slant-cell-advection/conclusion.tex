\section{Conclusion}
\label{sec:conclusion}

We have presented a new multidimensional method-of-lines transport scheme, cubicFit, that is suitable for complex flows on a wide variety of mesh types.  Constraints derived from a von Neumann stability analysis are applied during model initialisation to make the scheme stable over steep terrain on arbitrarily-structured meshes.
The scheme has a low computational cost at runtime, requiring only $n$ multiplies per face per time-step using a stencil with $n$ cells, with more expensive computations depending on the mesh geometry only.

The cubicFit scheme was compared to a multidimensional linear upwind scheme using three idealised numerical tests.
The first test transported a tracer horizontally above steep slopes on highly-distorted, two-dimensional terrain-following meshes.  Unlike the linearUpwind scheme, cubicFit was second-order convergent regardless of mesh distortions.
The second test transported a tracer over a mountainous lower boundary using terrain-following, cut cell and slanted cell meshes.
The cubicFit scheme was generally insensitive to the type of mesh and less sensitive to terrain steepness compared to linearUpwind.  The cubicFit scheme maintained accuracy up to the stability limit of a Courant number of one.
The third test evaluated the transport schemes in a standard deformational flow field on hexagonal icosahedra and cubed-sphere meshes.
In all tests, compared to the linearUpwind scheme, the cubicFit scheme was more stable and more accurate.

