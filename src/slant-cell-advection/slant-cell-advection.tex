\documentclass[times]{elsarticle}
\usepackage{fullpage}
\usepackage{graphicx}
\usepackage{amsmath}
\usepackage{mathtools}
\usepackage{xcolor}
\usepackage{bm}
\usepackage{natbib}
\usepackage[hidelinks]{hyperref}
\usepackage{doi}
\usepackage[final,babel]{microtype}
\usepackage[utf8]{inputenc}
\usepackage[british]{babel}
\usepackage{csquotes}
\usepackage[T1]{fontenc}
\usepackage{siunitx}
\usepackage[font={small}]{caption}

\newcommand{\iu}{{i\mkern1mu}}
\newcommand{\iunit}{\boldsymbol{\hat \imath}}
\newcommand{\junit}{\boldsymbol{\hat \jmath}}
\newcommand{\kunit}{\boldsymbol{\hat k}}
\newcommand{\TODO}[1]{\textcolor{purple}{TODO: \emph{#1}}}

\begin{document}

\begin{frontmatter}
\title{A finite volume transport scheme for atmospheric flows over steep terrain \\ \TODO{(working title)}}
\author[uor]{James Shaw\corref{cor}}
\ead{js102@zepler.net}
\author[uor]{Hilary Weller}

\cortext[cor]{Corresponding author}
\address[uor]{Department of Meteorology, University of Reading, Reading, United Kingdom}

\begin{abstract}
	\TODO{abstract}
\end{abstract}

\begin{keyword}
	\TODO{keywords}
\end{keyword}
\end{frontmatter}

\section{Introduction}

First, we present a multidimensional advection scheme that is computationally cheap and suitable for complex flows on a variety of meshes.  Second, we present a new type of Cartesian mesh, the slanted cell mesh, that avoids the small cell problem associated with cut cell meshes.   We apply the advection scheme to tests over steep orography and show that accurate results are obtained on slanted cell meshes.  Finally, we challenge the multidimensional advection scheme using a test of deformational flow on a geodesic mesh.

slantedCell motivation
\begin{itemize}
	\item operational models use terrain-following coordinates that can degrade the accuracy of pressure gradient calculations and horizontal transport
	\item the cut cell method can improve pressure gradient accuracy and more accurately represent horizontal transport, but the cut cell method suffers from the small cell problem
	\item the slanted cell method is an alternative presented in \citep{shaw-weller2016} that avoids the small cell problem whilst retaining accurate pressure gradient calculations and horizontal transport
\end{itemize}

cubicFit motivation
\begin{itemize}
	\item finite volume is desirable because it's inherently conservative, \TODO{what else?  FV3 do it, ECMWF are doing it}
	\item a transport scheme for arbitrary meshes is desirable because different meshes offer different trade-offs (\TODO{e.g. provide example for icosahedra vs cubed-sphere and citation}).  supporting arbitrary meshes lets us defer the choice of mesh.
	\item predicting meteorological variables at or near the surface is important because it's where we live, clouds can form near the surface, fog, pollution $\Rightarrow$ accurate advection is desirable
	\item cubicFit offers a good balance between accuracy and computational cost, and achieves accurate results near the ground
\end{itemize}

\TODO{why isn't linearUpwind good enough?  what other schemes compete with cubicFit and what can cubicFit offer in comparison with these competitors?}

\TODO{cubicFit belongs to a new class of finite volume transport scheme that use a least squares fit to approximate face values (is this true? what about OpenFOAM's existing \texttt{xxxFit} schemes?)}

\section{Types of meshes}
\subsection{Meshes for representing terrain}
\label{sec:meshes-terrain}

\TODO{describe BTF, cut cell method and slanted cell method.  Cite ASAM for the cut cell method.  Describe slanted cell method and cite \citep{shaw-weller2016}.}

\begin{equation}
	z = \left( H - h \right) \left( z^\star / H \right) + h \label{eqn:btf}
\end{equation}

\subsection{Meshing a spherical Earth}

\TODO{Describe icosahedral meshes (cite one or more of \citep{thuburn2014,heikes-randall1995a,heikes-randall1995b})} \\
\TODO{Describe cubed-sphere mesh (cite someone?)}

Following appendix A9 in \citep{lauritzen2014}, the average equatorial spacing $\Delta \lambda$ is used as a measure of mesh spacing.  It is defined as
\begin{align}
	\Delta \lambda = \ang{360} \frac{\overline{\Delta x}}{2 \pi R_e}
\end{align}
where $\overline{\Delta x}$ is the mean distance between cell centres and $R_e = \SI{6.3712e6}{\meter}$ is the radius of the Earth.

\section{Transport schemes for arbitrary meshes}
The cubicFit transport scheme is described here for arbitrary two-dimensional meshes and arbitrary, single-layer spherical meshes.  Section \ref{sec:results} compares results using the cubicFit scheme with results using the linearUpwind transport scheme, and so a description of the linearUpwind scheme is also provided here.

\TODO{move the derivation from the advection equation and timestepping details to here since they're common to both cubicFit and linearUpwind}

\TODO{somewhere in here we need to state the multidimensional Courant number and note that the Courant number may vary between cells belonging to the same mesh.  We should then say that the stability limit for both schemes is $\max{\mathrm{Co}} \leq 1$.}

\input{cubicFit}

\subsection{linearUpwind transport scheme}
\TODO{describe OpenFOAM's linearUpwind scheme and cite something? OpenFOAM docs? OpenFOAM github?}
\TODO{I need to mention that we use a Dirichlet boundary condition of $\phi = \SI{0}{\kilogram\per\meter\cubed}$ at the ground.  Results are unstable if I use  zero gradient boundary condition.}

\begin{figure}
	\caption{\TODO{I'll probably need a figure that shows linearUpwind's stencil and shows how it approximates $\phi_F$}}
\end{figure}



\section{Results}
\label{sec:results}

\TODO{the `slug' advection test case makes slanted cells look bad compared to BTF.  should we redress the balance with a different test?}
\TODO{somewhere mention that the second-order convergence is a limitation of the divergence discretisation.  With more DoF a higher order should be achievable.} \\
\TODO{should I do eigenmode analysis?  this would prove stability for particular meshes for arbitrary wind fields} \\
\TODO{somewhere define error norms}

\subsection{Transport over a mountainous lower boundary}
This test is a modification of the \citet{schaer2002} horizontal advection test.  The mountain height is raised, the wind field is aligned with the the terrain-following surfaces, and the tracer is moved downward so that it is advected over the ground.

\begin{itemize}
	\item Compare cubicFit with linearUpwind
	\item Compare errors on BTF, cut cells and slanted cells using a small timestep
	\item Show maximum timesteps for various mesh spacings using Courant number close to one
\end{itemize}

\begin{figure}
	\centering
	\includegraphics{../fig-mountainAdvection-meshes/fig-mountainAdvection-meshes.pdf}
	\caption{\TODO{BTF, cut cell and slanted cell meshes used for the slug over a mountain test}}
\end{figure}

\begin{figure}
	\centering
	\includegraphics{../fig-mountainAdvection-error/fig-mountainAdvection-error.pdf}
	\caption{\TODO{evolution of the slug over a mountain at $t=0$, $t=T/2$ and $t=T$.  mountain advection error contours for (left-to-right) BTF, cut cells and slanted cells; linearUpwind (top) and cubicFit (bottom).  Tracer contours 0.1.  Error contours 0.01.} \\
	\TODO{I could overlay l2 and linf errors onto these plots.  Might be nicer than tabulating them separately.}}
\end{figure}

\begin{figure}
	\centering
	\includegraphics{../fig-mountainAdvection-maxdt/fig-mountainAdvection-maxdt.pdf}
	\caption{\TODO{mountain advection maximum timesteps for BTF, cut cells and slanted cells for various mesh spacings.  Demonstrates first that cubicFit has no problems near the limit of stability and, second, that slanted cells scale predictably with mesh spacing.}}
\end{figure}



\subsection{Deformational flow on a sphere}

\TODO{how much detail do I need about OpenFOAM's global Cartesian coordinates, lack of 2D meshes and our correction for spherical geometry?}

Tests on cubed sphere and hexagons, again comparing cubicFit against linearUpwind.  \citet{lauritzen2012} had six classes of test and we will reproduce three of them:
\begin{enumerate}
	\item convergence tests with Gaussian hills
	\item minimal resolution test with cosine bell
	\item cosine bell in divergent flow
\end{enumerate}
We will not consider filament preservation, a "rough" distribution with a slotted cylinder, or correlation preservation.

\subsubsection{Numerical order of convergence using Gaussian hills}
\begin{figure}
	\centering
	\includegraphics{../fig-deformationSphere-initialTracer/fig-deformationSphere-initialTracer.pdf}
	\caption{\TODO{evolution of deformational flow test cases for Gaussian hills with plots at $t=0$, $t=T/2$ and $t=T$.  The analytic solution at $t=T$ is identical to the initial condition.  Cosine bells initial condition also plotted.  This figure is supposed to give a sense of what `should' happen, so plot at a high resolution using whichever mesh gives better results.}}
\end{figure}

\begin{figure}
	\centering
	\includegraphics{../fig-deformationSphere-gaussiansConvergence/fig-deformationSphere-gaussiansConvergence.pdf}
	\caption{\TODO{deformational flow l2 and linf convergence plots comparing cubed sphere and hexagons, cubicFit and linearUpwind.  This figure is comparable to \citet{lauritzen2012} figure 4.}}
\end{figure}

\subsubsection{``Minimal'' resolution using cosine bells}

\begin{figure}
	\centering
	\includegraphics{../fig-deformationSphere-cosBellsConvergence/fig-deformationSphere-cosBellsConvergence.pdf}
	\caption{\TODO{$\ell_2$ convergence for non-divergent deformational flow using Cosine bells.  Used to find ``minimal'' resolution.  Plot for hexagons and cubed sphere, cubicFit and linearUpwind.  Plot a heavy line for minimal resolution, as in \citet{lauritzen2012} figure 5.}}
\end{figure}

\subsubsection{Transport under divergent flow conditions using cosine bells}

\begin{figure}
	\centering
	\includegraphics{../fig-deformationSphere-divergentTracer/fig-deformationSphere-divergentTracer.pdf}
	\caption{\TODO{divergent flow at $t=T/2$ and $t=T$ comparing cubed sphere and hexagons, cubicFit and linearUpwind.  Corresponds to \citet{lauritzen2012} figure 9.}}
\end{figure}



\section{Conclusions}

The advection scheme is
\begin{itemize}
	\item suitable for complex flows on a variety of meshes
	\item computationally cheap at runtime, with more expensive computations depending only on the mesh geometry
	\item \TODO{convergence}
	\item stable for Courant numbers up to 1
\end{itemize}

\section{Acknowledgements}
\TODO{Supervisors, funding bodies.  ASAM group for the mesh generator---I should ask permission to use cut cell meshes in this paper.  Dr Tristan Pryer.  Dr Shing Hing Man.}

\section*{Appendix: One-dimensional von Neumann stability analysis}
Two analyses are performed in order to find stability constraints on the weights $\mathbf{w} = \mathbf{\tilde{b}_1^+} \cdot \mathbf{m}$ as appear in equation~\eqref{eqn:weightedPinv}.  The first analysis uses two points to derived separate constraints on the upwind weight $w_u$ and downwind weight $w_d$.  The second analysis uses three points to derive a constraint that considers all weights in a stencil.

\subsection*{Two-point analysis}
We start with the conservation equation for a dependent variable $\phi$ that is discrete-in-space and continuous-in-time
\begin{align}
\frac{\partial \phi_j}{\partial t} &= - u \frac{\phi_R - \phi_L}{\Delta x} \label{eqn:advectionLR} \\
%
\shortintertext{where the left and right fluxes, $\phi_L$ and $\phi_R$ are weighted averages of the neighbouring points.  Assuming that $u$ is positive}
%
\phi_L &= \alpha_u \phi_{j-1} + \alpha_d \phi_j \\
\phi_R &= \beta_u \phi_j + \beta_d \phi_{j+1}
\end{align}
where $\alpha_u$ and $\beta_u$ are the upwind weights and $\alpha_d$ and $\beta_d$ are the downwind weights for the left and right fluxes respectively.  A subscript $j$ denotes the value at a given point $x = j \Delta x$ where $\Delta x$ is a uniform mesh spacing.

At a given time $t = n \Delta t$ at time-level $n$ and with a time-step $\Delta t$, we assume a wave-like solution with an amplification factor $A$, such that
\begin{align}
	\phi_j^{(n)} &= A^n e^{\iu j k \Delta x}
\end{align}
where $\phi_j^{(n)}$ denotes a value of $\phi$ at position $j$ and time-level $n$.  Using this to rewrite the left-hand side of equation~\eqref{eqn:advectionLR}
\begin{align}
\frac{\partial \phi_j}{\partial t} &= \frac{\partial}{\partial t} \left( A^{t / \Delta t} \right) e^{ijk\Delta x} = \frac{\ln A}{\Delta t} A^n e^{ikj\Delta x} \\
%
\shortintertext{hence equation~\eqref{eqn:advectionLR} becomes}
%
\frac{\ln A}{\Delta t} &= - \frac{u}{\Delta x} \left( \beta_u + \beta_d e^{ik\Delta x} - \alpha_u e^{-ik\Delta x} - \alpha_d \right) \\
\ln A &= -c \left( \beta_u - \alpha_d + \beta_d \cos k\Delta x + \iu \beta_d \sin k \Delta x - \alpha_u \cos k\Delta x + \iu \alpha_u \sin k\Delta x \right)
%
\intertext{where the Courant number $c = u \Delta t / \Delta x$.
Let $\Re = \beta_u - \alpha_d + \beta_d \cos k\Delta x - \alpha_u \cos k\Delta x$ and
$\Im = \beta_d \sin k \Delta x + \alpha_u \sin k\Delta x$, then}
%
\ln A &= -c \left( \Re + \iu \Im \right) \\
A &= e^{-c \Re} e^{-\iu c \Im} \\
%
\shortintertext{and the complex modulus and complex argument of $A$ are found to be}
%
|A| &= e^{-c \Re} = \exp \left( -c \left( \beta_u - \alpha_d + \left(\beta_d - \alpha_u \right) \cos k\Delta x \right) \right) \quad \text{and} \\
\arg(A) &= -c \Im = -c \left( \beta_d + \alpha_u \right) \sin k\Delta x
\end{align}
For stability, we need $|A| \leq 1$ and $\arg(A) < 0$ for $c > 0$, so
\begin{align}
\beta_u - \alpha_d + \left( \beta_d - \alpha_u \right) \cos k\Delta x &\geq 0 \quad \forall k\Delta x \quad \text{and} \\
\beta_d + \alpha_u &> 0 \label{eqn:arg-ineq}
\end{align}
Imposing the additional constraints that $\alpha_u = \beta_u$ and $\alpha_d = \beta_d$:
\begin{align}
|A| &= \exp \left( -c \left( \alpha_u - \alpha_d \right) \left(1 - \cos k\Delta x \right) \right)
%
\intertext{and given $1 - \cos k\Delta x \geq 0$ for well-resolved waves, then}
%
\alpha_u - \alpha_d &\geq 0 \\
%
\shortintertext{which provides a lower bound on $\alpha_u$:}
\alpha_u &\geq \alpha_d \label{eqn:lower-bound}
\end{align}
Additionally, we do not want more damping than an upwind scheme (where $\alpha_u = \beta_u = 1$, $\alpha_d = \beta_d = 0$), having an amplification factor, $A_\mathrm{up}$:
\begin{align}
|A_\mathrm{up}| &= \exp \left( -c \left(1 - \cos k\Delta x \right) \right)
%
\intertext{So we need $|A| \geq |A_\mathrm{up}|$:}
%
-c \left( \alpha_u - \alpha_d \right) \left(1 - \cos k\Delta x \right) &\geq -c \left( 1- \cos k\Delta x \right) \\
\alpha_u - \alpha_d &\leq 1 \\
\alpha_u &\leq 1 + \alpha_d
%
\intertext{which provides an upper bound on $\alpha_u$.  Combining with eqn~\eqref{eqn:lower-bound} we can bound $\alpha_u$ on both sides:}
%
\alpha_d \leq \alpha_u \leq 1 + \alpha_d
\end{align}
Now, assume that $\alpha_u + \alpha_d = 1$ (or $\alpha_d = 1 - \alpha_u$), then
\begin{align}
	1 - \alpha_u < \alpha_u &\leq 1 + (1 - \alpha_u) \\
	0.5 \leq \alpha_u &\leq 1
%
\shortintertext{and, since $\alpha_u + \alpha_d = 1$, then}
%
	0 \leq \alpha_d &\leq 0.5
\end{align}


\subsection*{Three-point analysis}


\bibliographystyle{ametsoc2014}
\bibliography{references}

\end{document}
