\begin{abstract}
The inclusion of terrain in atmospheric models gives rise to mesh distortions near the lower boundary that can degrade accuracy and challenge the stability of transport schemes.
Multidimensional transport schemes avoid splitting errors on distorted and arbitrarily structured meshes, and method-of-lines schemes have low computational cost because they perform reconstructions at fixed points.

This paper presents a multidimensional method-of-lines finite volume transport scheme, ``cubicFit'', which is designed to be numerically stable on arbitrary meshes.
Constraints derived from a von Neumann stability analysis are imposed during model initialisation to obtain stability and improve accuracy in distorted regions of the mesh, and near steeply-sloping lower boundaries.
The reconstruction method has a low computational cost because most calculations depend upon the mesh only, with just one vector multiply per face needed per time-step.

The cubicFit scheme is evaluated using three, idealised numerical tests of atmospheric flow.  The first is a more challenging variant of a standard horizontal transport test on a severely distorted terrain-following mesh.  The second is a new test case that assesses accuracy near the ground by transporting a tracer at the lower boundary over steep terrain on terrain-following and cut-cell meshes.
The third is a standard test case that deforms a tracer in a vortical flow on hexagonal icosahedra and cubed-sphere meshes.
In all tests, cubicFit is stable and largely insensitive to mesh distortions, and cubicFit results are more accurate than those obtained using a multidimensional linear upwind transport scheme.  The cubicFit scheme is second-order accurate regardless of mesh distortions.
\end{abstract}

