\subsection{Transport over a mountainous lower boundary}
A two-dimensional transport test over mountains was developed in \citep{schaer2002} to study the effect of terrain-following coordinate transformations on numerical accuracy.  In this standard test, a tracer is positioned aloft and transported horizontally over wave-shaped terrain.  This test presents no particular challenge on cut cell meshes because there is no wind and zero tracer density near the ground \citep{good2014}.
Here we present a variation of this standard test case that challenges transport schemes on all mesh types: positioning the tracer near the ground and modifying the wind field so that it is tangential to terrain-following coordinate surfaces allows us to assess the accuracy of the cubicFit scheme near the lower boundary.

The domain is defined on an $x$--$z$ plane that is \SI{301}{\kilo\meter} wide and \SI{25}{\kilo\meter} high as measured between parallel boundary edges.  The domain is subdivided into a $301 \times 50$ mesh such that $\Delta x = \SI{1}{\kilo\meter}$ and $\Delta z = \SI{500}{\meter}$.

The terrain is wave-shaped, specified by the surface height $h$ such that
\begin{subequations}
\begin{align}
   h(x) &= h^\star \cos^2 ( \alpha x )
%
\intertext{where}
%
   h^\star(x) &= \left\{ \begin{array}{l l}
       h_0 \cos^2 ( \beta x ) & \quad \text{if $| x | < a$} \\
	0 & \quad \text{otherwise}
    \end{array} \right.
\end{align}
\end{subequations}
where $a = \SI{25}{\kilo\meter}$ is the mountain envelope half-width, $h_0 = \SI{3}{\kilo\meter}$ is the maximum mountain height, $\lambda = \SI{8}{\kilo\meter}$ is the wavelength, \(\alpha = \pi / \lambda\) and \(\beta = \pi / (2a)\).
Basic terrain following, cut cell and slanted cell meshes are constructed by modifying the uniform $301 \times 50$ mesh using this terrain profile.  The details of the various mesh generation methods were given in section~\ref{sec:meshes}.

\TODO{describe wind field}

A tracer with density $\phi$ is positioned upwind of the mountain at the ground.  It has the shape
\begin{align}
	\phi(x, z) &= \phi_0 \left\{ \begin{array}{l l}
		\cos^2 \left( \frac{\pi r}{2} \right) & \quad \text{if $r \leq 1$} \\
		0 & \quad \text{otherwise}
	\end{array} \right.
%
\intertext{with radius $r$ given by}
%
	r &= \sqrt{
		\left( \frac{x - x_0}{A_x} \right)^2 + 
		\left( \frac{z - z_0}{A_z} \right)^2
	}
\end{align}
where $A_x = \SI{25}{\kilo\meter}$, $A_z = \SI{10}{\kilo\meter}$ are the horizontal and vertical half-widths respectively, and $\phi_0 = \SI{1}{\kilogram\per\meter\cubed}$ is the maximum density of the tracer.  At $t = \SI{0}{\second}$, the tracer is centred at $(x_0, z_0) = (\SI{-50}{\kilo\meter}, \SI{0}{\kilo\meter})$ so that the tracer is upwind of the mountain and centred at the ground.
Tests are integrated forward in time for \SI{10000}{\second}.

\TODO{explain how we calculate the analytic solution}

\begin{itemize}
	\item Compare cubicFit with linearUpwind
	\item Compare errors on BTF, cut cells and slanted cells using a small timestep
	\item Show maximum timesteps for various mesh spacings using Courant number close to one
\end{itemize}

\begin{figure}
	\centering
	\includegraphics{../fig-mountainAdvection-meshes/fig-mountainAdvection-meshes.pdf}
	\caption{\TODO{BTF, cut cell and slanted cell meshes used for the slug over a mountain test}}
\end{figure}

\begin{figure}
	\centering
	\includegraphics{../fig-mountainAdvection-error/fig-mountainAdvection-error.pdf}
	\caption{\TODO{evolution of the slug over a mountain at $t=0$, $t=T/2$ and $t=T$.  mountain advection error contours for (left-to-right) BTF, cut cells and slanted cells; linearUpwind (top) and cubicFit (bottom).  Tracer contours 0.1.  Error contours 0.01.} \\
	\TODO{I could overlay l2 and linf errors onto these plots.  Might be nicer than tabulating them separately.}}
\end{figure}

\begin{figure}
	\centering
	\includegraphics{../fig-mountainAdvection-maxdt/fig-mountainAdvection-maxdt.pdf}
	\caption{\TODO{mountain advection maximum timesteps for BTF, cut cells and slanted cells for various mesh spacings.  Demonstrates first that cubicFit has no problems near the limit of stability and, second, that slanted cells scale predictably with mesh spacing.}}
\end{figure}

