Numerical simulations of atmospheric flows solve equations of motion that result in the transport of momentum, temperature, water species and trace gases.  The numerical representation of Earth's terrain complicates the transport problem because the mesh is necessarily distorted by the modification of the lower boundary.
As new atmospheric models use increasingly fine mesh spacing, meshes are able to resolve steep, small-scale slopes.  Transport across mesh surfaces tends to be more common near steep slopes, and this misalignment between the flow and mesh surfaces increases numerical errors \citep{schaer2002,shaw-weller2016}.
Further, operational weather forecasts must balance numerical accuracy and computational efficiency to meet strict time-to-solution constraints.  Hence, an efficient transport scheme is desired that yields accurate solutions, particularly near the surface where the weather affects us directly.
We present a new transport scheme that is generally insensitive to mesh distortions created by steep slopes, and achieves computational efficiency because most calculations are not repeated every time-step because they depend upon the mesh geometry only.

There are two main methods for representing terrain in atmospheric models: terrain-following layers and cut cells.  Both methods modify regular quadrilateral meshes to produce irregular meshes with cells that are aligned in vertical columns.  Most operational models use terrain-following layers in which horizontal mesh surfaces are moved upwards to accommodate the terrain.  A vertical decay function is chosen so that mesh surfaces slope less steeply with increasing height.
The most straightforward is the linear decay function used by the basic terrain-following transform \citep{galchen-somerville1975}.  Over time, more complex decay functions have been developed so that mesh surfaces are smoother \citep{simmons-burridge1981,schaer2002,leuenberger2010,klemp2011}.

An alternative to terrain-following layers is the cut cell method.  Cut cell meshes are constructed by `cutting' a regular quadrilateral mesh with a piecewise-linear representation of the terrain.  New vertices are created where the terrain intersects mesh edges, and cell volumes that lie beneath the ground are removed.  Cut cell meshes can have arbitrarily small cells that impose severe timestep constraints on explicit transport schemes.  Several techniques have been developed to allieviate this problem, known as the `small-cell problem': small cells can be merged with adjacent cells \citep{yamazaki2016}, cell volumes can be artificially increased \citep{steppeler2002}, or an implicit scheme can be used near the ground with an explicit scheme used aloft \citep{jebens2011}.

Another method for avoiding the small-cell problem was proposed by \citep{shaw-weller2016} in which cell vertices are moved vertically so that they are positioned at the terrain surface.  In this paper the method is referred to as the slanted cell method in order to distinguish it from the traditional cut cell method.  Slanted cell meshes do not suffer from arbitrarily small cells because the horizontal cell dimensions are not modified by the presence of terrain.

As well as vertical distortions over sloping terrain, transport schemes must also deal with horizontal mesh distortions on the sphere.  A wide variety of spherical meshes are used in global models \citep{staniforth-thuburn2012}.  The latitude--longitude mesh is in widespread use because it is orthogonal and can be represented by a computationally rectangular mesh.  However, small cells near the poles constrain the timestep for explicit methods, and parallel computation becomes inefficient.  Because of these problems there has been increased interest in alternative, quasi-uniform meshes.

\TODO{this paragraph seems disjoint from the others}
The quasi-uniform cubed-sphere mesh retains the quadrilateral cells found on the latitude--longitude mesh, and improves uniformity at the expense of orthogonality.  Starting with a cube with each face being a Cartesian mesh, the cubed-sphere mesh is constructed by a gnomic projection of each face onto the sphere.
Another class of quasi-uniform spherical mesh is constructed by successive refinement of an icosahedron.  The resultant mesh can be modified in different ways to improve orthogonality and hence numerical accuracy \citep{heikes-randall1995b,tomita2002}.  A thorough review of spherical meshes for atmospheric models is presented in \citep{staniforth-thuburn2012}.
\TODO{Most hexagonal icosahedral models use a completely orthogonal grid but skewness and uniformity can still be optimised}

Transport schemes may be classified as dimensionally-split, 2D multidimensional, or 3D multidimensional.  2D multidimensional schemes are often simply called `multidimensional' schemes since very few 3D multidimensional schemes have been used in atmospheric models.
Perhaps confusingly, dimensionally-split schemes are sometimes called multidimensional, too, because they use one-dimensional techniques for multidimensional transport.

Dimensionally-split schemes such as \citep{lin-rood1996,putman-lin2007,katta2015} calculate transport in each dimension separately before the flux contributions are combined.  Such schemes are computationally efficient and allow existing one-dimensional high-order methods to be used.
  Dimensionally-split schemes can suffer from `splitting errors' in which the tracer is artificially distorted when the velocity field is misaligned with the grid \citep{leonard1993}.  Errors can be reduced by explicitly accounting for transverse fluxes when combining fluxes \citep{leonard1996}, but splitting errors are nevertheless apparent in flows over steep terrain where meshes are highly distorted \citep{weller2017}.

In 2D multidimensional schemes the horizontal dimensions are considered together.  Such schemes are extended for three-dimensional transport by using splitting techniques in the vertical dimension.
There are several subclasses of 2D multidimensional schemes that include swept-area schemes (also called flux-form semi-Lagrangian or incremental remapping), 2D semi-Lagrangian finite volume schemes (also called conservative mesh remapping), and method-of-lines schemes.
Swept area schemes such as \citep{lashley2002,skamarock-menchaca2010,lauritzen2011,thuburn2014} calculate the flux through a cell face by integrating over the upstream area that is swept out over one time-step.  Such schemes differ in their choice of area approximation and spatial integration method.
2D semi-Lagrangian finite volume schemes such as \citep{iske-kaeser2004,lauritzen2010} integrate over departure cells that are found by tracing backward the trajectories of cell vertices.  These schemes are conservative because departure cells are constructed so that they do not overlap.
Method-of-lines schemes such as \citep{weller2009,skamarock-gassmann2011} use a spatial discretisation to reduce the transport PDE to an ODE that is typically solved using a multi-stage timestepping method.
There are many more types of atmospheric transport schemes, but all can be classified according to their treatment of the three spatial dimensions.  A more comprehensive overview is presented in \cite{lauritzen2014}.

Very few 3D multidimensional schemes have been used in atmospheric models \citep[e.g.][]{gassmann2013} although such schemes might be expected to be more accurate on horizontally non-orthogonal, spherical meshes with steep terrain.  The multidimensional method-of-lines schemes developed in \citep{weller2009,weller-shahrokhi2014} have only been used for two-dimensional flows on Cartesian planes and on the sphere, but the technique used naturally generalises to three dimensions.
In this paper, we present a new multidimensional method-of-lines scheme, `cubicFit', that improves upon the scheme in \citep{weller-shahrokhi2014}.  To reconstruct face values, the scheme fits a multidimensional polynomial over a cubic, upwind-biased stencil using a least-squares approach.  The implementation uses constraints derived from a von Neumann stability analysis to select appropriate polynomial fits for stencils in highly-distorted mesh regions.  The least-squares procedure depends upon the mesh geometry only, hence the scheme is computationally efficient, requiring only $n$ multiplies per cell face per time-step where $n$ is the size of the stencil.
\TODO{outline contents of subsequent sections}


%* lauritzen's chapter says that constraints to ensure conservation are stricter with semi-Lagrangian scheme compared to Eulerian because they must ensure that departure areas have no overlaps nor gaps

%how cheap are Eulerian schemes?
%* cubicFit requires n multiplies per face
%* what about skamarock-gassmann?
%  - n multiplies per face (stencil is smaller than ours)
%  * what about miura2007?
%  * what about skamarock-menchaca?

%  how cheap are swept-area schemes?
%  * "simplified" flux integration uses sub-grid reconstruction of the immediate upstream cell
%  * SFF-CSLAM has rigorous area approximation, traces each vertex upstream, integration by Gaussian quadrature
%  * miura2007 (ICON) is simpler: one velocity per face, swept area is a rhomboid, first-order (linear) reconstruction polynomial -- first-order in space, second-order in time
%  * are there any "cheap" swept-area schemes that are properly second-order?


%\TODO{talk about accuracy near the ground: what is there in the literature about this?  there's a bunch of stuff that's not met-specific}
% walko-avissar Weak oscillations generated near the lower boundary in OLAM 
% almgren1997 says that losing 2nd order accuracy at boundary should not be a problem for atmospheric flows over orography -- **do we think this is (still) true?**
% ye1999 achieve 2nd order near the boundary but use dimensional splitting
% zaengl2012: ICON, noisy small-scale circulations with weak/no ambient flow, see zaengl2002, zaengl2004, klemp2011


% something about why we want accuracy near the ground for weather forecasts: clouds can form near the surface, fog, pollution etc
% describe existing classes of transport schemes used in atmospheric models (use lauritzen2014)
% existing uses of least squares fit


