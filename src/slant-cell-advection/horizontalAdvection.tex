\subsection{Horizontal transport over mountains}
A two-dimensional transport test was developed by Sch\"{a}r et al. \citep{schaer2002} to study the effect of terrain-following coordinate transformations on numerical accuracy.  In this standard test, a tracer is positioned aloft and transported horizontally over wave-shaped mountains.  The test challenges transport schemes because the tracer must cross mesh layers, which acts to reduce numerical accuracy \citep{schaer2002,shaw-weller2016}.
Here we use a more challenging variant of this test, as described by Weller et al. \citep{weller2017}, that has steeper mountains and highly-distorted terrain-following meshes.
Results are compared using the linearUpwind and cubicFit transport schemes using a range of mesh spacings.

The domain is defined on a rectangular $x$--$z$ plane that is \SI{301}{\kilo\meter} wide and \SI{25}{\kilo\meter} high as measured between parallel boundary edges.
Boundary conditions are imposed on the tracer density $\phi$ such that $\phi = \SI{0}{\kilo\gram\per\meter\cubed}$ at the inlet boundary, and a zero normal gradient
$\partial \phi / \partial n = \SI{0}{\kilo\gram\per\meter\tothe{4}}$ is imposed at the outlet, ground and top boundaries.
A range of mesh spacings are chosen so that $\Delta x \mathbin{:} \Delta z = 2\mathbin{:}1$ to match the original test specification from Sch\"{a}r et al. \citep{schaer2002}.

The terrain is wave-shaped, specified by the surface height $h$ such that
\begin{subequations}
\begin{align}
   h(x) &= h^\star \cos^2 ( \alpha x )
%
\intertext{where}
%
   h^\star(x) &= \left\{ \begin{array}{l l}
       h_0 \cos^2 ( \beta x ) & \quad \text{if $| x | < a$} \\
	0 & \quad \text{otherwise}
    \end{array} \right.
\end{align}
\end{subequations}
where $a = \SI{25}{\kilo\meter}$ is the mountain envelope half-width, $h_0 = \SI{6}{\kilo\meter}$ is the maximum mountain height, $\lambda = \SI{8}{\kilo\meter}$ is the wavelength, \(\alpha = \pi / \lambda\) and \(\beta = \pi / (2a)\).  Note that, in order to make this test more challenging, the mountain height $h_0$ is double the mountain height used by \citep{schaer2002}.

A basic terrain-following (BTF) mesh is constructed by modifying the uniform mesh using this terrain profile.
The BTF method uses a linear decay function so that mesh surfaces become horizontal at the top of the model domain \citep{galchen-somerville1975},
\begin{equation}
	z(x,y) = \left( H - h(x,y) \right) \left( z^\star / H \right) + h(x,y) \label{eqn:btf}
\end{equation}
where $z$ is the geometric height at a horizontal point $(x, y)$, $H$ is the height of the domain, $h(x,y)$ is the surface elevation and $z^\star$ is the computational height of a mesh surface.  If there was no terrain then $h = 0$ and $z = z^\star$.

A velocity field is prescribed with uniform horizontal flow aloft and zero flow near the ground,
\begin{align}
	u(z) = u_0 \left\{ \begin{array}{l l}
		1 & \quad \text{if $z \geq z_2$} \\
		\sin^2 \left( \frac{\pi}{2} \frac{z - z_1}{z_2 - z_1} \right) & \quad \text{if $z_1 < z < z_2$} \\
		0 & \quad \text{otherwise}
	\end{array} \right.	
\end{align}
where $u_0 = \SI{10}{\meter\per\second}$, $z_1 = \SI{7}{\kilo\meter}$ and $z_2 = \SI{8}{\kilo\meter}$, as specified by Weller et al. \citep{weller2017}.

A tracer with density $\phi$ has the shape
\begin{subequations}
\begin{align}
	\phi(x, z) &= \phi_0 \left\{ \begin{array}{l l}
		\cos^2 \left( \frac{\pi r}{2} \right) & \quad \text{if $r \leq 1$} \\
		0 & \quad \text{otherwise}
	\end{array} \right.
%
\intertext{with radius $r$ given by}
%
	r &= \sqrt{
		\left( \frac{x - x_0}{A_x} \right)^2 + 
		\left( \frac{z - z_0}{A_z} \right)^2
	}
\end{align}
%
\label{eqn:tracer}
\end{subequations}
where $A_x = \SI{25}{\kilo\meter}$, $A_z = \SI{9}{\kilo\meter}$ are the horizontal and vertical half-widths respectively, and $\phi_0 = \SI{1}{\kilogram\per\meter\cubed}$ is the maximum density of the tracer.  At $t = \SI{0}{\second}$, the tracer is centred at $(x_0, z_0) = (\SI{-50}{\kilo\meter}, \SI{12}{\kilo\meter})$ so that the tracer is upwind of the mountain and centred at the ground.

\begin{figure}
	\centering
	\includegraphics{../fig-horizontalAdvection-convergence/fig-horizontalAdvection-convergence.pdf}
%
	\caption{Numerical convergence of the two-dimensional horizontal transport test over mountains.  $\ell_2$ errors (equation~\ref{eqn:l2-error}) and $\ell_\infty$ errors (equation~\ref{eqn:linf-error}) are marked at mesh spacings between \SI{5000}{\meter} and \SI{250}{\meter} using linearUpwind and cubicFit transport schemes on basic terrain-following meshes. \TODO{get \SI{250}{\meter} result for cubicFit}}
	\label{fig:horizontalAdvection-convergence}
\end{figure}

Tests are integrated for \SI{10000}{\second} using time-steps that are proportional to the mesh spacing so that the maximum Courant number is about \num{0.4}.  By the end of integration the tracer is positioned downwind of the mountain.
The analytic solution at $t = \SI{10000}{\second}$ is centred at $(x_0, z_0) = (\SI{50}{\kilo\meter}, \SI{10}{\kilo\meter})$.  Error norms are calculated by subtracting the analytic solution from the numerical solution,
\begin{align}
	\ell_2 &= \sqrt{\frac{\sum_c \left(\phi - \phi_T \right)^2 \mathcal{V}_c}{\sum_c \left(\phi_T^2 \mathcal{V}_c \right)}} \label{eqn:l2-error} \\
	\ell_\infty &= \frac{\max_c |\phi - \phi_T|}{\max_c |\phi_T|} \label{eqn:linf-error}
\end{align}
where $\phi$ is the numerical value, $\phi_T$ is the analytic value, $\sum_c$ denotes a summation over all cells $c$ in the domain, and $\max_c$ denotes a maximum value of any cell.

Tests were performed using the linearUpwind and cubicFit schemes at mesh spacings between $\Delta x = \SI{250}{\meter}$ and $\Delta x = \SI{5000}{\meter}$.
Numerical convergence in the $\ell_2$ and $\ell_\infty$ norms is plotted in figure~\ref{fig:horizontalAdvection-convergence}.
The linearUpwind scheme becomes second-order convergent only at higher resolutions, but the cubicFit scheme is second-order convergent at all but the coarsest mesh spacings where errors are saturated for both schemes.

This test demonstrates that, unlike linearUpwind, cubicFit is second-order convergent in the domain interior irrespective of mesh distortions.  In the next test, we assess the numerical accuracy of the transport schemes near a distorted, mountainous lower boundary.

