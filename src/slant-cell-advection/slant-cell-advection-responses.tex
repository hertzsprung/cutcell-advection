\documentclass[times]{elsarticle}
\usepackage{fullpage}
\usepackage{xcolor}
\usepackage{natbib}
\usepackage[hidelinks]{hyperref}
\usepackage{amsmath}
\usepackage{mathtools}
\usepackage[final,babel]{microtype}
\usepackage[utf8]{inputenc}
\usepackage[british]{babel}
\usepackage{csquotes}
\usepackage[T1]{fontenc}
\usepackage{siunitx}
\usepackage{enumitem}

\newcommand{\TODO}[1]{\textcolor{darkgray}{TODO \textit{#1}}}
\newcommand{\revone}[1]{\textcolor{purple}{\textbf{#1}}}
\newcommand{\revtwo}[1]{\textcolor{teal}{\textbf{#1}}}
\newcommand{\revthree}[1]{\textcolor{olive}{\textbf{#1}}}
\newcommand{\revother}[1]{\textcolor{orange}{\textbf{#1}}}

\newlist{comment}{enumerate}{1}                                                 
\setlist[comment,1]{label={\arabic{commenti}.},leftmargin=*,resume=comment}

\renewenvironment{abstract}{\global\setbox\absbox=\vbox\bgroup
\hsize=\textwidth\def\baselinestretch{1}%
\noindent\unskip\ignorespaces}
{\egroup}

\begin{document}
\begin{frontmatter}
\title{Responses to reviewers \\
\vspace*{0.5em}
\large{Multidimensional method-of-lines transport for atmospheric flows over steep terrain using arbitrary meshes} \\
\vspace*{0.5em}
\normalsize{JCOMP-D-17-00161}}
\author{James Shaw}
\author{Hilary Weller}
\author{John Methven}
\author{Terry Davies}

\begin{abstract}
We greatly appreciate the reviewers' detailed comments and questions.  The revised manuscript highlights changes with a bold typeface and the text colour corresponds to the section headings below.
	
The revised manuscript also highlights a small number of additional changes \revother{in orange}.  Notable additions are listed here:
\begin{enumerate}
	\item The dense subset definition was incorrect.  The definition has been corrected in the revised manuscript (\TODO{line nnn, page nnn}).
	\item The revised manuscript mentions a new transport scheme that is designed for cut cell meshes \citep{steppeler-klemp2017} (\TODO{line nnn, page nnn})
	\item Similar uses of the moving least-squares reconstruction in other disciplines are mentioned in the revised introduction (\TODO{line nnn, page nnn})
\end{enumerate}
\end{abstract}
\end{frontmatter}

\section*{\revone{Reviewer \#1}}
This paper presents a multidimensional method-of-lines finite-volume transport scheme,
which is second-order accurate and offers high solution quality and robustness under
strongly deformed meshes. Among others, the latter is of increasing importance for 
atmospheric models with complex orography. The proposed scheme is named "cubic-fit", 
as it fits multidimensional polynomials of up to third-order over an upwind stencil
using a least-squares approach. The construction of the polynomials takes into account 
constraints derived from a linear stability analysis for stencils in regions of  
highly-deformed computational meshes.

The accuracy and stability of the cubic-fit scheme is studied with a number of well 
chosen test cases that are relevant to atmospheric models. 

In my view, the presented developments definitely warrant publication in JCP. 
The content of the paper appears of high quality and adheres to the JCP standards. 
The text is well structured and clearly written. It only needs little improvement 
as given by the subsequent comments. 

I recommend acceptance of the paper subject to minor revisions by the authors.

\subsection*{Comments}

\begin{quotation}
\begin{comment}
\item  It is explained in various places that the cubic-fit scheme
    offers favourable computational efficiency due to the fact
    that most of the reconstruction depends on the mesh geometry
    only, and this part can be precomputed. It would be very interesting
    to see numbers in the paper. It doesn't have to be a comprehensive
    study, but just to get an impression of the cost, e.g. how does
    the runtime of the cubic-fit compare to the second-order upwind
    scheme? Or, if one runs cubic-fit and the upwind scheme such that
    both produce about the same $\ell_2$ error (by using a higher resolution
    with the upwind scheme), what is then the difference in runtime?
\end{comment}
\end{quotation}
\TODO{linearUpwind has $n+1$ multiplies where $n$ is the number of cells neighbouring the upwind cell}

\TODO{I also like the idea of comparing computational cost (measured in number of multiplies) for a fixed $\ell_2$ error.  It would be natural to do this for the `minimal' resolution measurement that we've already documented for linearUpwind and cubicFit.}

\begin{quotation}
\begin{comment}
\item  As only 2D simulations have been considered in the paper, I
    wondered how difficult it will be to extend the cubic-fit
    scheme to 3D? It could useful to provide a short comment
    about this, just whether complications are anticipated or
    not. (See also the subsequent comment.)
\end{comment}
\end{quotation}
\TODO{
\begin{itemize}
	\item The stencil construction procedure already applies to two or three dimensions
	\item The definition of candidate polynomials also generalises naturally to three dimensions
	\item Assuming that a 3D mesh comprises prismatic cells arranged in columns then the computational cost in 3D will be three times the computational cost in 2D
\end{itemize}
	I would like to get a 3D advection test working on a Cartesian mesh.  Maybe just transport the Sch\"ar blob in a 3D, diagonal velocity field.  I needn't include this in the manuscript
}

\begin{quotation}
\begin{comment}
\item  Appendix A, p.22: How restrictive is the 1D stability analysis
    for the multidimensional domains? May this become a more
    serious issue in 3D configurations?
\end{comment}
\end{quotation}
\TODO{This restriction could be referring to the amount of damping, or the number of polynomial terms that are removed.  To address the latter, I could plot \texttt{polynomialTerms} for faces of a particular mesh (and owner/neighbour choice).}

\TODO{Regarding damping, we did trade some accuracy on good-quality meshes for stability on certain poor-quality meshes, so the scheme is more damping than it could be.  Should we try to analyse this?}

\TODO{I'm not sure what to expect regarding stability/accuracy in 3D, especially on distorted meshes.}

\begin{quotation}
\begin{comment}
\item  In various places throughout the paper "icosahedra" is used
    instead of "icosadedral". I suggest the latter should be used
    consistently throughout the paper.
\end{comment}
\end{quotation}
Now using the term `hexagonal-icosahedral mesh' throughout.

\begin{quotation}
\begin{comment}
\item  p.3, end of first paragraph: I suggest to mention here a
    recent advancement of the finite-volume MPDATA (Kühnlein and
    Smolarkiewicz, J. Comput. Phys. 2017,
    \url{http://dx.doi.org/10.1016/j.jcp.2016.12.054}). Among others, this
    paper demonstrates applicability of MPDATA for 3D compressible
    atmospheric dynamics on arbitrary hybrid unstructured meshes.
\end{comment}
\end{quotation}
Thank you, included.

\begin{quotation}
\begin{comment}
\item  p.3, paragraph after Eq. (2b): if there is no specific need, I
    sugggest to remove "zero-dimensional".
\end{comment}
\end{quotation}
Removed.

\begin{quotation}
\begin{comment}
\item  p.4, last paragraph in Section 2.0: "methods are described next"
    instead of "methods described next".
\end{comment}
\end{quotation}
Fixed.

\begin{quotation}
\begin{comment}
\item  p.5, first paragraph: Is it "arbitrary meshes" or "arbitrary
    structured meshes"?
\end{comment}
\end{quotation}
Now using the term `arbitrary meshes' throughout.

\begin{quotation}
\begin{comment}
\item  p.5, last paragraph of Section 2.1.1.: "dirichlet" should be
    upper case "Dirichlet".
\end{comment}
\end{quotation}
Fixed.

\begin{quotation}
\begin{comment}
\item p.7, caption of Fig. 3, last sentence: It is stated that a
    von Neumann boundary condition is assumed. However, the
    footnote on p.5 says that Neumann BCs are excluded from the
    set of stencil boundary faces?
\end{comment}
\end{quotation}
The original manuscript incorrectly conflated two issues.  In figure 3 and in all test cases, there is no normal flow at the lower boundary and so no boundary condition is required.  A zero gradient boundary condition is applied at the outlet boundary of all two-dimensional tests.

The revised manuscript clarifies that the lower boundary in figure 3 has no normal flux.  The footnote on \TODO{page nnn} longer mentions the numerical instabilities associated with an inconsistent extrapolation because a Neumann boundary condition should never have been applied to the lower boundary.  The manuscript also clarifies boundary conditions in section 3.1 and section 3.2.

\begin{quotation}
\begin{comment}
\item p.9, caption of Fig. 4, first sentence: "A one-dimensional
    least squares fits" should probably read "One-dimensional
    least squares fits".
\end{comment}
\end{quotation}
Fixed.

\begin{quotation}
\begin{comment}
\item p.12, caption of Fig. 6, last but one sentence: "domain is
    shown" instead of "domain in shown".
\end{comment}
\end{quotation}
Fixed.

\begin{quotation}
\begin{comment}
\item Section 3: Why are most of the experiments run with maximum CFL
    numbers < 0.5 or even < 0.4, although the cubic-fit scheme
    permits a maximum CFL < 1 (Fig. 10)? Is this because the upwind
    scheme, to which cubic-fit is compared to, is more limited with
    respect to the time step for the considered test cases?
\end{comment}
\end{quotation}
Lauritzen et al. \citep{lauritzen2012} recommend using a `typical' Courant number for transport tests on the sphere, and so we follow their recommendation throughout.  This is now stated \TODO{on page nnn} in the revised manuscript.
A maximum Courant number of about 0.4 avoids any accuracy and stability issues caused by time-stepping.
\TODO{A separate series of tests confirm that the stability limit has a Courant number of\TODO{nnn}, as presented \TODO{on page nnn} in the revised manuscript.  Say something about how I'd wrongly assumed that the limit was $\mathrm{Co} = 1.0$.}

\begin{quotation}
\begin{comment}
\item p. 18, last but one paragraph, last sentence: correct "imprintingin".
\end{comment}
\end{quotation}
Fixed.


\section*{\revtwo{Reviewer \#2}}

\subsection*{General remarks}

The manuscript presents a new method-of-lines based transport scheme
in which fluxes are computed by fitting a multidimensional
polynomial over an upwind-biased set of grid cells. The scheme
has low computation and memory costs. It is suitable for
structured and unstructured grids, and maintains accuracy
on strongly distorted grids such as those used in atmospheric models
over steep terrain. An idealized analysis suggests how the
construction of the polynomial should be constrained to obtain
conditional stability of the scheme, and such stability appears
to be achieved in practice.
I believe the manuscript will be of interest to those developing
atmospheric models and to the wider CFD community. I would be happy to
recommend publication after the points below are addressed.


\subsection*{Main points}

\begin{quotation}
\begin{comment}
\setcounter{commenti}{0}
\item It is claimed (in several places) that the scheme requires `just one
vector multiply per face per time step'. Since the two-stage Heun scheme
(2) is used, surely the cost must be two vector multiplies per face per time
step (or, more generally, one vector multiply per face per {\it stage})?
\end{comment}
\end{quotation}
Correct.  The term `time-step' has been replaced with `time-stage' everywhere this calculation is discussed.

\begin{quotation}
\begin{comment}
\item It is claimed in several places that the scheme is stable, but some qualification
is appropriate. First, the scheme is {\it conditionally} stable; the Courant number
should not be too large. Then Appendix~A gives an analysis deriving constraints
on the polynomial reconstruction that should be satisfied for stability. However,
the analysis is idealized in various ways: one-dimensional, regular grid, constant $v$,
two- and three-cell stencils, and ignoring the time discretization (also see below).
That is fair enough---it might not be possible to generalize the analysis to the full
scheme---and it is clearly stated at the end of the Appendix that the constraints are
{\it hypothesized} to apply more generally. However, in section 2.1.4 the language is
stronger and suggests that these constraints are definitive: ``...stability constraints...'',
``...must satisfy...'', etc.
\end{comment}
\end{quotation}
A note has been included in the revised introduction to say that the scheme is conditionally stable (\TODO{line nnn, page nnn}).
Section 2.1.4 has been reworded to say, ``\revtwo{To avoid numerical instabilities, a simplified, one-dimensional von Neumann analysis was performed, presented in appendix A.  The analysis is used to impose three stability constraints on the weights $\mathbf{w}$}''.

\begin{quotation}
\begin{comment}
\item P14. The calculation of the longest stable timestep {\it assumes} that the stability
limit is given by $\mathrm{Co}=1$. Given the idealized nature of the analysis in
Appendix~A, and that most experiments are done with $\mathrm{Co} \approx 0.4$,
at least some empirical investigation of the stability limit should be discussed
to confirm that it is indeed (exactly or approximately) $\mathrm{Co}=1$.
\end{comment}
\end{quotation}
\TODO{Reviewer one raised similar concerns about the CFL criterion so I've put my response next to their comment for now.}

\begin{quotation}
\begin{comment}
\item The presentation of the analysis in Appendix~A is somewhat confusing. It appears
to use a time discretization by introducing a time step $\Delta t$ and an amplification
factor $A$, (and hence a Courant number $v \Delta t / \Delta x$) whereas the
analysis is, in fact, continuous in time. It might be clearer to set
$\phi(x_j , t) = \hat{\phi}(t) \mathrm{e}^{i j k \Delta x}$ and form an ODE for
$\hat{\phi}$, etc. I think the conclusion would be the same.
\end{comment}
\end{quotation}
\TODO{I've tried this and, yes, the conclusion is the same.  It's perhaps clearer to me using an amplification factor $A$ from the outset, but maybe only because that's the way I was taught to do stability analyses.  Nevertheless, I'm happy to revise as the reviewer suggests.}

\begin{quotation}
\begin{comment}
\item Might it be possible to include the time discretization (2) in the stability analysis?
It would be interesting to know if the weak instability of (2) (noted on p3) affects
the constraints (57), (58), (63).
\end{comment}
\end{quotation}
\TODO{I'd like to do this but I'm not sure how to proceed analytically.  There is a paper on Runge--Kutta stability for advection \citep{baldauf2008} that could help me.}
\TODO{If I choose to do no further analysis then the experimental results of maximum stable timesteps should suffice.}

\begin{quotation}
\begin{comment}
\item The Courant number as defined by (4) would appear to vanish for a non-divergent
flow. A more natural definition might be to sum only over outflow faces (and
remove the factor $1/2$). For non-divergent flow that would be equivalent to
replacing $\mathbf{u} \cdot \mathbf{S}_f$ by $| \mathbf{u} \cdot \mathbf{S}_f |$
in~(4).
\end{comment}
\end{quotation}
Equation (4) should use the absolute value $\lvert \mathbf{u} \cdot \mathbf{S}_f \rvert$.  This is how the Courant number is calculated in OpenFOAM\footnote{\url{https://github.com/OpenFOAM/OpenFOAM-dev/blob/master/src/finiteVolume/cfdTools/incompressible/CourantNo.H\#L39}}.  Equation (4) has been corrected in the revised manuscript.

\begin{quotation}
\begin{comment}
\item Section 3.2 and Table~1: what resolution is used?
\end{comment}
\end{quotation}
Added a sentence near the beginning of section 3.2.  The mesh spacing is the same as Sch\"ar et al. \citep{schaer2002} with $\Delta x = \SI{1000}{\meter}$ and $\Delta z = \SI{500}{\meter}$.

\subsection*{Minor points}

\begin{quotation}
\begin{comment}
\item It would be interesting to know if there was some rationale for the choice of
the polynomial (5) and the stencil discussed in  section 2.1.1. Given that the scheme
is second order convergent, might it be possible to use a lower degree polynomial
and smaller stencil?
\end{comment}
\end{quotation}
Section 2.1 now includes a comment on the choice, saying, ``\revtwo{The stencil is upwind-biased to improve numerical stability, and the multidimensional cubic polynomial is chosen to improve accuracy in the direction of flow \citep{leonard1993}.}'' (\TODO{line nnn, page nnn}).

A previous version of cubicFit was used in the dynamical core developed by Weller and Shahrokhi \citep{weller-shahrokhi2014} in order to correctly simulation Sch\"ar mountain waves \citep{schaer2002}.  They found that smaller stencils and lower degree polynomials produced incorrect results, although comparisons between advection schemes were not presented in the paper.

\begin{quotation}
\begin{comment}
\item The first two paragraphs on p2 refer to quadrilateral meshes. It seems the discussion
has slipped into the 2D vertical slice case without explicitly saying so.
\end{comment}
\end{quotation}
The term `quadrilateral' has been removed from these paragraphs since it is sufficient to say that the methods modify regular meshes.  These meshes may be rectilinear for a limited area domain or curvilinear for a global domain.

\begin{quotation}
\begin{comment}
\item P2, 5th paragraph. Perhaps mention that dimensionally-split schemes are only possible
when the grid permits it (e.g.\ not a hexagonal grid).
\end{comment}
\end{quotation}
The revised manuscript explains these limitations, ``\revtwo{Dimensionally-split schemes have only been used with quadrilateral meshes where dimensions are inherently separable.  Special treatment is required at the corners of cubed-sphere panels where local coordinates differ \citep{putman-lin2007,katta2015}.
For similar reasons, dimensionally-split schemes have only be used with terrain-following coordinate transforms and not cut cells.}''

\begin{quotation}
\begin{comment}
\item P9, last sentence of section 2.1. Should $w_d$ be $m_d$?
\end{comment}
\end{quotation}
\TODO{}
Yes, fixed.

\begin{quotation}
\begin{comment}
\item The discussion of previous work is generally very good. On p2 is it worth mentioning
the SLICE scheme of Zerroukat and co-authors? In the first paragraph of section 3.3, credit
for the hexagonal-icosahedral mesh should probably be given to Heikes and Randall.
\end{comment}
\end{quotation}
The revised manuscript now mentions the three-dimensional semi-Lagrangian finite volume scheme, SLICE-3D \citep{zerroukat-allen2012}.

In addition to citing Thuburn et al. \citep{thuburn2014}, the manuscript now cites the original Heikes and Randall articles \citep{heikes-randall1995a,heikes-randall1995b} on which the hexagonal-icosahedral mesh generator is based.

\begin{quotation}
\begin{comment}
\item Some brief comments on how the approach extends to three dimensions would be
appropriate.
\end{comment}
\end{quotation}
\TODO{Reviewer one also asked us to comment on an extension to 3D.  I've responded to their request for now.}


\subsection*{Typos, etc}

\begin{quotation}
\begin{comment}
\item Introduction, second sentence: `modification of the lower boundary' is an odd way to
think of it.
\end{comment}
\end{quotation}
Reworded to say, ``the mesh is necessarily distorted \revtwo{next to} the lower boundary''.

\begin{quotation}
\begin{comment}
\item P2, 5 lines from the bottom: `...non-simply connected domains...'. I found this
phrase confusing.
\end{comment}
\end{quotation}
The revised manuscript includes an explanation that is more intuitive and describes the example illustrated by Lauritzen et al. \citep{lauritzen2011}.

\begin{quotation}
\begin{comment}
\item P3 line 17: `cubic, upwind-biased stencil'. Surely it is the polynomial that is
cubic, rather than the stencil?
\end{comment}
\end{quotation}
Yes, reworded to say, ``the scheme fits \revtwo{a multidimensional cubic polynomial over an upwind-biased stencil}''.

\begin{quotation}
\begin{comment}
\item P3 below (2): `stable ... with some upwinding'. Presumably some qualification is required
(with sufficient upwinding)?
\end{comment}
\end{quotation}
Reworded to say ``with \revtwo{sufficient upwinding}'' as suggested.

\begin{quotation}
\begin{comment}
\item P5 footnote: could be made
\end{comment}
\end{quotation}
Fixed.

\begin{quotation}
\begin{comment}
\item P14 5 lines from the bottom: cubic should be cubicFit?
\end{comment}
\end{quotation}
Fixed.

\begin{quotation}
\begin{comment}
\item P18 2 lines from the bottom: imprinting in
\end{comment}
\end{quotation}
Fixed.


\section*{\revthree{Reviewer \#3}}
The author describes  a standard finite volume algorithm for solving the advection equation on unstructured
grids. Using a method of lines the spatial discretization consists in computing face values of a cell
centered given function by high order polynomial point wise approximation in contrast to a more common
area wise procedure. The proposed algorithm has three phases, choose of two upwind dominated stencils
for each face, determination of a subset of candidate polynomials with at most order three, weighting
of the individual contributions in a least squares approximation to increase stability with respect 
to the advection equation.

The stencil choice starts with the determination of opposing faces in the upwind cell with respect to 
to the destination face, and adding further cells in a two step procedure. The candidate polynomials
with at most order three are all polynomials where the corresponding least squares problem has full rank.
For the third step the candidate polynomials are ordered with respect to the number of unknowns and 
and larger minimal singular value in case of polynomials with the same number. In the second step
following the first ordering a weighting sequence is tested until a stability criterion is fulfilled
which can finally end in a simple upwind scheme. 

The paper is well written and illustrates all steps by intuitive examples. The new test example
for the two-dimensional example with the boundary following velocity field enriches the number of
problems for testing numerical advection schemes.

\subsection*{Comments and questions}

\begin{quotation}
\begin{comment}
\setcounter{commenti}{0}
\item Can you comment on the stencil for triangular grids. In case of a slightly perturbed triangular grid
with  equilateral triangles the stencil changes in comparison the unperturbed case.
\end{comment}
\end{quotation}
We haven't explored triangular meshes in this paper, and more work would be needed to evaluate the suitability of stencils and stability criteria on these meshes.
We believe that studies on triangular meshes are less worthwhile because they suffer from computational modes \citep{weller2012}.
We would not expect the stencil topology to change for triangles that are slightly perturbed from equilateral because such triangles will always have two opposing faces as given by equations (6) and (8).

\begin{quotation}
\begin{comment}
\item Why do you prefer the point wise approximation?
\end{comment}
\end{quotation}
Most swept-area approaches must enforce integral constraints on the polynomial reconstruction.  Unlike swept-area approaches that require one matrix-vector multiply per face per time-stage, a point-wise approximation requires only one vector-vector multiply.
The introduction has been modified to reinforce this point, saying, ``the cubicFit scheme has \revthree{a computational cost that is lower than most swept-area schemes and more comparable to dimensionally-split schemes}'' (\TODO{line nnn, page nnn}).

\begin{quotation}
\begin{comment}
\item The singular value test should be relative to the largest singular value.
\end{comment}
\end{quotation}
We tried some experiments using the ratio of smallest to largest singular values and found some sensitivities to this calculation, but it was not clear which choice resulted in more accurate simulations.
For consistency, we have retained the original calculations in both parts of the implementation.  The revised manuscript includes a footnote on \TODO{line nnn, page nnn} that describes our findings.

\begin{quotation}
\begin{comment}
\item The weighting can degrade the condition number of the least squares problem.
\end{comment}
\end{quotation}
We restrict multipliers to be at most 1024 so that the matrix equation does not become ill-conditioned.  Ill-conditioning was a reason for including the singular value test, and we have not experienced problems solving ill-conditioned matrix equations in our experiments.

\begin{quotation}
\begin{comment}
\item Is there a smoothing of hexagonal-icosahedral mesh after refinement.
\end{comment}
\end{quotation}
The hexagonal-icosahedral mesh is not smoothed but iteratively optimised.  The revised manuscript includes citations for this procedure \citep{heikes-randall1995a,heikes-randall1995b}.

\begin{quotation}
\begin{comment}
Can you comment on the projection to the tangent plane for the spherical test example. 
Note that projection to another grid can change the interpolation/approximation essentially, 
like the transformation of a boundary following grid in physical space to a Cartesian grid in 
computational space. Why is the interpolation/approximation not done in 3 dimensional 
physical space.
\end{comment}
\end{quotation}
The tangent-plane projection has been used in existing method-of-lines transport schemes \citep{lashley2002,skamarock-menchaca2010,skamarock-gassmann2011}, and these studies found that accuracy was largely insensitive to the details of the projection method.  The revised manuscript clarifies our implementation, saying, ``\revthree{cell centres are projected perpendicular to a tangent plane is defined at the face centre}'' (\TODO{line nnn, page nnn}).

If the approximation was performed in 3D then polynomial terms involving $x$, $y$ and $z$ would be required.  It would no longer be possible to fit higher-order terms with the same, compact stencil because the polynomial would have to include more terms in all three dimensions.

\section*{References}
\bibliographystyle{elsarticle-num}
\bibliography{references}
\end{document}
