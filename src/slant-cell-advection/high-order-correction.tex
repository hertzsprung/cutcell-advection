\documentclass{article}
\usepackage{fullpage}
\usepackage{amsmath}
\usepackage{tikz}
\usepackage{bm}
\usepackage{natbib}
\usepackage[hidelinks]{hyperref}
\usepackage{siunitx}
\usepackage{doi}

\newcommand{\Co}{C}
\newcommand{\vect}{\bm}
\newcommand{\iu}{{i\mkern1mu}}
\newcommand{\TODO}[1]{\textcolor{purple}{TODO: \emph{#1}}}

\title{A high-order correction to the one-dimensional cubicFit transport scheme}
\author{James Shaw}

\begin{document}
\maketitle

A transport scheme is `super-convergent' when its order of convergence is higher on uniform meshes than on non-uniform meshes.
For example, the transport scheme by \citet{skamarock-gassmann2011} is super-convergent because it is first-order on non-uniform meshes and third-order on uniform meshes.
Without a high-order correction, the one-dimensional cubicFit transport scheme is not super-convergent because it is second-order convergent on both uniform and non-uniform meshes.

Here I describe a correction cubicFit that results in fourth-order convergence on uniform meshes.  The correction technique that I use is inspired by the Taylor series expansion used by \citet{skamarock-gassmann2011}.
The corrected cubic scheme retains second-order convergence on non-uniform meshes with improved absolute accuracy compared to the uncorrected scheme.

The one-dimensional linear transport of a dependent variable $\phi$ is given by
\begin{align}
	\frac{\partial \phi}{\partial t} = - u \frac{\partial \phi}{\partial x} \label{eqn:transport}
\end{align}
where $u$ is a constant, positive velocity.
The term on the right-hand side of equation~\eqref{eqn:transport} is called the flux divergence.
The finite volume method offers one way to discretise the flux divergence by considering flux across faces of a cell,
\begin{align}
	- u \frac{\partial \phi}{\Delta x} \approx - u \frac{\phi_R - \phi_L}{\Delta x} \label{eqn:fluxdiv}
\end{align}
where $\phi_L$ and $\phi_R$ are approximate values of $\phi$ at the left and right faces respectively, and $\Delta x$ is the distance between the faces.  
The cubicFit scheme is used to approximate face values $\phi_L$ and $\phi_R$ from surrounding cell centre values.  In one dimension, the cubicFit scheme exactly interpolates the value of a dependent variable $\phi$ at face $f$ using the neighbouring downwind and three upwind cell centre values.  This arrangement is shown in figure~\ref{fig:cubicFit}.
The one-dimensional cubic interpolation is
\begin{align}
	\phi = a_1 + a_2 x + a_3 x^2 + a_4 x^3 \text{.} \label{eqn:cubic}
\end{align}
Assuming a uniform mesh with $\Delta x = 1$ and choosing the position of $\phi_{i+1/2}$ to be $x=0$ I evaluate equation~\eqref{eqn:cubic} at the cell centres $\phi_{i-2}, \ldots, \phi_{i+1}$ to form the matrix equation
\begin{align}
	\mathbf{B} \mathbf{a} &= \bm{\phi} \\
	\begin{bmatrix}
		1 & -5/2 & 25/4 & -125/8 \\
		1 & -3/2 &  9/4 & -27/8 \\
		1 & -1/2 &  1/4 &  -1/8 \\
		1 &  1/2 &  1/4 &   1/8 \\
	\end{bmatrix}
	\begin{bmatrix}
		a_1 \\
		a_2 \\
		a_3 \\
		a_4
	\end{bmatrix}
	&=
	\begin{bmatrix}
		\phi_{i-2} \\
		\phi_{i-1} \\
		\phi_i \\
		\phi_{i+1}
	\end{bmatrix} \text{.}
\end{align}
The unknown coefficients $\mathbf{a}$ are found by inverting $\mathbf{B}$ such that $\mathbf{a} = \mathbf{B}^{-1} \bm{\phi}$.  The inverse matrix is
\begin{align}
	\mathbf{B}^{-1} = 
	\frac{1}{48}
	\begin{bmatrix}
		3 & -15 & 45 & 15 \\
		2 & -6 & -42 & 46 \\
		-12 & 60 & -84 & 36 \\
		-8 & 24 & -24 & 8
	\end{bmatrix}
\end{align}
Since I chose $x=0$ to be the position $i+1/2$ then 
\begin{align}
	\phi_{i+1/2} = a_1 = 
	\frac{1}{16}
	\begin{bmatrix}
		1 \\ -5 \\ 15 \\ 5
	\end{bmatrix}
	\cdot
	\begin{bmatrix}
		\phi_{i-2} \\
		\phi_{i-1} \\
		\phi_i \\
		\phi{i+1}
	\end{bmatrix} \text{.}
\end{align}
Using this weighted sum for both left and right faces on the uniform mesh with $\Delta x = 1$, the flux divergence in equation~\eqref{eqn:fluxdiv} becomes
\begin{align}
	- u \frac{\phi_R - \phi_L}{\Delta x}
	&=
	- u \left(\phi_{i+1/2} - \phi_{i-1/2} \right) \\
	&=
	- u \cdot \frac{1}{16} \left( 
	\begin{bmatrix}
		0 \\ 1 \\ -5 \\ 15 \\ 5
	\end{bmatrix}
	-
	\begin{bmatrix}
		1 \\ -5 \\ 15 \\ 5 \\ 0
	\end{bmatrix}
	\right)
	\cdot 
	\begin{bmatrix}
		\phi_{i-3} \\
		\phi_{i-2} \\
		\phi_{i-1} \\
		\phi_i \\
		\phi{i+1}
	\end{bmatrix} \\
	&=
	- u \cdot \frac{1}{16}
	\begin{bmatrix}
		-1 \\ 6 \\ -20 \\ 10 \\ 5
	\end{bmatrix}
	\cdot 
	\begin{bmatrix}
		\phi_{i-3} \\
		\phi_{i-2} \\
		\phi_{i-1} \\
		\phi_i \\
		\phi{i+1}
	\end{bmatrix} \text{.}
\end{align}

The finite difference method offers another way to discretise the flux divergence of cell $i$ with a cubic approximation using cell centre values $\phi_{i-2}, \ldots, \phi_{i+1}$.  A matrix equation is constructed using equation~\eqref{eqn:cubic} evaluated at every cell centre.  For convenience, assume that $\Delta x = 1$ and that $x=0$ at the cell centre position of $\phi_i$, hence the matrix equation becomes
\begin{align}
	\mathbf{B} \mathbf{a} &= \bm{\phi} \\
	\begin{bmatrix}
		1 & -2 & 4 & -8 \\
		1 & -1 & 1 & -1 \\
		1 &  0 & 0 &  0\\
		1 &  1 & 1 &  1\\
	\end{bmatrix}
	\begin{bmatrix}
		a_1 \\
		a_2 \\
		a_3 \\
		a_4
	\end{bmatrix}
	&=
	\begin{bmatrix}
		\phi_{i-2} \\
		\phi_{i-1} \\
		\phi_i \\
		\phi_{i+1}
	\end{bmatrix}
\end{align}
The unknown coefficients $\mathbf{a}$ are found by calculating the inverse matrix,
\begin{align}
	\mathbf{B}^{-1} = 
	\frac{1}{6}
	\begin{bmatrix}
		0 & 0 & 6 & 0 \\
		1 & -6 & 3 & 2 \\
		0 & 3 & -6 & 3 \\
		-1 & 3 & -3 & 1
	\end{bmatrix}
\end{align}
To calculate the flux divergence I calculate the derivative $\partial \phi / \partial x = a_2 + 2 a_3 x + 3 a_4 x^2$.  Evaluating the flux divergence at $\phi_i$ where $x=0$ then $\partial \phi_i / \partial x = a_2$.  Hence I find that the finite difference weighted sum is
\begin{align}
	-u \frac{\partial \phi_i}{\partial x} = -u a_2 =
	-u \cdot \frac{1}{6}
	\begin{bmatrix}
		1 \\ -6 \\ 3 \\ 2
	\end{bmatrix}
	\cdot
	\begin{bmatrix}
		\phi_{i-2} \\
		\phi_{i-1} \\
		\phi_i \\
		\phi_{i+1}
	\end{bmatrix} \label{eqn:fd-fluxdiv}
\end{align}
The cubic finite difference approximation given in equation~\eqref{eqn:fd-fluxdiv} is conservative on uniform meshes.  This can be demonstrated by decomposing the weights vector,
\begin{align}
	-u \cdot \frac{1}{6}
	\begin{bmatrix}
		1 \\ -6 \\ 3 \\ 2
	\end{bmatrix}
	\cdot
	\begin{bmatrix}
		\phi_{i-2} \\
		\phi_{i-1} \\
		\phi_i \\
		\phi_{i+1}
	\end{bmatrix}
	&= 
	-u \cdot
	\frac{1}{6}
	\left(
	\begin{bmatrix}
		0 \\ -1 \\ 5 \\ 2
	\end{bmatrix}
	-
	\begin{bmatrix}
		-1 \\ 5 \\ 2 \\ 0
	\end{bmatrix}
	\right)
	\cdot
	\begin{bmatrix}
		\phi_{i-2} \\
		\phi_{i-1} \\
		\phi_i \\
		\phi_{i+1}
	\end{bmatrix} \\
	&=
	-u \left(
	\frac{1}{6}
	\begin{bmatrix}
		-1 \\ 5 \\ 2
	\end{bmatrix}
	\cdot
	\begin{bmatrix}
		\phi_{i-1} \\
		\phi_i \\
		\phi_{i+1}
	\end{bmatrix}
	-
	\frac{1}{6}
	\begin{bmatrix}
		-1 \\ 5 \\ 2
	\end{bmatrix}
	\cdot
	\begin{bmatrix}
		\phi_{i-2} \\
		\phi_{i-1} \\
		\phi_i
	\end{bmatrix} \right) \text{.}
\end{align}
Notice that the flux divergence has been rewritten as the difference between right and left fluxes (equation~\ref{eqn:fluxdiv}).

\section*{Calculating the high-order correction}

% TODO: then invert and read off the coefficients, using them in the derivative calculation
%\begin{align}
%	- u \frac{\partial \phi}{\Delta x} \approx -u \left( \right)
%\end{align}

\begin{figure}
	\centering
	\caption{\TODO{cubicFit in 1D}}
	\label{fig:cubicFit}
\end{figure}

\bibliographystyle{ametsoc2014}
\bibliography{references}

\end{document}
