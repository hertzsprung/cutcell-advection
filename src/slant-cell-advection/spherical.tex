\section*{Appendix B: Mesh geometry on a spherical Earth}

The cubicFit transport scheme is implemented using the OpenFOAM CFD library.  Unlike many atmospheric models that use spherical coordinates, OpenFOAM uses global, three-dimensional Cartesian coordinates.  In order to perform the experiments on a spherical Earth presented in section~\ref{sec:deformationSphere}, it is necessary for velocity fields and mesh geometries to be expressed in these global Cartesian coordinates.

\subsection*{Velocity field specification}
The non-divergent velocity field in section~\ref{sec:deformationSphere} is specified as a streamfunction $\Psi(\lambda, \theta)$.  Instead of calculating velocity vectors, the flux $\mathbf{u}_f \cdot \mathbf{S}_f$ through a face $f$ is calculated directly from the streamfunction,
\begin{align}
	\mathbf{u}_f \cdot \mathbf{S}_f	= \sum_{e\:\in\:f} \mathbf{e} \cdot \mathbf{x}_e \Psi(e) \label{eqn:nondiv-spherical-flux}
\end{align}
where $e \in f$ denotes the edges $e$ of face $f$, $\mathbf{e}$ is the edge vector joining its two vertices, $\mathbf{x}_e$ is the position vector of the edge midpoint, and $\Psi(e)$ is the streamfunction evaluated at the same position.
Edge vectors are directed in a counter-clockwise orientation.

\subsection*{Spherical mesh construction}

Since OpenFOAM does not support two-dimensional spherical meshes, instead, we construct meshes that have a single layer of cells that are \SI{2000}{\meter} deep, having an inner radius $r_1 = R_e - \SI{1000}{\meter}$ and an outer radius $r_2 = R_e + \SI{1000}{\meter}$.
By default, OpenFOAM meshes comprise polyhedral cells with straight edges and flat faces.  This is problematic for spherical meshes because face areas and cell volumes are too small.
For tests on a spherical Earth, we override the default configuration and calculate our own face areas, cell volumes, face centres and cell centres that account for the spherical geometry.  

A face is assumed to be either a surface face or radial face.
A surface face has any number of vertices, all of equal radius.
A radial faces have four vertices with two different radii, $r_1$ and $r_2$, and two different horizontal coordinates, $(\lambda_1, \theta_1)$ and $(\lambda_2, \theta_2)$.
A radial face centre is modified so that it has a radius $R_e$.  The latitudinal and longitudinal components of a radial face centre need no modification.
The face area $A_f$ for a radial face $f$ is the area of the annular sector,
\begin{align}
	A_f = \frac{d}{2} \left\lvert r_2^2 - r_1^2 \right\rvert
\end{align}
where $d$ is the great-circle distance between $(\lambda_1, \theta_1)$ and $(\lambda_2, \theta_2)$.

To calculate the centre of a surface face $f$, a new vertex is created that is positioned at the mean of the face vertices.  Note that this centre position, $\mathbf{\tilde{c}}_f$, is used in intermediate calculations and it is not the face centre position.
Next, the surface face is subdivided into spherical triangles that share this new vertex.
The face centre direction and radius are calculated separately.  The face centre direction $\mathbf{\hat{r}}$ is the mean of the spherical triangle centres weighted by their solid angle
\begin{align}
	\mathbf{\hat{r}} = \frac
	{\sum_{t\:\in\:f}{\Omega_t \left(\mathbf{x}_{t,1} + \mathbf{x}_{t,2} + \mathbf{\tilde{c}}_f \right)}}
	{\left\lvert \sum_{t\:\in\:f}{\Omega_t \left(\mathbf{x}_{t,1} + \mathbf{x}_{t,2} + \mathbf{\tilde{c}}_f \right)} \right\rvert} \label{eqn:face-centre-dir}
\end{align}
where $t\in f$ denotes the spherical triangles $t$ of face $f$, $\Omega_t$ is spherical triangle's solid angle which is calculated using l'Huilier's theorem, $\mathbf{x}_{t,1}$ and $\mathbf{x}_{t,2}$ are the positions of the vertices shared by the face $f$ and spherical triangle $t$, and $\mathbf{\tilde{c}}_f$ is the position of the centre vertex shared by all spherical triangles of face $f$.
The face centre radius $r$ is the mean radius of the face vertices, again weighted by the solid angle of each spherical triangle,
\begin{align}
	r = \frac
	{\sum_{t\in f}{\Omega_t \left(\left\lvert \mathbf{x}_{t,1} \right\rvert + \left\lvert \mathbf{x}_{t,2} \right\rvert \right)/2}}
	{\Omega_f} \label{eqn:face-centre-mag}
\end{align}
where the solid angle $\Omega_f$ of face $f$ is the sum of the solid angles of the constituent spherical triangles,
\begin{align}
	\Omega_f = \sum_{t\in f}{\Omega_t}
\end{align}
We use equations~\eqref{eqn:face-centre-dir} and \eqref{eqn:face-centre-mag} to calculate the centre $\mathbf{c}_f$ of the face,
\begin{align}
	\mathbf{c}_f = r\:\mathbf{\hat{r}} \label{eqn:face-centre}
\end{align}
The area vector $\mathbf{S}_f$ of the surface face $f$ is the sum of the spherical triangle areas \citep{vanbrummelen2013},
\begin{align}
	\mathbf{S}_f = r^2 \Omega_f \mathbf{\hat{r}}
\end{align}
Cell centres and cell volumes are corrected by considering faces that are not normal to the sphere such that
\begin{align}
	\frac{\left(\mathbf{S}_f \cdot \mathbf{c}_f\right)^2}{\left\lvert \mathbf{S}_f \right\rvert^2 \left\lvert \mathbf{c}_f \right\rvert^2} > 0 \label{eqn:surface-faces}
\end{align}
Let $\mathcal{F}$ be the set of faces satisfying equation~\eqref{eqn:surface-faces}.  Then, the cell volume $\mathcal{V}_c$ is
\begin{align}
	\mathcal{V}_c = \frac{1}{3} \sum_{f\:\in\:\mathcal{F}} \mathbf{S}_f \cdot \mathbf{c}_f
\end{align}
which can be thought of as the area $A$ integrated between $r_1$ and $r_2$ such that 
$\int_0^R{A(r)\:\mathrm{d}r} = \int_{r_1}^{r_2}{r^2 \Omega\:\mathrm{d}r} = \frac{1}{3} \Omega \left( r_2^3 - r_1^3 \right)$.
The cell centre is modified so that it has a radius $R_e$, which is consistent with radial faces.

Edges can be classified in a similar manner to faces where surface edges are tangent to the sphere and radial faces are normal to the sphere.  The edge midpoints, $\mathbf{c}_e$, are used to calculate the face flux for non-divergent velocity fields (equation~\ref{eqn:nondiv-spherical-flux}).
For transport tests, corrections to edge midpoints are unnecessary.  Due to the choice of $r_1$ and $r_2$ during mesh construction, the midpoint of a radial edge is at a radial distance of $R_e$ which is necessary for the correct calculation of non-divergent velocity fields.
The position of surface edge midpoints is unimportant because these edges do not contribute to the face flux since $\mathbf{e} \cdot \mathbf{c}_e = 0$.
Edge lengths are the straight-line distance between the two vertices and not the great-circle distance.  Again, the edge lengths are not corrected because it makes no difference to the face flux calculation.

