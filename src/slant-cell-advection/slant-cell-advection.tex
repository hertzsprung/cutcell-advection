\documentclass{article}
\usepackage{graphicx}
\usepackage{amsmath}
\usepackage{amssymb}
\usepackage{mathtools}
\usepackage{xcolor}
\usepackage{bm}

\usepackage{natbib}
\usepackage[hidelinks]{hyperref}
\usepackage{doi}
\usepackage{charter}
\usepackage[bitstream-charter]{mathdesign}
\usepackage[final,babel]{microtype}
\usepackage[utf8]{inputenc}
\usepackage[british]{babel}
\usepackage{csquotes}
\usepackage[T1]{fontenc}


\title{Locally-accurate advection over steep terrain on a new type of Cartesian mesh \\ \TODO{(working title)}}
\author{James Shaw}

\newcommand{\iunit}{\boldsymbol{\hat \imath}}
\newcommand{\junit}{\boldsymbol{\hat \jmath}}
\newcommand{\kunit}{\boldsymbol{\hat k}}
\newcommand{\TODO}[1]{\textcolor{purple}{TODO: \emph{#1}}}

\begin{document}
\maketitle

\section{Introduction}

Increasing spatial resolution means that numerical weather prediction models are able to resolve smaller-scale, steep slopes in terrain.  A large majority of operational models use terrain-following coordinates to represent orography, but such models can suffer from increased numerical errors near steep slopes where the coordinate transformation becomes almost singular \TODO{citations}.  Such singularities can be avoided by using the cut-cell method, which has been proposed as an alternative to terrain-following coordinates.  The cut-cell method represents terrain using piecewise polynomial functions that intersect with an orthogonal, Cartesian mesh.  Several studies of idealised flows have found that cut-cell models produce results with an accuracy that is comparable to, or better than, terrain-following models \citep{yamazaki-satomura2008,good2014,shaw-weller2016}.

\TODO{however, cut cells bring their own problems: the small cell problem, and loss of accuracy near the lower boundary.  this paper is going to address these two issues\ldots hopefully!}

%Using the horizontal advection test case by \citet{schaer2002}, \citet{good2014} found that advection errors were smaller

The cut cell method intersects a terrain surface with an orthogonal, Cartesian mesh.
Cells that are entirely above ground are unaltered, and cells that lie entirely beneath the ground are removed.
Cells that are intersected by the terrain are called `cut cells'.  The terrain within each cut cell is represented by a polynomial surface.  This surface is is typically bilinear, and interpolates the four terrain elevations which are known at the cell edges (Fig.~\ref{fig:cut-cell}).  Higher order polynomials have also been tried, e.g. \citet{kirkpatrick2003}.

\TODO{I suppose I should try to clear up the terminology mess: cut cells, shaved cells, immersed boundaries, embedded boundaries -- there's some literature that tries to clarify this, can't remember what, but I should read that first}

\begin{figure}
	\caption{\TODO{an example diagram of a 3D cell with a bilinear terrain surface, similar to Fig 4(top) in \cite{walko-avissar2008b}, highlighting face areas open to the air}}
	\label{fig:cut-cell}
\end{figure}

Explicit, finite volume advection schemes must satisfy the Courant--Friedrichs--Lewy (CFL) criterion in order to be stable.  The multidimensional CFL criterion constrains the timestep, $\Delta t$, such that
\begin{align}
	\Delta t \leq \min_\mathrm{cells} \frac{2\mathcal{V}}{\sum_\mathrm{faces} F}
\end{align}
where $\mathcal{V}$ is the volume of a cell and $F$ is the flux across a face.  Hence, the maximum timestep is constrained by the largest fluxes through the smallest cells.  \TODO{we need to explain this further because it's important that we understand about large fluxes across small cell dimensions, but maybe that can wait until later.}
The cut cell method can create arbitrarily small cells that impose severe constraints on the timestep.

There are a variety of techniques that alleviate the timestep constraint for cut cell meshes.  One technique is to combine small cells with neighbouring cells, so that the smallest cell volume and, hence, the maximum timestep, is increased.  \citet{ye1999} identified cells with cell centres below ground, and combined each of them with the cell above.  They applied the technique to a two-dimensional model of viscous incompressible flow.  The technique ensures that the smallest cell volume is at least half the volume of an uncut cell, hence the maximum timestep is also constrained to be at least half the maximum timestep for the equivalent mesh with no cut cells.
\citet{yamazaki2016} extended this cell-combining technique to a three-dimensional, fully-compressible, non-hydrostatic atmospheric model, in which small cells are combined vertically or horizontally depending on the steepness of the terrain slope in the small cell.  \TODO{and? so what?}

\TODO{cell combining has disadvantages: velocity points may not have pressure points on both sides \citep{kirkpatrick2003}, so it is no longer possible to separate horizontal and vertical pressure gradient calculations \citep{walko-avissar2008b}.  \citet{kirkpatrick2003} also says that it is difficult to extend cell merging techniques to three dimensions, but \citet{yamazaki2016} shows this can be done reasonably straightforwardly.}

Another technique that alleviates the timestep constraint is the thin-wall approximation, in which the computational volume of a cell unaltered when it is cut.  In other words, a cut cell is assigned a computational volume equal to its original, uncut, physical volume.  The area of a cell face is calculated as the area unoccupied by solid terrain.  In the thin-wall approximation, a cut cell can be conceptualised as a cardboard box without a lid and with side walls that are infinitesimally thin.  Straight lines are cut along the side walls so that the box's upper edges lie along the bilinear terrain surface.  \TODO{I might need another figure to show the thin-wall approximation, or I might be able to refer back to Fig.~\ref{fig:cut-cell}.}

By using computational volumes that are equally large for cut and uncut cells, the thin-wall approximation imposes no additional timestep constraints compared to an equivalent Cartesian mesh with no cut cells.  The cell-combination technique still constrains the timestep because smaller cell volumes still remain where cells are not combined.  Hence, the thin-wall technique should permit timesteps up to twice as large as those permitted by the cell-combination technique.

\TODO{but thin-wall approximation also has difficulties: \citet{yamazaki-satomura2010} talk about ``internal inconsistency'' (presumably they mean that cell volumes are inconsistent with face areas, but I'll email Hiroe and check this).  \citet{walko-avissar2008b} provide a nice example of a thin-wall problem, possibly adapted from \citet{calhoun-leveque2000}, describing the retardation of a cold front in uniform horizontal flow along a flat surface that includes long, thin, rectangular cut cells.}

\TODO{as a slight aside, \citet{calhoun-leveque2000} talk about using ``capacity functions'' and talk about flow through porous media, but this technique seems to be somewhat (if not entirely) equivalent to the thin-wall approximation}

\TODO{I should also discuss the partially implicit method of \citet{jebens2011}.  And perhaps other methods that I have yet to satisfactorily understand: a volume-of-fluid approach by \citet{almgren1997}, and the master--slave cell-linking approach by \citet{kirkpatrick2003}.}

\TODO{what about existing work on unstructured/boundary fitted meshes?  \citet{almgren1997} has masses of citations for loads of different mesh techniques.}

\vspace*{2em}


\TODO{The next topics to discuss are: loss of accuracy at the lower boundary (2nd order schemes often become only 1st-order accurate locally), then how our slanted cell mesh technique and \texttt{cubicUpwindCPCFit} advection scheme might be able to increase timesteps and increase local accuracy\ldots maybe!}

\section{Mesh generation}
\TODO{describe how we construct a slanted cell mesh.  at some point in the paper we must justify the choice of 'tolerance', which was set to $2\Delta z/5$ in \citet{shaw-weller2016} without much justification.}

\TODO{although we are only considering 2D idealised tests in this paper, it would be nice to show that our method extends to 3D -- why don't we try it out with some real digital surface model data?}

\section{Multidimensional advection scheme}
\TODO{describe the advection scheme}

\section{Results}
\TODO{use the thermal advection test from \citet{shaw-weller2016}, but with steeper slopes.  compare with BTF meshes and cut cell meshes created using the ASAM mesh generator.  use Co close to 1 and report on the timesteps for each type of mesh.}

\TODO{report $\ell_2$ error norms, min/max values for BTF, slantedCell and cutCell meshes}

\TODO{I haven't thought much about this yet, but there's a lot of literature that says ``2nd-order schemes become 1st-order at the lower boundary'', and talk about global accuracy versus local accuracy.  I'm not sure how to measure local order of accuracy, but I think I should examine order-of-accuracy for \texttt{cubicUpwindCPCFit}.  The work by \citet{ye1999} is quite similar to ours: they use polynomial stencils to interpolate onto boundaries, and test an explicit advection scheme which they show is 2nd order globally and locally.}

\TODO{should we compare \texttt{cubicUpwindCPCFit} with a more standard scheme?  such as QUICK, maybe?  because it's well-known, fairly high-order, implemented in OpenFOAM, \citet{kirkpatrick2003} uses it\ldots  lower order schemes are more likely to be too diffusive or just unstable, especially on BTF meshes}

\TODO{I don't think I want to go beyond advection scheme/advection tests in this paper because I think it will lose its focus.  that said, I have had some very good results of resting atmosphere on the slanted cell mesh compared to cut cell and BTF meshes, and it might be interesting to see how steep we can go, and what happens to the spurious circulations\ldots}

\section{Conclusions}


\section{Acknowledgements}
\TODO{Supervisors, funding bodies.  ASAM group for the mesh generator---I should ask permission to use cut cell meshes in this paper}

\bibliographystyle{ametsoc2014}
\bibliography{references}

\end{document}
