\subsection{Transport over a mountainous lower boundary}
\label{sec:mountainAdvection}

The horizontal transport test in the previous section is useful for assessing numerical accuracy on terrain-following meshes, but it presents no particular challenge on cut cell meshes because there is no flow through the distorted cut cells near the ground \citep{good2014}.
Here we present another variant of the standard horizontal transport test that challenges transport schemes on all mesh types.  By positioning the tracer next to the ground and modifying the velocity field, we can assess the accuracy of the cubicFit scheme near the lower boundary.  Results using the cubicFit scheme are compared with the linearUpwind scheme on basic terrain-following, cut cell and slanted cell meshes.


Cut cell meshes are constructed using the ASAM grid generator \citep{jaehn2015,asam2010}.  Slanted cell meshes are constructed following the approach by Shaw and Weller \citep{shaw-weller2016}: vertices that are underground are moved up to the surface and zero-area faces and zero-volume cells are removed.  Unlike \citep{shaw-weller2016}, vertices are never moved downwards.

Cell edges in the central region of the domain are shown in figure~\ref{fig:mountainAdvection-meshes} for each of the three mesh types.
Cells in the BTF mesh are highly distorted over steep slopes (figure~\ref{fig:mountainAdvection-meshes}a) while the cut cell mesh (figure~\ref{fig:mountainAdvection-meshes}b) and slanted cell mesh (figure~\ref{fig:mountainAdvection-meshes}c) are orthogonal everywhere except for cells nearest the ground.

\begin{figure}
	\centering
	\includegraphics{../fig-mountainAdvection-meshes/fig-mountainAdvection-meshes.pdf}
	\caption{Two dimensional $x$-$z$ meshes created with the (a) basic terrain-following, (b) cut cell, and (c) slanted cell methods, and used for the tracer transport tests in section~\ref{sec:mountainAdvection}.  Cell edges are marked by thin black lines.  The peak mountain height $h_0 = \SI{5}{\kilo\meter}$.
The velocity field is the same for all mesh types with streamlines marked on each panel by thick red lines.  The velocity field (equation~\ref{eqn:streamfunc-btf}) follows the lower boundary and becomes entirely horizontal above $H_1 = \SI{10}{\kilo\meter}$.
Only the lowest \SI{10}{\kilo\meter} for the central region of the domain in shown.  The entire domain is \SI{301}{\kilo\meter} wide and \SI{25}{\kilo\meter} high.}
	\label{fig:mountainAdvection-meshes}
\end{figure}

Similar to the approach by \citep{shaw-weller2016}, a velocity field is chosen that follows the terrain at the surface and becomes entirely horizontal aloft.
A streamfunction $\Psi$ is used so that the discrete velocity field is non-divergent, such that
\begin{equation}
	\Psi(x,z) = -u_0 H_1 \frac{z - h}{H_1 - h} \label{eqn:streamfunc-btf}
\end{equation}
where $u_0 = \SI{10}{\meter\per\second}$, which is the horizontal velocity where $h(x) = 0$.
There is no normal flow at the lower boundary, and the velocity field becomes purely horizontal above $H_1 = \SI{10}{\kilo\meter}$.  Elsewhere, the flow is predominantely horizontal, with non-zero vertical velocities only above sloping terrain.
The horizontal and vertical components of velocity, $u$ and $w$, are given by
\begin{align}
	u &= -\frac{\partial \Psi}{\partial z} = u_0 \frac{H_1}{H_1 - h}, \quad w = \frac{\partial \Psi}{\partial x} = u_0 H_1 \frac{\mathrm{d} h}{\mathrm{d} x} \frac{H_1 - z}{\left( H_1 - h \right)^2} \label{eqn:uw-btf} \text{ ,}\\
	\frac{\mathrm{d} h}{\mathrm{d} x} &= - h_0 \left[ 
		\beta \cos^2 \left( \alpha x \right) \sin \left( 2 \beta x \right) +
		\alpha \cos^2 \left( \beta x \right) \sin \left( 2 \alpha x \right)
	\right] \text{ .}
\end{align}
Unlike the horizontal transport test in \citep{schaer2002}, the velocity field presented here extends from the top of the domain all the way to the ground.

The flow is deliberately misaligned with the BTF, cut cell and slanted cell meshes away from the ground (figure~\ref{fig:mountainAdvection-meshes}) to ensure that flow always crosses mesh surfaces in order to challenge the transport scheme.
The value of $H_1$ is chosen to be much smaller than the domain height $H$ in equation~\eqref{eqn:btf} so that flow crosses the surfaces of the BTF mesh.
This is evident in figure~\ref{fig:mountainAdvection-meshes}a where the the velocity streamlines are tangential to the mesh only at the ground.


\begin{table}
	\centering
\begin{tabular}{l S S S S S}
\hline
	& \multicolumn{5}{c}{Peak mountain height $h_0$ (\si{\kilo\meter})} \\
	Mesh type & 0 & 3 & 4 & 5 & 6 \\
\hline
	BTF & 40 & 16 & 10 & 8 & 5 \\
	Cut cell & 40 & 1.6 & 1.6 & 0.5 & 1.5  \\
	Slanted cell & 40 & 8 & 6.25 & 5 & 4  \\
\hline
\end{tabular}
%
	\caption{Time-steps (\si{\second}) for the two-dimensional transport test over a mountainous lower boundary.  The time-steps were chosen so that the maximum Courant number was between \num{0.36} and \num{0.46}.}
	\label{tab:mountainAdvection:timesteps}
\end{table}

The tracer is again defined by equation~\eqref{eqn:tracer} but is now positioned at the ground with $(x_0, z_0) = (\SI{-50}{\kilo\meter}, \SI{0}{\kilo\meter})$ with half-widths $A_x = \SI{25}{\kilo\meter}$ and $A_z = \SI{10}{\kilo\meter}$.
Tests are integrated forward for \SI{10000}{\second}.  The time-step was chosen for each mesh so that the maximum Courant number was about \num{0.4} (table~\ref{tab:mountainAdvection:timesteps}).
An analytic solution at \SI{10000}{\second} is obtained by calculating the new horizontal position of the tracer.  Integrating along the trajectory yields $t$, the time taken to move from the left side of the mountain to the right \citep{shaw-weller2016}:
\begin{align}
	\mathrm{d}t &= \mathrm{d}x / u(x) \\
	t &= \int_0^x \frac{H - h(x)}{u_0 H}\:\mathrm{d}x \\
	t &= \frac{x}{u_0} - \frac{h_0}{16 u_0 H} \left[ 4x + \frac{\sin 2 (\alpha + \beta) x}{\alpha + \beta} 
 \frac{\sin 2(\alpha - \beta) x}{\alpha - \beta} + 2 \left( \frac{\sin 2\alpha x}{\alpha} + \frac{\sin 2\beta x}{\beta} \right) \right]
\end{align}
By solving this equation we find that \(x(t=\SI{10000}{\second}) = \SI{54342.8}{\meter}\) when $h_0 = \SI{5}{\kilo\meter}$.

The tracer density boundary conditions are the same as those in the previous test.
Since the cubicFit transport scheme uses values at boundaries with Dirichlet boundary conditions, the cubicFit scheme uses only inlet boundary values in this test case.

\begin{figure}
	\centering
	\includegraphics{../fig-mountainAdvection-tracer/fig-mountainAdvection-tracer.pdf}
	\caption{Evolution of the tracer in the two-dimensional transport test over a mountainous lower boundary.  The tracer is transported to the right over the wave-shaped terrain.  Tracer contours are every \SI{0.1}{\kilo\gram\per\meter\cubed}.  The result obtained using the cubicFit scheme on the basic terrain-following mesh is shown at $t=\SI{0}{\second}$, $t=\SI{5000}{\second}$ and $t=\SI{10000}{\second}$ with solid black contours. The analytic solution at $t=\SI{10000}{\second}$ is shown with dotted contours.
	The shaded box indicates the region that is plotted in figure~\ref{fig:mountainAdvection-errors}.}
	\label{fig:mountainAdvection-tracer}
\end{figure}

Three series of tests were performed using similar configurations.  The first series uses a peak mountain height of $h_0 = \SI{5}{\kilo\meter}$ to examine errors on different mesh types using the two transport schemes.  The second series varies the peak mountain height to examine the sensitivity of the transport schemes to mesh distortions.  The third series verifies accuracy at Courant numbers close to 1 and examines the longest stable time-step for different mesh types.

For the first series of tests with $h_0 = \SI{5}{\kilo\meter}$, tracer contours at the initial time $t=\SI{0}{\second}$, half-way time $t=\SI{5000}{\second}$, and end time $t=\SI{10000}{\second}$ are shown in figure~\ref{fig:mountainAdvection-tracer} using the cubicFit scheme on the BTF mesh.  As apparent at $t=\SI{5000}{\second}$, the tracer is distorted by the terrain-following velocity field as it passes over the mountain, but its original shape is restored once it has cleared the mountain by $t=\SI{10000}{\second}$.
A small phase lag is apparent when the numerical solution marked with solid contour lines is compared with the analytic solution marked with dotted contour lines.

\begin{figure}
	\centering
	\includegraphics{../fig-mountainAdvection-error/fig-mountainAdvection-error.pdf}
	\caption{Tracer contours at $t=\SI{10000}{\second}$ for the two-dimensional tracer transport tests over a mountainous lower boundary.  A region in the lee of the mountain is plotted corresponding to the shaded area in figure~\ref{fig:mountainAdvection-tracer}.  Results are presented on BTF, cut cell and slanted cell meshes (shown in figure~\ref{fig:mountainAdvection-meshes}) using the linearUpwind and cubicFit transport schemes.  The numerical solutions are marked by solid black lines.  The analytic solution is marked by dotted lines.  Contours are every \SI{0.1}{\kilo\gram\per\meter\cubed}.}
	\label{fig:mountainAdvection-errors}
\end{figure}

Numerical errors are more clearly revealed by subtracting the analytic solution from the numerical solution.
Error fields are compared between BTF, cut cell and slanted cell meshes using the linearUpwind scheme (figures~\ref{fig:mountainAdvection-errors}a, \ref{fig:mountainAdvection-errors}b and \ref{fig:mountainAdvection-errors}c respectively) and the cubicFit scheme (figures~\ref{fig:mountainAdvection-errors}d, \ref{fig:mountainAdvection-errors}e and \ref{fig:mountainAdvection-errors}f respectively).
Results are least accurate using the linearUpwind scheme on the slanted cell mesh (figure~\ref{fig:mountainAdvection-errors}c).  The final tracer is slightly distorted and does not extend far enough towards the ground.
The $\ell_\infty$ error magnitude is reduced by using the linearUpwind scheme on the cut cell mesh (figure~\ref{fig:mountainAdvection-errors}b), but the shape of the error remains the same.
The cubicFit scheme is less sensitive to the choice of mesh with similar error magnitudes on the BTF mesh (figure~\ref{fig:mountainAdvection-errors}d), cut cell mesh (figure~\ref{fig:mountainAdvection-errors}e) and slanted cell mesh (figure~\ref{fig:mountainAdvection-errors}f).  Errors using the cubicFit scheme on cut cell and slanted cell meshes are much smaller than the errors using the linearUpwind scheme on the same meshes.

\begin{figure}
	\centering
	\includegraphics{../fig-mountainAdvection-l2ByMountainHeight/fig-mountainAdvection-l2ByMountainHeight.pdf}
%
	\caption{Error measures for the two-dimensional tracer transport tests over a mountainous lower boundary.  Peak mountain heights $h_0$ are from \SIrange{0}{6}{\kilo\meter}.  Results are compared on BTF, cut cell and slanted cell meshes using the linearUpwind and the cubicFit schemes.  At $h_0 = \SI{0}{\kilo\meter}$ the terrain is entirely flat and the BTF, cut cell and slanted cell meshes are identical.  At $h_0 = \SI{6}{\kilo\meter}$ the linearUpwind scheme is unstable on the slanted cell mesh.}
	\label{fig:mountainAdvection-l2ByMountainHeight}
\end{figure}

To further examine the performance of the cubicFit scheme in the presence of steep terrain, a second series of tests were performed in which the peak mountain height was varied from \SIrange{0}{6}{\kilo\meter} keeping all other parameters constant.
Results were obtained on BTF, cut cell and slanted cell meshes using the linearUpwind scheme and cubicFit scheme.  Again, the time-step was chosen for each test so that the maximum Courant number was about \num{0.4} (table~\ref{tab:mountainAdvection:timesteps}).  The $\ell_2$ error was calculated by subtracting the analytic solution from the numerical solution (figure~\ref{fig:mountainAdvection-l2ByMountainHeight}).
Note that the analytic solution is a function of mountain height, with the tracer travelling farther over higher mountains due to non-divergent flow through a narrower channel.
In all cases, error increases with increasing mountain height because steeper slopes lead to greater mesh distortions.
Errors are identical for a given transport scheme when $h_0 = \SI{0}{\kilo\meter}$ and the ground is entirely flat because the BTF, cut cell and slanted cell meshes are identical.
Compared with the cubicFit scheme, the linearUpwind scheme is more sensitive to the mesh type and mountain height.  The linearUpwind scheme is unstable on the slanted cell mesh with a peak mountain height $h_0 = \SI{6}{\kilo\meter}$ despite using a Courant number of \num{0.428}.
In contrast, the cubicFit scheme is less sensitive to the mesh type and errors grow more slowly with increasing mountain height.  The cubicFit scheme yields stable results in all tests.

\begin{figure}
	\centering
	\includegraphics{../fig-mountainAdvection-maxdt/fig-mountainAdvection-maxdt.pdf}
	\caption{Longest stable time-steps, $\Delta t_\mathrm{max}$, for the two-dimensional tracer transport test over a mountainous lower boundary.  Results were obtained on basic terrain-following, cut cell and slanted cell meshes at mesh spacings between $\Delta x = \SI{5000}{\meter}$ and $\Delta x = \SI{125}{\meter}$.  The tests were integrated with a maximum Courant number close to 1, while $\Delta t_\mathrm{max}$ is calculated as the time-step corresponding to a maximum Courant number of exactly 1.}
	\label{fig:mountainAdvection-maxdt}
\end{figure}

A final series of tests were performed on BTF, slanted cell and cut cell meshes using the cubicFit scheme with a variety of mesh spacings between $\Delta x = \SI{5000}{\meter}$ and $\Delta x = \SI{125}{\meter}$.  $\Delta z$ was chosen so that a constant aspect ratio is preserved such that $\Delta x / \Delta z = 2$.  In order to verify that cubicFit is accurate near the stability limit, time-steps were chosen so that the maximum Courant number was close to, but smaller than, one.  Stable results were obtained in all tests and the cubic scheme was largely insensitive to the choice of time-step.

This series of tests also enables a comparison of longest stable time-steps between mesh types.  The longest stable time-step for a maximum Courant number of one can be calculated as $\Delta t_\mathrm{max} = \Delta t / \max(\mathrm{Co})$ where $\Delta t$ is the time-step used in a particular test run and $\max(\mathrm{Co})$ is the maximum Courant number for that test run.
The longest stable time-steps for BTF, cut cell and slanted cell meshes are presented in figure~\ref{fig:mountainAdvection-maxdt}.  BTF meshes permit the longest time-steps of all three mesh types since cells are almost uniform in volume.  As expected, the longest stable time-step scales linearly with BTF mesh spacing except at the coarsest resolutions where the mountains are poorly represented.
There is no such linear scaling on cut cell meshes because these meshes can have arbitrarily small cells.  The time-step constraints on cut cell meshes are the most severe of the three mesh types.  Slanted cell meshes have a slightly stronger time-step constraint than BTF meshes but still exhibit similar linear scaling with mesh spacing.  Furthermore, a dynamical model that uses slanted cell meshes instead of BTF meshes is expected to calculate pressure gradients more accurately \citep{shaw-weller2016}.

The transport tests presented in this section demonstrate that the cubicFit scheme is suitable for flows over very steep terrain on two-dimensional terrain-following, cut cell and slanted cell meshes.  The cubicFit scheme is less sensitive to the mesh type and mountain steepness compared to the linearUpwind scheme.  The linearUpwind scheme becomes unstable over very steep slopes but the cubicFit scheme is stable for all tests.  In the next section, we evaluate the cubicFit scheme using more complex, deformational flows on icosahedral meshes and cubed-sphere meshes.
